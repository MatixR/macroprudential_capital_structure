% Copyright 2004 by Till Tantau <tantau@users.sourceforge.net>.
%
% In principle, this file can be redistributed and/or modified under
% the terms of the GNU Public License, version 2.
%
% However, this file is supposed to be a template to be modified
% for your own needs. For this reason, if you use this file as a
% template and not specifically distribute it as part of a another
% package/program, I grant the extra permission to freely copy and
% modify this file as you see fit and even to delete this copyright
% notice. 

\documentclass{beamer}
% Replace the \documentclass declaration above
% with the following two lines to typeset your 
% lecture notes as a handout:
%\documentclass{article}
%\usepackage{beamerarticle}
\usepackage{graphicx}
\usepackage{natbib}
\usepackage{sansmathaccent,longtable,booktabs,threeparttable,afterpage,mathtools,anyfontsize}
\pdfmapfile{+sansmathaccent.map}
% There are many different themes available for Beamer. A comprehensive
% list with examples is given here:
% http://deic.uab.es/~iblanes/beamer_gallery/index_by_theme.html
% You can uncomment the themes below if you would like to use a different
% one:
%\usetheme{AnnArbor}
%\usetheme{Antibes}
%\usetheme{Bergen}
%\usetheme{Berkeley}
%\usetheme{Berlin}
%\usetheme{Boadilla}
%\usetheme{boxes}
%\usetheme{CambridgeUS}
%\usetheme{Copenhagen}
%\usetheme{Darmstadt}
%\usetheme{default}
%\usetheme{Frankfurt}
%\usetheme{Goettingen}
%\usetheme{Hannover}
%\usetheme{Ilmenau}
%\usetheme{JuanLesPins}
%\usetheme{Luebeck}
%\usetheme{Madrid}
%\usetheme{Malmoe}
%\usetheme{Marburg}
%\usetheme{Montpellier}
%\usetheme{PaloAlto}
%\usetheme{Pittsburgh}
%\usetheme{Rochester}
%\usetheme{Singapore}
%\usetheme{Szeged}
%\usetheme{Warsaw}

%***************************************************
% Fix input with \midrule problem
%***************************************************

\newcommand{\sym}[1]{\rlap{#1}}% Thanks to David Carlisle

\newcommand*{\captionsource}[2]{%
	\caption[{#1}]{%
		#1%
		\\\hspace{\linewidth}%
		\textbf{Source:} #2%
	}%
}
\newcommand{\horrule}[1]{\rule{\linewidth}{#1}} % Create horizontal rule command with 1 argument of height

%***************************************************
% Fix input with \midrule problem
%***************************************************

\makeatletter
\newcommand\primitiveinput[1]
{\@@input #1 }
\makeatother

\title{The Impact of Bank Regulation on Firms' Capital Structure: Evidence from Multinationals}

% A subtitle is optional and this may be deleted
%\subtitle{Gauti B. Eggertsson and Paul Krugman}
\author{Lucas Avezum \and Harry Huizinga \and Louis Raes}

%\author{F.~Author\inst{1} \and S.~Another\inst{2}}
% - Give the names in the same order as the appear in the paper.
% - Use the \inst{?} command only if the authors have different
%   affiliation.

%\institute[Universities of Somewhere and Elsewhere] % (optional, but mostly needed)
%{
 % \inst{1}%
  %Department of Computer Science\\
  %University of Somewhere
  %\and
  %\inst{2}%
  %Department of Theoretical Philosophy\\
  %University of Elsewhere}
% - Use the \inst command only if there are several affiliations.
% - Keep it simple, no one is interested in your street address.

\date{02/05/2018}
% - Either use conference name or its abbreviation.
% - Not really informative to the audience, more for people (including
%   yourself) who are reading the slides online

%\subject{Theoretical Computer Science}
% This is only inserted into the PDF information catalog. Can be left
% out. 

% If you have a file called "university-logo-filename.xxx", where xxx
% is a graphic format that can be processed by latex or pdflatex,
% resp., then you can add a logo as follows:

% \pgfdeclareimage[height=0.5cm]{university-logo}{university-logo-filename}
% \logo{\pgfuseimage{university-logo}}

% Delete this, if you do not want the table of contents to pop up at
% the beginning of each subsection:
%\AtBeginSubsection[]
%{
%  \begin{frame}<beamer>{Outline}
%    \tableofcontents[currentsection,currentsubsection]
%  \end{frame}
%}

% Let's get started
\begin{document}
	\begin{frame}
	\titlepage
\end{frame}

\begin{frame}{Introduction}{Research Question}

Does bank regulation affect non-financial firms' capital structure?\\

\vspace{\baselineskip}

We study the effect of:
\vspace{\baselineskip}
\begin{itemize}
	\item Restrictions on banking activities,
	\item Restrictions on financial conglomerates,
\item	Capital requirement stringency and
	\item Official supervisory power,
		
\end{itemize}
\vspace{\baselineskip}
The \textbf{cost of debt} is the transmission channel to firms' financial leverage. 
\end{frame}


\begin{frame}{Introduction}{Relevance}

\begin{itemize}
	\item Bank regulation and bank performance: \cite{barth2013}, \cite{anginer2014does},\cite{caprio2014macro}, \cite{demirguc2013bank}.
	\vspace{\baselineskip}
	\item Bank regulation and firm performance: \cite{amore2013credit}, \cite{jimenez2017macroprudential}.
	\vspace{\baselineskip}
	\item Determinants of firms' capital structure:
	\cite{leary2009bank}, \cite{rajan1995we}, \cite{oztekin2015capital}.   

\end{itemize}


\end{frame}

\begin{frame}{Introduction}{Method}

Theory
\begin{itemize}
	\item Model that relates bank regulation to firms' capital structure.	
\end{itemize}
\vspace{\baselineskip}
Data
\begin{itemize}
	\item Worldwide survey on bank regulation.
	\item Worldwide information on firms' balance sheet and ownership.
	\item We study around 400 thousand firms belonging to 60 thousand multinational groups hosted in 61 countries from 2005 to 2011.	
\end{itemize}
	\vspace{\baselineskip}
	Strategy
\begin{itemize}
	\item We compare firms within multinational corporations hosted in several countries
	and consequently exposed to bank regulation at different levels. 
	
	
\end{itemize}




 
\end{frame}

\begin{frame}{Introduction}{Results}

 Net effects of bank regulation on firms' capital structure depends on the tax rate: \\
\begin{itemize}
	
	\item Direct effect: in the absence of taxes, higher cost of debt decreases the firm's value.
	\item Indirect effect: gains or losses from the interest tax shield mitigate the direct effect.
\end{itemize}
\vspace{\baselineskip}
We find that the direct effect of:
\begin{itemize}
	\item Restrictions on banking activities and on financial conglomerates to be positive and,
	\item	Tighter capital requirement and greater official supervisory power to be negative.
	
\end{itemize}

\end{frame}

\begin{frame}{Introduction}{Outline}
  \tableofcontents
\end{frame}
\section{Model}
\begin{frame}<presentation:0>{Model}{Balance sheets and financial leverage}
Based on \cite*{huizinga2008capital}
\begin{itemize}
	\item  Balance sheet:
\end{itemize}
\begin{equation}
\begin{aligned}
A_i=I_i+L_i, \quad i=1,...,n-1.
\end{aligned}
\label{eq:sub balance sheet}
\end{equation}
\begin{equation}
\begin{aligned}
A_p+\sum_{i=1}^{n-1}I_i=E_p+L_p. 
\end{aligned}
\label{eq:parent balance sheet}
\end{equation}
\begin{itemize}
	\item  Financial leverage:
\end{itemize}
\begin{equation*}
\begin{aligned}
\lambda_i=L_i/A_i, \quad i=1,...,n-1.
\end{aligned}
\label{eq:sub leverage}
\end{equation*}
\begin{equation}
\begin{aligned}
\lambda_m=\frac{\sum_{i=1}^{n}L_i}{\sum_{i=1}^{n}A_i}=\sum_{i=1}^{n}\lambda_i\rho_i, 
\end{aligned}
\label{eq:total leverage}
\end{equation} 
\end{frame}

\begin{frame}<presentation:0>{Model}{Costs associated with leverage}
\begin{itemize}
	\item  Expected cost of bankruptcy \citep{luciano2014guarantees}:
\end{itemize}
\begin{equation}
\begin{aligned}
C_m=\frac{\gamma}{2}\lambda_m^2\bigg(\sum_{i=1}^{n}A_i\bigg).
\end{aligned}
\label{eq:cost bankruptcy}
\end{equation}
\begin{itemize}
\item  Costs/benefits associated to incentives to local managers:
\end{itemize}
\begin{equation}
\begin{aligned}
C_i=\frac{\mu}{2}(\lambda_i-\lambda^*)^2A_i-\frac{\mu}{2}(\lambda^*)^2A_i, \quad i=1,...,n.
\end{aligned}
\label{eq:agency cost}
\end{equation}
\begin{itemize}
	\item  Cost of debt:
\end{itemize}
\begin{equation}
\begin{aligned}
r_i=\delta_i(1+\phi'\Pi_i)
\end{aligned}
\label{eq:cost of debt}
\end{equation}

\end{frame}

\begin{frame}{Model}{Multinational's value}

\begin{itemize}
	\item Leveraged firm's value:
\end{itemize}
\begin{equation}
\begin{aligned}
V_i^L=V_i^U+\tau_{i}L_i-\phi'\Pi_iL_i+\tau_{i}\phi'\Pi_iL_i-C_i,  \quad i=1,...,n.
\end{aligned}
\label{eq:v_l_2}
\end{equation}	
\begin{itemize}
	\item Multinational's value:
\end{itemize}
\begin{equation}
\begin{aligned}
V_m^L=V_m^U+\sum_{i=1}^{n}\tau_iL_i-\phi'\sum_{i=1}^{n}\Pi_iL_i+\phi'\sum_{i=1}^{n}\tau_i\Pi_i L_i-C_m-\sum_{i=1}^{n}C_i
\end{aligned}
\label{eq:v_l}
\end{equation}
\end{frame}


\begin{frame}{Model}{Optimal leverage}
\begin{itemize}
	\item  Optimal leverage:
\end{itemize}
\begin{equation}
\begin{aligned}
\lambda_i = &\theta_0\lambda^*+\theta_1\tau_i-\theta_2\Pi_i+\theta_2\Pi_i\tau_{i}\\
&+\theta_3\sum_{j=1}^{n}(\tau_i-\tau_j)\rho_j-\theta_4\sum_{j=1}^{n}(\Pi_i-\Pi_j)\rho_j\\
&+\theta_4\sum_{j=1}^{n}(\Pi_i\tau_i-\Pi_j\tau_j)\rho_j, \quad i=1,...,n
\end{aligned}
\label{eq:optimal leverage in theory}
\end{equation}
\begin{equation*}
\begin{aligned}
&\theta_0=\frac{\mu}{(\mu+\gamma)}, \ \theta_1=\frac{1}{(\mu+\gamma)}, \
\theta_2=\frac{1}{(\mu+\gamma)}\phi', \\
&\theta_3=\frac{\gamma}{\mu(\mu+\gamma)}, \
\theta_4=\frac{\gamma}{\mu(\mu+\gamma)}\phi'.
\end{aligned}
\end{equation*}
\end{frame}



\section{Empirical Strategy}
\begin{frame}{Empirical Strategy}
\begin{itemize}
	\item  Optimal leverage:
\end{itemize}
\begin{equation}
\begin{aligned}
\lambda_i = &\theta_0\lambda^*+(\theta_1+\theta_3)\tau_i-(\theta_2+\theta_4)\Pi_i+(\theta_2+\theta_4)\Pi_i\tau_{i}\\
&-\theta_3\sum_{j=1}^{n}\tau_j\rho_j+\theta_4\sum_{j=1}^{n}\Pi_j\rho_j-\theta_4\sum_{j=1}^{n}\Pi_j\tau_j\rho_j, \quad i=1,...,n
\end{aligned}
\label{eq:optimal leverage in theory 2}
\end{equation}

\end{frame}

\section{Empirical Strategy}
\begin{frame}{Empirical Strategy}
Result lead to the following regression equation:
\begin{equation}
\begin{aligned}
\lambda_{imcst}=&\beta_0\tau_{ct}+\beta_1\Pi_{ct}+\beta_2\tau_{ct}\Pi_{ct}+\alpha_{mt}+\alpha_{c}+\alpha_{s}+\beta_3X_{imcst}+\varepsilon_{imcst}
\label{eq:optimal leverage empirically 1}
\end{aligned}
\end{equation}
$\beta_1$ and $\beta_2$ capture the direct and indirect total effect on multinationals, respectively. 
\end{frame}

\section{Data}

\begin{frame}{Data}{World Bank Regulation
	and Supervision Survey \citep{barth2013bank}}

	\begin{itemize}
	\item Hypothesis: restrictions on banking activities increase non-financial firms' leverage ($\beta_1>0$).
	
\vspace{\baselineskip}

	\item Hypothesis: restrictions on financial conglomerates increase non-financial firms' leverage ($\beta_1>0$)
	
\vspace{\baselineskip}

	\item Hypothesis: Tightening capital requirements decrease non-financial firms' leverage ($\beta_1<0$)
	

\vspace{\baselineskip}

	\item Hypothesis: greater official supervisory power decreases non-financial firms' leverage ($\beta_1<0$)
	
\end{itemize}

\end{frame}

\begin{frame}{Data}{Firm-level}
\begin{itemize}
	\item We use the Orbis database compiled by the Bureau Van Dijk.
	
	
\end{itemize}
\begin{center}
	\begin{tiny}	
		\begin{longtable}{lcccccc}\\
			\label{reg:benchmark}\\
		\primitiveinput{../../../output/tables/summary/summary.tex}
			\hline 			
	\end{longtable}	
	\end{tiny}
\end{center}
\end{frame}

\section{Results}

\begin{frame}{Results}

{\fontsize{6}{7}\selectfont		
\begin{longtable}{lcccccc}\\
	\label{reg:benchmark}\\
	\multicolumn{7}{c}{Table \ref{reg:benchmark} - Regression results, bank regulation and firms' capital structure}\\
	\multicolumn{7}{c}{(\textit{Dependent variable}: financial leverage)}
	\\ \hline \hline \addlinespace
	Model & (1) & (2) & (3) & (4) & (5) & (6) \\  \endfirsthead
	\multicolumn{7}{c}{Table \ref{reg:benchmark} - Regression results, bank regulation and firms' capital structure }\\
	\multicolumn{7}{c}{(\textit{Dependent variable}: financial leverage)\textit{(Continued)}}
	\\ \hline \hline \addlinespace Model & (1) & (2) & (3) & (4) & (5) & (6) \\ \hline \\ \endhead
	\hline
	\multicolumn{7}{r}{{\textit{(Continued)}}}\\ \endfoot 
	\endlastfoot
	\primitiveinput{../../../output/tables/regressions/benchmark_table.tex}
	\hline 			
\end{longtable}		
}

\end{frame}

\begin{frame}{Results}{Total effects}

{\fontsize{6}{7}\selectfont 		
	\begin{longtable}{lcc}\\
		\label{reg:effect}\\
		\multicolumn{3}{c}{Table \ref{reg:effect} - Total effects}\\
		\multicolumn{3}{c}{(Considering the average level of tax rate in the sample (47\%))}
		\\ \hline \hline \addlinespace
		Effect & One standard deviation increase & From least to most stringent  \\  \endfirsthead
		\multicolumn{3}{c}{Table \ref{reg:effect} - Multinational as representative sample}\\
		\multicolumn{3}{c}{(\textit{Dependent variable}: financial leverage)\textit{(Continued)}}
		\\ \hline \hline \addlinespace Model & (1)\\ \hline \\ \endhead
		\hline
		\multicolumn{3}{r}{{\textit{(Continued)}}}\\ \endfoot 	
		\endlastfoot
		\primitiveinput{../../../output/tables/regressions/effect_table.tex}
		\hline 			
	\end{longtable}		
}


\end{frame}


\begin{frame}{Results}{}

{\fontsize{5}{6}\selectfont 		
	\begin{longtable}{lcc}\\
		\label{reg:sub}\\
		\multicolumn{3}{c}{Table \ref{reg:sub} - Different effects from restrictions}\\
		\multicolumn{3}{c}{(\textit{Dependent variable}: financial leverage)}
		\\ \hline \hline \addlinespace
		 
		Model & (1) & (2)\\	
		\endfirsthead
		\multicolumn{3}{c}{Table \ref{reg:sub} - Different effects from restrictions}\\
		\multicolumn{3}{c}{(\textit{Dependent variable}: financial leverage)\textit{(Continued)}}
		\\ \hline \hline \addlinespace Model & (1) & (2)  \\ \hline \\ \endhead
		\hline
		\multicolumn{3}{r}{{\textit{(Continued)}}}\\ \endfoot  	
		\endlastfoot
		\primitiveinput{../../../output/tables/regressions/sub_indexes_table.tex}
		\hline 			
	\end{longtable}		
}


\end{frame}

\begin{frame}{Results}{Full sample of firms}

{\fontsize{4}{5}\selectfont 		
		\begin{longtable}{lccccc}\\
		\label{reg:rep}\\
		\multicolumn{6}{c}{Table \ref{reg:rep} - Full sample of firms}\\
		\multicolumn{6}{c}{(\textit{Dependent variable}: financial leverage)}
		\\ \hline \hline \addlinespace
		Model & (1) & (2) & (3) & (4) & (5) \\  \endfirsthead
		\multicolumn{6}{c}{Table \ref{reg:rep} - Multinational as representative sample}\\
		\multicolumn{6}{c}{(\textit{Dependent variable}: financial leverage)\textit{(Continued)}}
		\\ \hline \hline \addlinespace Model & (1) & (2) & (3) & (4) & (5) \\ \hline \\ \endhead
		\hline
		\multicolumn{6}{r}{{\textit{(Continued)}}}\\ \endfoot 	
		\endlastfoot
		\primitiveinput{../../../output/tables/regressions/representative_table.tex}
		\hline 			
	\end{longtable}		
}


\end{frame}


\begin{frame}{Conclusion}

Net effects of bank regulation on firms' capital structure depends on the tax rate: \\
\begin{itemize}
	
	\item Direct effect: in the absence of taxes, higher cost of debt decreases the firm's value.
	\item Indirect effect: gains or losses from the interest tax shield mitigate the direct effect.
\end{itemize}
\vspace{\baselineskip}
We find that the direct effect of:
\begin{itemize}
	\item Restrictions on banking activities and on financial conglomerates to be positive and,
	\item	Tighter capital requirement and greater official supervisory power to be negative.
	
\end{itemize}

\end{frame}



\begin{frame}{Bibliography}{}
\begin{tiny}
	\bibliography{../bibliography/references}
	\bibliographystyle{../bibliography/te}
\end{tiny}
\end{frame}
\end{document}


