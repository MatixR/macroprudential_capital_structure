% Copyright 2004 by Till Tantau <tantau@users.sourceforge.net>.
%
% In principle, this file can be redistributed and/or modified under
% the terms of the GNU Public License, version 2.
%
% However, this file is supposed to be a template to be modified
% for your own needs. For this reason, if you use this file as a
% template and not specifically distribute it as part of a another
% package/program, I grant the extra permission to freely copy and
% modify this file as you see fit and even to delete this copyright
% notice. 

\documentclass{beamer}
% Replace the \documentclass declaration above
% with the following two lines to typeset your 
% lecture notes as a handout:
%\documentclass{article}
%\usepackage{beamerarticle}
\usepackage{graphicx}
\usepackage{natbib}
\usepackage{sansmathaccent,longtable,booktabs,threeparttable,afterpage,mathtools,anyfontsize}
\pdfmapfile{+sansmathaccent.map}
% There are many different themes available for Beamer. A comprehensive
% list with examples is given here:
% http://deic.uab.es/~iblanes/beamer_gallery/index_by_theme.html
% You can uncomment the themes below if you would like to use a different
% one:
%\usetheme{AnnArbor}
%\usetheme{Antibes}
%\usetheme{Bergen}
%\usetheme{Berkeley}
%\usetheme{Berlin}
\usetheme{Boadilla}
%\usetheme{boxes}
%\usetheme{CambridgeUS}
%\usetheme{Copenhagen}
%\usetheme{Darmstadt}
%\usetheme{default}
%\usetheme{Frankfurt}
%\usetheme{Goettingen}
%\usetheme{Hannover}
%\usetheme{Ilmenau}
%\usetheme{JuanLesPins}
%\usetheme{Luebeck}
%\usetheme{Madrid}
%\usetheme{Malmoe}
%\usetheme{Marburg}
%\usetheme{Montpellier}
%\usetheme{PaloAlto}
%\usetheme{Pittsburgh}
%\usetheme{Rochester}
%\usetheme{Singapore}
%\usetheme{Szeged}
%\usetheme{Warsaw}

%***************************************************
% Fix input with \midrule problem
%***************************************************

\newcommand{\sym}[1]{\rlap{#1}}% Thanks to David Carlisle

\newcommand*{\captionsource}[2]{%
	\caption[{#1}]{%
		#1%
		\\\hspace{\linewidth}%
		\textbf{Source:} #2%
	}%
}
\newcommand{\horrule}[1]{\rule{\linewidth}{#1}} % Create horizontal rule command with 1 argument of height

%***************************************************
% Fix input with \midrule problem
%***************************************************

\makeatletter
\newcommand\primitiveinput[1]
{\@@input #1 }
\makeatother

\title{The Impact of Bank Regulation on Firms' Capital Structure: Evidence from Multinationals}

% A subtitle is optional and this may be deleted
%\subtitle{Gauti B. Eggertsson and Paul Krugman}
\author{Lucas Avezum \and Harry Huizinga \and Louis Raes}

%\author{F.~Author\inst{1} \and S.~Another\inst{2}}
% - Give the names in the same order as the appear in the paper.
% - Use the \inst{?} command only if the authors have different
%   affiliation.

%\institute[Universities of Somewhere and Elsewhere] % (optional, but mostly needed)
%{
 % \inst{1}%
  %Department of Computer Science\\
  %University of Somewhere
  %\and
  %\inst{2}%
  %Department of Theoretical Philosophy\\
  %University of Elsewhere}
% - Use the \inst command only if there are several affiliations.
% - Keep it simple, no one is interested in your street address.

\date{02/05/2018}
% - Either use conference name or its abbreviation.
% - Not really informative to the audience, more for people (including
%   yourself) who are reading the slides online

%\subject{Theoretical Computer Science}
% This is only inserted into the PDF information catalog. Can be left
% out. 

% If you have a file called "university-logo-filename.xxx", where xxx
% is a graphic format that can be processed by latex or pdflatex,
% resp., then you can add a logo as follows:

% \pgfdeclareimage[height=0.5cm]{university-logo}{university-logo-filename}
% \logo{\pgfuseimage{university-logo}}

% Delete this, if you do not want the table of contents to pop up at
% the beginning of each subsection:
%\AtBeginSubsection[]
%{
%  \begin{frame}<beamer>{Outline}
%    \tableofcontents[currentsection,currentsubsection]
%  \end{frame}
%}

% Let's get started
\begin{document}
	\begin{frame}
	\titlepage
\end{frame}

\begin{frame}{Introduction}{Research Question}

Does bank regulation affect non-financial firms' capital structure?\\

\vspace{\baselineskip}

We study the effect of:
\vspace{\baselineskip}
\begin{itemize}
	\item Restrictions on banking activities,
	\item Restrictions on financial conglomerates,
\item	Capital requirement stringency and
	\item Official supervisory power,
		
\end{itemize}
\vspace{\baselineskip}
The \textbf{cost of debt} is the transmission channel to firms' financial leverage. 
\end{frame}


\begin{frame}{Introduction}{Relevance}

\begin{itemize}
	\item Bank regulation and bank performance: \cite{barth2013}, \cite{anginer2014does},\cite{caprio2014macro}, \cite{demirguc2013bank}.
	\vspace{\baselineskip}
	\item Bank regulation and firm performance: \cite{amore2013credit}, \cite{jimenez2017macroprudential}.
	\vspace{\baselineskip}
	\item Determinants of firms' capital structure:
	\cite{leary2009bank}, \cite{rajan1995we}, \cite{oztekin2015capital}.   

\end{itemize}


\end{frame}

\begin{frame}{Introduction}{Method}

Theory
\begin{itemize}
	\item Model that relates bank regulation to firms' capital structure.	
\end{itemize}
\vspace{\baselineskip}
Data
\begin{itemize}
	\item Worldwide survey on bank regulation.
	\item Worldwide information on firms' balance sheet and ownership.
	\item We study around 400 thousand firms belonging to 60 thousand multinational groups hosted in 61 countries from 2005 to 2011.	
\end{itemize}
	\vspace{\baselineskip}
	Strategy
\begin{itemize}
	\item We compare firms within multinational corporations hosted in several countries
	and consequently exposed to bank regulation at different levels. 
	
	
\end{itemize}




 
\end{frame}

\begin{frame}{Introduction}{Results}

 Effects of bank regulation on firms' capital structure changed since the crisis: \\
\begin{itemize}
	
	\item Before: restrictions on financial conglomerates and insurance activities were important.
	\item After: bank capital stringency and restriction on activities related to investment banking became central.
\end{itemize}
\vspace{\baselineskip}
Effects are not exclusive to multinational groups.\\
\vspace{\baselineskip}
Together, the effects explain 4.1\% of sample variation in leverage.  \\
\vspace{\baselineskip}
Tax shield mitigate the direct effect.

\end{frame}

\begin{frame}{Introduction}{Outline}
  \tableofcontents
\end{frame}
\section{Model}
\begin{frame}{Model}{Balance sheets and financial leverage}
Based on \cite*{huizinga2008capital}
\begin{itemize}
	\item  Balance sheet:
\end{itemize}
\begin{equation}
\begin{aligned}
A_i=I_i+L_i, \quad i=1,...,n-1.
\end{aligned}
\label{eq:sub balance sheet}
\end{equation}
\begin{equation}
\begin{aligned}
A_p+\sum_{i=1}^{n-1}I_i=E_p+L_p. 
\end{aligned}
\label{eq:parent balance sheet}
\end{equation}
\begin{itemize}
	\item  Financial leverage:
\end{itemize}
\begin{equation*}
\begin{aligned}
\lambda_i=L_i/A_i, \quad i=1,...,n-1.
\end{aligned}
\label{eq:sub leverage}
\end{equation*}
\begin{equation}
\begin{aligned}
\lambda_m=\frac{\sum_{i=1}^{n}L_i}{\sum_{i=1}^{n}A_i}=\sum_{i=1}^{n}\lambda_i\rho_i, 
\end{aligned}
\label{eq:total leverage}
\end{equation} 
\end{frame}

\begin{frame}{Model}{Costs associated with leverage}
\begin{itemize}
	\item  Expected cost of bankruptcy \citep{luciano2014guarantees}:
\end{itemize}
\begin{equation}
\begin{aligned}
C_m=\frac{\gamma}{2}\lambda_m^2\bigg(\sum_{i=1}^{n}A_i\bigg).
\end{aligned}
\label{eq:cost bankruptcy}
\end{equation}
\begin{itemize}
\item  Costs/benefits associated to incentives to local managers:
\end{itemize}
\begin{equation}
\begin{aligned}
C_i=\frac{\mu}{2}(\lambda_i-\lambda^*)^2A_i-\frac{\mu}{2}(\lambda^*)^2A_i, \quad i=1,...,n.
\end{aligned}
\label{eq:agency cost}
\end{equation}
\begin{itemize}
	\item  Cost of debt:
\end{itemize}
\begin{equation}
\begin{aligned}
r_i=\delta_i(1+\phi'\Pi_i)
\end{aligned}
\label{eq:cost of debt}
\end{equation}

\end{frame}

\begin{frame}{Model}{Multinational's value}
\begin{itemize}
	\item  Unleveraged firm's value:
\end{itemize}
\begin{equation}
\begin{aligned}
V_i^U=\frac{R_i}{\delta_i}(1-\tau_{i}),  \quad i=1,...,n.
\end{aligned}
\label{eq:v_u}
\end{equation}
\begin{itemize}
	\item Leveraged firm's value:
\end{itemize}
\begin{equation}
\begin{aligned}
V_i^L=L_i+\frac{(R_i-r_iL_i)}{\delta_i}(1-\tau_{i})-C_i,  \quad i=1,...,n.
\end{aligned}
\label{eq:v_l_1}
\end{equation}
\begin{equation}
\begin{aligned}
V_i^L=V_i^U+\tau_{i}L_i-\phi'\Pi_iL_i+\tau_{i}\phi'\Pi_iL_i-C_i,  \quad i=1,...,n.
\end{aligned}
\label{eq:v_l_2}
\end{equation}	
\begin{itemize}
	\item Multinational's value:
\end{itemize}
\begin{equation}
\begin{aligned}
V_m^L=V_m^U+\sum_{i=1}^{n}\tau_iL_i-\phi'\sum_{i=1}^{n}\Pi_iL_i+\phi'\sum_{i=1}^{n}\tau_i\Pi_i L_i-C_m-\sum_{i=1}^{n}C_i
\end{aligned}
\label{eq:v_l}
\end{equation}
\end{frame}


\begin{frame}{Model}{Optimal leverage}
\begin{itemize}
	\item  Optimal leverage:
\end{itemize}
\begin{equation}
\begin{aligned}
\lambda_i = &\theta_0\lambda^*+\theta_1\tau_i-\theta_2\Pi_i+\theta_2\Pi_i\tau_{i}\\
&+\theta_3\sum_{j=1}^{n}(\tau_i-\tau_j)\rho_j-\theta_4\sum_{j=1}^{n}(\Pi_i-\Pi_j)\rho_j\\
&+\theta_4\sum_{j=1}^{n}(\Pi_i\tau_i-\Pi_j\tau_j)\rho_j, \quad i=1,...,n
\end{aligned}
\label{eq:optimal leverage in theory}
\end{equation}
\begin{equation*}
\begin{aligned}
&\theta_0=\frac{\mu}{(\mu+\gamma)}, \ \theta_1=\frac{1}{(\mu+\gamma)}, \
\theta_2=\frac{1}{(\mu+\gamma)}\phi', \\
&\theta_3=\frac{\gamma}{\mu(\mu+\gamma)}, \
\theta_4=\frac{\gamma}{\mu(\mu+\gamma)}\phi'.
\end{aligned}
\end{equation*}
\end{frame}




\section{Empirical Strategy}
\begin{frame}{Empirical Strategy}
Result lead to the following regression equation:
\begin{equation}
\begin{aligned}
\lambda_{imcst}=&\beta_0\tau_{ct}+\beta_1\Pi_{ct}+\beta_2\tau_{ct}\Pi_{ct}+\alpha_{mt}+\alpha_{c}+\alpha_{s}+\beta_3X_{imcst}+\varepsilon_{imcst}
\label{eq:optimal leverage empirically 1}
\end{aligned}
\end{equation}

\end{frame}

\section{Data}

\begin{frame}{Data}{World Bank Regulation
	and Supervision Survey \citep{barth2013bank}}

\begin{itemize}
	\item Addressed to the head of banking supervision.
	\vspace{\baselineskip}
	\item Comparable across countries and time.

	
\end{itemize}
	

\end{frame}

\begin{frame}{Data}{World Bank Regulation
	and Supervision Survey \citep{barth2013bank}}

	\textit{4.1 What are the conditions under which banks can engage in securities activities?\\
	a. A full range of these activities can be conducted directly in banks,\\
	b. A full range of these activities are offered but all or some of these activities must be conducted in subsidiaries, or in another part of a common holding company or parent,\\
	c. Less than the full range of activities can be conducted in banks, or subsidiaries, or in another part of a common holding company or parent,\\
	d. None of these activities can be done in either banks or subsidiaries, or in another part of a common holding company or parent.}
	\begin{itemize}
	\item Hypothesis: restrictions on banking activities increase non-financial firms' leverage ($\beta_1>0$).
	
\end{itemize}

\end{frame}

\begin{frame}{Data}{World Bank Regulation
	and Supervision Survey \citep{barth2013bank}}

\textit{4.4 What are the conditions under which banks can engage in nonfinancial businesses except those businesses that are auxiliary to banking business (e.g. IT company, debt collection company etc.) ?\\
a. Nonfinancial activities can be conducted directly in banks,\\
b. Nonfinancial activities must be conducted in subsidiaries, or in another part of a common holding company or parent\\
c. Nonfinancial activities may be conducted in subsidiaries, or in another part of a common holding company or parent, but subject to regulatory limit or approval,\\
d. None of these activities can be done in either banks or subsidiaries, or in another part of a common holding company or parent.
}
\begin{itemize}
	\item Hypothesis: restrictions on financial conglomerates increase non-financial firms' leverage ($\beta_1>0$)
	
\end{itemize}

\end{frame}

\begin{frame}{Data}{World Bank Regulation
	and Supervision Survey \citep{barth2013bank}}

\textit{3.1 Which regulatory capital adequacy regimes did you use as of end of 2010 and for which banks does each regime apply to (if using more than one regime)?\\
	Mark the appropriate response below and specify for which types of banks each regime applies\\
	a. Basel I\\
	b. Basel II\\
	c. Leverage ratio\\
	d. Other (please explain)
}
\begin{itemize}
	\item Hypothesis: Tightening capital requirements decrease non-financial firms' leverage ($\beta_1<0$)
	
\end{itemize}

\end{frame}

\begin{frame}{Data}{World Bank Regulation
	and Supervision Survey \citep{barth2013bank}}

\textit{12.3.2 Can the supervisory authority force a bank to change its internal organizational structure?}	\\  
\vspace{\baselineskip}
	\textit{11.1 Please indicate whether the following enforcement powers are available to the supervisory agency\\
f. Require banks to constitute provisions to cover actual or potential losses \\
j. Require banks to reduce or suspend dividends to shareholders \\
k. Require banks to reduce or suspend bonuses and other remuneration to bank directors and managers}
\vspace{\baselineskip}
\begin{itemize}
	\item Hypothesis: greater official supervisory power decreases non-financial firms' leverage ($\beta_1<0$)
	
\end{itemize}

\end{frame}

\begin{frame}{Data}{Firm-level}
\begin{itemize}
	\item We use the Orbis database compiled by the Bureau Van Dijk.
	\vspace{\baselineskip}
	\item Worldwide accounting and ownership information.
	\vspace{\baselineskip}
	\item Subsidiaries: a firm that has more than 50\% of its share owned by another firm.
	\vspace{\baselineskip}
	\item Parent firm: a firm that owns one or more companies but none of its shareholders have more than 50\% of its shares.
	\vspace{\baselineskip}
	\item Multinational corporation: group of firms with common parent firm.
	
\end{itemize}
\begin{center}
	\begin{tiny}	
	\end{tiny}
\end{center}
\end{frame}

\section{Results}

\begin{frame}{Results}

{\fontsize{6}{7}\selectfont		
\begin{longtable}{lcccccc}\\
	\label{reg:benchmark}\\
	\multicolumn{7}{c}{Table \ref{reg:benchmark} - Regression results, bank regulation and firms' capital structure}\\
	\multicolumn{7}{c}{(\textit{Dependent variable}: financial leverage)}
	\\ \hline \hline \addlinespace
	Model & (1) & (2) & (3) & (4) & (5) & (6) \\  \endfirsthead
	\multicolumn{7}{c}{Table \ref{reg:benchmark} - Regression results, bank regulation and firms' capital structure }\\
	\multicolumn{7}{c}{(\textit{Dependent variable}: financial leverage)\textit{(Continued)}}
	\\ \hline \hline \addlinespace Model & (1) & (2) & (3) & (4) & (5) & (6) \\ \hline \\ \endhead
	\hline
	\multicolumn{7}{r}{{\textit{(Continued)}}}\\ \endfoot 
	\endlastfoot
	\primitiveinput{../../../output/tables/regressions/benchmark_table.tex}
	\hline 			
\end{longtable}		
}

\end{frame}

\begin{frame}{Results}{Before 2008}

{\fontsize{6}{7}\selectfont		
	\begin{longtable}{lcccccc}\\
		\label{reg:prior}\\
		\multicolumn{7}{c}{Table \ref{reg:prior} - Regression results, bank regulation and firms' capital structure}\\
		\multicolumn{7}{c}{(\textit{Dependent variable}: financial leverage)}
		\\ \hline \hline \addlinespace
		Model & (1) & (2) & (3) & (4) & (5) & (6) \\  \endfirsthead
		\multicolumn{7}{c}{Table \ref{reg:prior} - Regression results, bank regulation and firms' capital structure }\\
		\multicolumn{7}{c}{(\textit{Dependent variable}: financial leverage)\textit{(Continued)}}
		\\ \hline \hline \addlinespace Model & (1) & (2) & (3) & (4) & (5) & (6) \\ \hline \\ \endhead
		\hline
		\multicolumn{7}{r}{{\textit{(Continued)}}}\\ \endfoot 
		\endlastfoot
		\primitiveinput{../../../output/tables/regressions/benchmark_table_prior.tex}
		\hline 			
	\end{longtable}		
}

\end{frame}

\begin{frame}{Results}{After 2008}

{\fontsize{6}{7}\selectfont		
	\begin{longtable}{lcccccc}\\
		\label{reg:post}\\
		\multicolumn{7}{c}{Table \ref{reg:post} - Regression results, bank regulation and firms' capital structure}\\
		\multicolumn{7}{c}{(\textit{Dependent variable}: financial leverage)}
		\\ \hline \hline \addlinespace
		Model & (1) & (2) & (3) & (4) & (5) & (6) \\  \endfirsthead
		\multicolumn{7}{c}{Table \ref{reg:post} - Regression results, bank regulation and firms' capital structure }\\
		\multicolumn{7}{c}{(\textit{Dependent variable}: financial leverage)\textit{(Continued)}}
		\\ \hline \hline \addlinespace Model & (1) & (2) & (3) & (4) & (5) & (6) \\ \hline \\ \endhead
		\hline
		\multicolumn{7}{r}{{\textit{(Continued)}}}\\ \endfoot 
		\endlastfoot
		\primitiveinput{../../../output/tables/regressions/benchmark_table_post.tex}
		\hline 			
	\end{longtable}		
}

\end{frame}


\begin{frame}{Results}{}

{\fontsize{5}{6}\selectfont 		
	\begin{longtable}{lcccccc}\\
		\label{reg:sub}\\
		\multicolumn{7}{c}{Table \ref{reg:sub} - Different effects from restrictions}\\
		\multicolumn{7}{c}{(\textit{Dependent variable}: financial leverage)}
		\\ \hline \hline \addlinespace
		& \multicolumn{2}{c}{Full sample} & \multicolumn{2}{c}{Before 2008} & \multicolumn{2}{c}{After 2008} \\  
		Model & (1) & (2) & (3) & (4)& (5)& (6) \\	
		\endfirsthead
		\multicolumn{7}{c}{Table \ref{reg:sub} - Different effects from restrictions}\\
		\multicolumn{7}{c}{(\textit{Dependent variable}: financial leverage)\textit{(Continued)}}
		\\ \hline \hline \addlinespace Model & (1) & (2) & (3) & (4)& (5)& (6)  \\ \hline \\ \endhead
		\hline
		\multicolumn{7}{r}{{\textit{(Continued)}}}\\ \endfoot  	
		\endlastfoot
		\primitiveinput{../../../output/tables/regressions/sub_indexes_table.tex}
		\hline 			
	\end{longtable}		
}


\end{frame}

\begin{frame}{Results}{Full sample of firms}

{\fontsize{6}{7}\selectfont 		
		\begin{longtable}{lccccc}\\
		\label{reg:rep}\\
		\multicolumn{6}{c}{Table \ref{reg:rep} - Full sample of firms}\\
		\multicolumn{6}{c}{(\textit{Dependent variable}: financial leverage)}
		\\ \hline \hline \addlinespace
		Model & (1) & (2) & (3) & (4) & (5) \\  \endfirsthead
		\multicolumn{6}{c}{Table \ref{reg:rep} - Multinational as representative sample}\\
		\multicolumn{6}{c}{(\textit{Dependent variable}: financial leverage)\textit{(Continued)}}
		\\ \hline \hline \addlinespace Model & (1) & (2) & (3) & (4) & (5) \\ \hline \\ \endhead
		\hline
		\multicolumn{6}{r}{{\textit{(Continued)}}}\\ \endfoot 	
		\endlastfoot
		\primitiveinput{../../../output/tables/regressions/representative_table.tex}
		\hline 			
	\end{longtable}		
}


\end{frame}


\begin{frame}{Results}{Total effects}

{\fontsize{6}{7}\selectfont 		
	\begin{longtable}{lcc}\\
		\label{reg:effect}\\
		\multicolumn{3}{c}{Table \ref{reg:effect} - Total effects}\\
		\multicolumn{3}{c}{(Considering the average level of tax rate in the sample (49\%))}
		\\ \hline \hline \addlinespace
		Effect & One standard deviation increase & From least to most stringent  \\  \endfirsthead
		\multicolumn{3}{c}{Table \ref{reg:effect} - Multinational as representative sample}\\
		\multicolumn{3}{c}{(\textit{Dependent variable}: financial leverage)\textit{(Continued)}}
		\\ \hline \hline \addlinespace Model & (1)\\ \hline \\ \endhead
		\hline
		\multicolumn{3}{r}{{\textit{(Continued)}}}\\ \endfoot 	
		\endlastfoot
		\primitiveinput{../../../output/tables/regressions/effect_table_full_sample.tex}
		\hline 			
	\end{longtable}		
}


\end{frame}


\begin{frame}{Conclusion}

 Effects of bank regulation on firms' capital structure changed since the crisis: \\
\begin{itemize}
	
	\item Before: restrictions on financial conglomerates and insurance activities were important.
	\item After: bank capital stringency and restriction on activities related to investment banking became central.
\end{itemize}
	    \vspace{\baselineskip}
   Effects are not exclusive to multinational groups.\\
       \vspace{\baselineskip}
   Together, the effects explain 4.1\% of sample variation in leverage.   
	

\end{frame}



\begin{frame}{Bibliography}{}
\begin{tiny}
\bibliography{C:/Users/u1273941/Research/Projects/macroprudential_capital_structure/analysis/code/text/bibliography/references}
\bibliographystyle{C:/Users/u1273941/Research/Projects/macroprudential_capital_structure/analysis/code/text/bibliography/te}
\end{tiny}
\end{frame}
\end{document}


