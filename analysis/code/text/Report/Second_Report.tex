\documentclass[12pt]{article}
\usepackage[utf8]{inputenc}
\usepackage{amssymb,amsmath,amsfonts,eurosym,geometry,ulem,caption,color,xcolor,multicol,setspace,sectsty,comment,footmisc,caption,natbib,pdflscape,subfigure,array,hyperref,verbatim,mathpazo,longtable,ntheorem,siunitx}
%\usepackage[normalem]{ulem}
\usepackage[pdftex]{graphicx}
\graphicspath{{Imagens/}}
\usepackage{fullpage}
\usepackage{indentfirst}
\usepackage{changepage}

%\setlength{\pdfpagewidth}{8.5in} \setlength{\pdfpageheight}{11in}
%\setlength{\textheight}{8.5in} \setlength{\topmargin}{0.0in}
%\setlength{\headheight}{0.0in} \setlength{\headsep}{0.0in}
%\setlength{\leftmargin}{0.5in}
%\setlength{\oddsidemargin}{0.0in}
%\setlength{\parindent}{2em}
%\setlength{\parskip}{\baselineskip}%
%\setlength{\textwidth}{6.5in}
%\linespread{1.6}
\newcommand*{\captionsource}[2]{%
  \caption[{#1}]{%
    #1%
    \\\hspace{\linewidth}%
    \textbf{Source:} #2%
  }%
}
\newcommand{\horrule}[1]{\rule{\linewidth}{#1}} % Create horizontal rule command with 1 argument of height

\onehalfspacing
\newtheorem{theorem}{Theorem}
\newtheorem{corollary}[theorem]{Corollary}
\newtheorem{proposition}{Proposition}
\newenvironment{proof}[1][Proof]{\noindent\textbf{#1.} }{\ \rule{0.5em}{0.5em}}

\newtheorem{hyp}{Hypothesis}
\newtheorem{subhyp}{Hypothesis}[hyp]
\renewcommand{\thesubhyp}{\thehyp\alph{subhyp}}

\newcommand{\red}[1]{{\color{red} #1}}
\newcommand{\blue}[1]{{\color{blue} #1}}

\newcolumntype{L}[1]{>{\raggedright\let\newline\\arraybackslash\hspace{0pt}}m{#1}}
\newcolumntype{C}[1]{>{\centering\let\newline\\arraybackslash\hspace{0pt}}m{#1}}
\newcolumntype{R}[1]{>{\raggedleft\let\newline\\arraybackslash\hspace{0pt}}m{#1}}

\geometry{left=1.2in,right=1.2in,top=1.2in,bottom=1.2in}
\begin{document}
	
%	\begin{titlepage}
		\title{Macroprudential Policies and Capital Structure: Evidence from Multinationals}
		\author{Lucas Avezum}
		%\author{Lucas Avezum\thanks{abc} \and \and Harry Huizinga\thanks{abc} \and Louis Raes\thanks{abc}}
		\date{\today}
		\maketitle
		%\begin{abstract}
		%	\noindent This paper studies how macroprudential policies, specifically reserve and capital requirements, relate to non-financial firms' capital structure. We first present a model where macroprudential instruments affect banks' costs which in turn are transmitted to firms' leverage through their cost of debt. Using a firm-level dataset of multinationals hosted in 37 countries we are able to identify domestic and international effects. The empirical results show that tighter capital requirements reduce firms' leverage while multinationals have an extra incentive to lower debt levels in subsidiaries hosted in countries with relative higher reserve requirements. Moreover, the domestic effect is found to be stronger for riskier firms.  \\
		%	\vspace{0in}\\
			%\noindent\textbf{Keywords:} key1, key2, key3\\
		%	\vspace{0in}\\
			%\noindent\textbf{JEL Codes:} key1, key2, key3\\
			
		%	\bigskip
	%	\end{abstract}
	%	\setcounter{page}{0}
	%	\thispagestyle{empty}
%	\end{titlepage}
%	\pagebreak \newpage
	
	
	
	
	\doublespacing
	


	 
	\section{Report on results} \label{sec:result}
	 This report presents some  of the results following improvement suggestions. In section \ref{sec:benchmark} the same model that was in the thesis is shown. The differences come from updating the dataset and adjusting some details (such as keeping the sample constant across regressions and computing tightening and loosening as 1 unit instead of 0.25). The results changed slightly the magnitude but not the direction.
	 
	 In section \ref{sec:base_year} I changed the year base to 2008 as a robustness test. There are some small changes in the results that I believe to be driven by change in sample, but the coefficients of interest remain statistically significant and with expected sign.
	 
	 Lastly, section \ref{sec:barth} shows the results when we use the capital stringency 2011 survey from \cite*{barth2013bank}. This survey refers to the years 2008, 2009 and 2010, which are the only years included in the regressions. Unfortunately, the results do not follow what we would expect, despite being statistically significant in some specifications. I also tried to merge \cite{barth2013bank} and \cite{cerutti2017changes} but the results do not change.   	      
 		
 		\section{Thesis benchmark} \label{sec:benchmark}
 		
 	\begin{small}
 	{\setstretch{1.0}
 		\begin{longtable}{lcccccc}\\
	\label{tab:number of firms}\\
	\multicolumn{7}{c}{Table \ref{tab:number of firms} - Summary statistics  (\textit{Base year}: 2007=1)}\\ \hline \hline\\
	
	Variable   &   Observations &    Mean &          SD&         Min&    Median &    Max\\
	
	\hline \endfirsthead
	
	\multicolumn{7}{c}{Table \ref{tab:number of firms} - Summary statistics (\textit{Base year}: 2007=1) (\textit{Continued})}\\ \hline \hline\\
	
	Variable   &   Observations &    Mean &          SD&         Min&     Median  &  Max\\
	
	\hline \endhead
	\hline
	\multicolumn{7}{r}{{\textit{(Continued)}}}\\ \endfoot
	
	\endlastfoot
	
	Panel A: main variables         &      &        &        &        &       \\
\quad Financial leverage  &      367.014&        1.09&        1.08&        0.00&        0.97&       12.80\\
\quad Reserve requirement &      367.014&       -0.41&        0.67&       -6.00&        0.00&        8.25\\
\quad Capital requirement &      367.014&        0.50&        0.69&        0.00&        0.00&        2.00\\
\quad Tax rate            &      367.003&        0.96&        0.08&        0.62&        0.99&        1.37\\
\quad Reserve requirement spillover&      367.014&       -0.12&        0.51&       -7.88&        0.00&        5.88\\
\quad Capital requirement spillover&      367.014&        0.01&        0.17&       -2.00&        0.00&        1.63\\
\quad Tax rate spillover  &      367.014&       -0.00&        0.05&       -0.46&        0.00&        0.52\\
         &      &        &        &        &       \\
Panel B: firm-level controls         &      &        &        &        &       \\
\quad Tangibility         &      367.014&        1.55&        2.66&        0.00&        1.00&       21.82\\
\quad Log of fixed assets &      367.014&        1.00&        0.08&        0.68&        1.00&        1.38\\
\quad Profitability       &      367.014&        0.72&        4.07&      -23.38&        0.74&       23.21\\
\quad Opportunity         &      367.014&       -0.02&        0.10&       -0.42&       -0.01&        0.34\\
\quad Risk                &      367.014&        0.09&        0.13&        0.00&        0.06&        1.96\\
         &      &        &        &        &       \\
Panel C: country-level controls         &      &        &        &        &       \\
\quad Inflation rate      &      367.014&        0.02&        0.02&       -0.01&        0.02&        0.22\\
\quad GDP growth rate     &      367.014&        0.00&        0.03&       -0.15&        0.00&        0.10\\
\quad Private credit to GDP&      367.014&        1.04&        0.13&        0.39&        1.07&        1.59\\
\quad Policy rate         &      337.340&        0.38&        0.32&        0.00&        0.31&        1.30\\
\quad Political Risk      &      367.014&        0.95&        0.05&        0.84&        0.97&        1.05\\
\quad Exchange rate risk  &      367.014&        0.99&        0.10&        0.10&        1.00&        1.11\\
\quad Law and order       &      367.014&        1.00&        0.01&        0.67&        1.00&        1.33\\
\hline
\end{longtable}

 	}
 \end{small}
 
 	\begin{small}
 	{\setstretch{1.0}
 		\begin{tabular}{lccc}
\multicolumn{4}{c}{Effect of macroprudential policies on firm's financial leverage} \\ \hline
 & (1) & (2) & (3) \\
VARIABLES &  &  &  \\ \hline
 &  &  &  \\
Reserve req. on local currency & -0.052* & -0.330* &  \\
 & (0.027) & (0.179) &  \\
LTV ratio limits & 0.019 & 0.068 &  \\
 & (0.078) & (0.299) &  \\
Capital requirement & -0.112*** & -0.544*** &  \\
 & (0.041) & (0.198) &  \\
Capital buffer - real estate & 0.119 & 0.938** &  \\
 & (0.076) & (0.413) &  \\
Reserve req. on local currency*tax & 0.021 & 0.375** &  \\
 & (0.029) & (0.183) &  \\
LTV ratio limits*tax & -0.048 & -0.057 &  \\
 & (0.081) & (0.313) &  \\
Capital requirement*tax & 0.109** & 0.596*** &  \\
 & (0.042) & (0.205) &  \\
Capital buffer - real estate*tax & -0.027 & -1.039** &  \\
 & (0.084) & (0.435) &  \\
Reserve req. on local currency spillover & -0.061*** &  & -0.054*** \\
 & (0.017) &  & (0.020) \\
LTV ratio limits spillover & 0.002 &  & 0.008 \\
 & (0.012) &  & (0.017) \\
Capital requirement spillover & 0.031** &  & 0.038* \\
 & (0.014) &  & (0.021) \\
Capital buffer - real estate spillover & -0.086*** &  & -0.069** \\
 & (0.021) &  & (0.027) \\
Tax rate & 0.007 & 0.197** &  \\
 & (0.018) & (0.081) &  \\
Tax rate spillover & -0.026 &  & -0.005 \\
 & (0.020) &  & (0.035) \\
Tangibility & -0.010*** & -0.016*** & -0.009*** \\
 & (0.001) & (0.002) & (0.001) \\
Log of fixed assets & 0.631*** & 1.113*** & 0.774*** \\
 & (0.017) & (0.053) & (0.023) \\
Profitability & -0.004*** & -0.004*** & -0.005*** \\
 & (0.000) & (0.001) & (0.000) \\
Opportunity & 0.012** & 0.075*** & 0.055*** \\
 & (0.005) & (0.024) & (0.008) \\
Volatility of profits &  & 0.118*** & 0.106*** \\
 &  & (0.018) & (0.016) \\
Inflation & 0.260*** & 0.255 &  \\
 & (0.049) & (0.290) &  \\
GDP growth rate & 0.060** & 0.288 &  \\
 & (0.028) & (0.177) &  \\
Private credit to GDP & 0.043*** & 0.030 &  \\
 & (0.010) & (0.041) &  \\
Policy rate & 0.007 & -0.088* &  \\
 & (0.009) & (0.050) &  \\
Political Risk & -0.036* & -0.022 &  \\
 & (0.021) & (0.113) &  \\
Exchange rate risk & -0.009 & -0.079*** &  \\
 & (0.007) & (0.030) &  \\
Law and order & 0.067 & 2.877 &  \\
 & (0.633) & (2.000) &  \\
 &  &  &  \\
Observations & 2,134,105 & 828,172 & 2,575,636 \\
R-squared & 0.75 & 0.32 & 0.55 \\
Number of firms & 391,131 & 173,322 & 489,708 \\
Firm fixed effects & Yes & No & No \\
Year fixed effects & Yes & No & No \\
Multinational*year fixed effects & No & Yes & No \\
Industry fixed effects & No & Yes & Yes \\
Multinational fixed effects & No & No & Yes \\
 Country*year fixed effects & No & No & Yes \\ \hline
\multicolumn{4}{c}{ Robust standard errors in parentheses} \\
\multicolumn{4}{c}{ *** p$<$0.01, ** p$<$0.05, * p$<$0.1} \\
\end{tabular}

 	}
 \end{small}
 
	\section{Different base date} \label{sec:base_year}
	
		\begin{small}
		{\setstretch{1.0}
			\begin{longtable}{lcccccc}\\
	\label{tab:number of firms 2008}\\
	\multicolumn{7}{c}{Table \ref{tab:number of firms 2008} - Summary statistics (\textit{Base year}: 2008=1)}\\ \hline \hline\\
	
	Variable   &   Observations &    Mean &          SD&         Min&    Median &    Max\\
	
	\hline \endfirsthead
	
	\multicolumn{7}{c}{Table \ref{tab:number of firms 2008} - Summary statistics (\textit{Base year}: 2008=1) (\textit{Continued})}\\ \hline \hline\\
	
	Variable   &   Observations &    Mean &          SD&         Min&     Median  &  Max\\
	
	\hline \endhead
	\hline
	\multicolumn{7}{r}{{\textit{(Continued)}}}\\ \endfoot
	
	\endlastfoot
	
	Panel A: main variables         &      &        &        &        &       \\
\quad Financial leverage  &      354.082&        1.08&        0.86&        0.00&        0.99&       10.08\\
\quad Reserve requirement &      354.082&       -0.38&        0.64&       -6.00&        0.00&        6.25\\
\quad Capital requirement &      354.082&        0.49&        0.68&        0.00&        0.00&        2.00\\
\quad Tax rate         &      354.059&        0.97&        0.08&        0.64&        0.99&        1.39\\
\quad Reserve requirement spillover&      354.082&       -0.05&        0.35&       -5.90&        0.00&        6.50\\
\quad Capital requirement spillover&      354.082&        0.00&        0.12&       -1.96&        0.00&        1.55\\
\quad  Tax rate spillover          &      354.082&       -0.00&        0.05&       -0.41&        0.00&        0.54\\
          &      &        &        &        &       \\
 Panel B: firm-level controls         &      &        &        &        &       \\
\quad Tangibility         &      354.082&        1.33&        1.84&        0.00&        0.99&       15.65\\
\quad Log of fixed assets &      354.082&        1.00&        0.07&        0.71&        1.00&        1.32\\
\quad Profitability       &      354.082&        0.80&        3.89&      -21.22&        0.80&       22.35\\
\quad Opportunity         &      354.082&       -0.01&        0.10&       -0.42&        0.00&        0.34\\
\quad Risk                &      354.082&        0.09&        0.13&        0.00&        0.06&        1.96\\
         &      &        &        &        &       \\
Panel C: country-level controls         &      &        &        &        &       \\
\quad Inflation rate      &      354.082&        0.02&        0.01&       -0.01&        0.02&        0.15\\
\quad GDP growth rate     &      354.082&        0.01&        0.03&       -0.15&        0.01&        0.10\\
\quad Private credit to GDP&      354.082&        1.00&        0.09&        0.50&        1.02&        1.99\\
\quad Interest rate    &      325.209&        0.37&        0.30&        0.00&        0.31&        1.23\\
\quad Political risk   &      354.082&        0.96&        0.05&        0.86&        0.98&        1.14\\
\quad Exchange rate risk&      354.082&        1.12&        0.33&        0.40&        1.05&       10.00\\
\quad Law order        &      354.082&        1.00&        0.01&        0.75&        1.00&        1.20\\
\hline
\end{longtable}

		}
	\end{small}

	 	\begin{small}
	 	{\setstretch{1.0}
	 		\begin{longtable}{lcccc}\\
	\label{reg:base_year}\\
	\multicolumn{5}{c}{Table \ref{reg:base_year} - Regression results, macroprudential policies and capital structure }\\
	\multicolumn{5}{c}{(\textit{Dependent variable}: financial leverage)(\textit{Base year}: 2008=1)}
	\\ \hline \hline
	Model & (1) & (2) & (3) & (4)  \\ \hline
	&  &  &  \\ \endfirsthead
	\multicolumn{5}{c}{Table \ref{reg:base_year} - Regression results, macroprudential policies and capital structure }\\
	\multicolumn{5}{c}{(\textit{Dependent variable}: financial leverage)(\textit{Base year}: 2008=1)\textit{(Continued)}}
	\\ \hline \hline
	Model & (1) & (2) & (3) & (4)  \\ \hline 
	&  &  &  & \\ \endhead
	\hline
	\multicolumn{5}{r}{{\textit{(Continued)}}}\\ \endfoot	
	\endlastfoot
	Effects of macroprudential policies  &  &  &  \\
\quad Capital requirement & -0.199*** & -0.187*** &  &  \\
 & (0.042) & (0.042) &  &  \\
\quad Reserve requirement & -0.056* & -0.040 &  &  \\
 & (0.029) & (0.029) &  &  \\
\quad Capital requirement*tax & 0.216*** & 0.217*** &  &  \\
 & (0.042) & (0.042) &  &  \\
\quad Reserve requirement*tax & 0.053* & 0.054** &  &  \\
 & (0.027) & (0.027) &  &  \\
\quad Capital requirement spillover & -0.024 & -0.024 & -0.019 &  \\
 & (0.018) & (0.018) & (0.017) &  \\
\quad Reserve requirement spillover & -0.018 & -0.017 & -0.011 &  \\
 & (0.011) & (0.011) & (0.012) &  \\
\quad Capital requirement*risk &  & -0.157** & -0.147** & -0.220*** \\
 &  & (0.065) & (0.063) & (0.072) \\
\quad Reserve requirement*risk &  & -0.162** & -0.155** & -0.187** \\
 &  & (0.077) & (0.075) & (0.085) \\
   Effects of corporate tax &  &  &  \\
\quad Tax rate & -0.040 & -0.041 &  &  \\
 & (0.044) & (0.044) &  &  \\
\quad Tax rate spillover & 0.144** & 0.144** & 0.092 &  \\
 & (0.056) & (0.056) & (0.056) &  \\
    Firm characteristics &  &  &  \\
\quad Risk & 0.036 & 0.043 & 0.041 & 0.062** \\
 & (0.029) & (0.026) & (0.026) & (0.027) \\
\quad Tangibility & 0.003 & 0.003 & 0.003 & 0.003 \\
 & (0.004) & (0.004) & (0.004) & (0.005) \\
\quad Log of fixed assets & 0.624*** & 0.621*** & 0.611*** & 0.688*** \\
 & (0.094) & (0.094) & (0.095) & (0.104) \\
\quad Profitability & -0.002 & -0.002 & -0.002* & -0.002 \\
 & (0.001) & (0.001) & (0.001) & (0.001) \\
\quad Opportunity & 0.060*** & 0.061*** & 0.073*** & 0.101*** \\
 & (0.016) & (0.016) & (0.017) & (0.023) \\
     Macroeconomic controls &  &  &  \\
\quad Inflation rate & -0.080 & -0.099 &  &  \\
 & (0.164) & (0.162) &  &  \\
\quad GDP growth rate & 0.088 & 0.088 &  &  \\
 & (0.106) & (0.106) &  &  \\
\quad Private credit to GDP & -0.055* & -0.055* &  &  \\
 & (0.033) & (0.033) &  &  \\
\quad Political risk & -0.084 & -0.080 &  &  \\
 & (0.073) & (0.073) &  &  \\
\quad Exchange rate risk & -0.028** & -0.029** &  &  \\
 & (0.014) & (0.014) &  &  \\
\quad Law and order & -0.200 & -0.219 &  &  \\
 & (0.211) & (0.210) &  &  \\
 &  &  &  &  \\
Observations & 354,059 & 354,059 & 354,082 & 354,082 \\
Number of multinationals & 22,929 & 22,929 & 22,930 & 22,930 \\
Year fixed effects & Yes & Yes & No & No \\
Multinational fixed effects & Yes & Yes & Yes & No \\
Country*year fixed effects & No & No & Yes & Yes \\
 Multinational*year fixed effects & No & No & No & Yes \\ \hline
\multicolumn{5}{c}{ Robust standard errors in parentheses} \\
\multicolumn{5}{c}{ *** p$<$0.01, ** p$<$0.05, * p$<$0.1} \\
\end{longtable}

	 	}
	 \end{small}
 
		\section{\cite{barth2013bank} dataset} \label{sec:barth}
	 
	 	\begin{small}
	 	{\setstretch{1.0}
	 		\begin{longtable}{lcccccc}\\
	\label{tab:summary barth}\\
	\multicolumn{7}{c}{Table \ref{tab:summary barth} - Summary statistics with alternative dataset}\\ \hline \hline\\
	
	Variable   &   Observations &    Mean &          SD&         Min&    Median &    Max\\
	
	\hline \endfirsthead
	
	\multicolumn{7}{c}{Table \ref{tab:summary barth} - Summary statistics with alternative dataset (\textit{Continued})}\\ \hline \hline\\
	
	Variable   &   Observations &    Mean &          SD&         Min&     Median  &  Max\\
	
	\hline \endhead
	\hline
	\multicolumn{7}{r}{{\textit{(Continued)}}}\\ \endfoot
	
	\endlastfoot
	
	Panel A: main variables         &      &        &        &        &       \\
\quad Financial leverage  &      392.716&        0.61&        0.27&        0.00&        0.65&        1.00\\
\quad Overall capital strigency&      392.716&        5.17&        1.31&        1.00&        6.00&        7.00\\
\quad Tax rate            &      392.693&        0.55&        0.12&        0.20&        0.56&        0.84\\
\quad Overall capital strigency spillover&      392.716&        0.16&        1.19&       -5.82&        0.03&        5.94\\
\quad Tax rate spillover  &      392.716&        0.02&        0.11&       -0.55&        0.01&        0.53\\
          &      &        &        &        &       \\
Panel B: firm-level controls         &      &        &        &        &       \\
\quad Tangibility         &      392.716&        0.32&        0.29&        0.00&        0.23&        1.00\\
\quad Log of fixed assets &      392.716&       14.39&        2.95&        5.51&       14.53&       21.27\\
\quad Profitability       &      392.716&        0.09&        0.17&       -1.33&        0.08&        0.70\\
\quad Opportunity         &      392.716&       -0.01&        0.11&       -0.42&       -0.00&        0.34\\
\quad Risk                &      392.716&        0.10&        0.16&        0.00&        0.06&        1.96\\
         &      &        &        &        &       \\
Panel C: country-level controls         &      &        &        &        &       \\
\quad Inflation rate      &      392.716&        0.02&        0.02&       -0.01&        0.02&        0.22\\
\quad GDP growth rate     &      392.716&        0.01&        0.03&       -0.15&        0.01&        0.10\\
\quad Private credit to GDP&      392.716&        0.99&        0.35&        0.21&        0.95&        2.51\\
\quad Policy rate         &      355.646&        2.16&        1.34&        0.20&        1.25&       12.54\\
\quad Political Risk      &      392.716&       78.70&        6.10&       56.50&       79.00&       94.00\\
\quad Exchange rate risk  &      392.716&        9.50&        0.71&        1.00&        9.50&       10.00\\
\quad Law and order       &      392.716&        4.86&        0.61&        1.50&        5.00&        6.00\\
\hline\end{longtable}
	 	}
	 \end{small}
	 
	 	\begin{small}
	 	{\setstretch{1.0}
	 		\begin{longtable}{lcccc}\\
	\label{reg:barth}\\
	\multicolumn{5}{c}{Table \ref{reg:barth} - Alternative dataset, macroprudential policies and capital structure}\\
	\multicolumn{5}{c}{(\textit{Dependent variable}: financial leverage)}
	\\ \hline \hline
	Model & (1) & (2) & (3) & (4)  \\ \hline
	&  &  &  \\ \endfirsthead
	\multicolumn{5}{c}{Table \ref{reg:barth} - Regression results, macroprudential policies and capital structure }\\
	\multicolumn{5}{c}{(\textit{Dependent variable}: financial leverage)\textit{(Continued)}}
	\\ \hline \hline
	Model & (1) & (2) & (3) & (4)  \\ \hline 
	&  &  &  & \\ \endhead
	\hline
	\multicolumn{5}{r}{{\textit{(Continued)}}}\\ \endfoot	
	\endlastfoot
	Effects of macroprudential policies  &  &  &  \\
\quad Overall capital strigency & 0.008*** & 0.008*** &  &  \\
 & (0.002) & (0.003) &  &  \\
\quad Overall capital strigency*tax & -0.015*** & -0.014*** &  &  \\
 & (0.004) & (0.004) &  &  \\
\quad Overall capital strigency spillover & 0.002** & 0.002** & 0.001 &  \\
 & (0.001) & (0.001) & (0.001) &  \\
\quad Overall capital strigency*risk &  & 0.004 & 0.006* & 0.007* \\
 &  & (0.003) & (0.003) & (0.003) \\
   Effects of corporate tax &  &  &  \\
\quad Tax rate & 0.138*** & 0.135*** &  &  \\
 & (0.027) & (0.027) &  &  \\
\quad Tax rate spillover & 0.029** & 0.029** & -0.007 &  \\
 & (0.014) & (0.014) & (0.014) &  \\
    Firm characteristics &  &  &  \\
\quad Risk & 0.047*** & 0.025 & 0.020 & 0.014 \\
 & (0.006) & (0.018) & (0.018) & (0.019) \\
\quad Tangibility & -0.263*** & -0.263*** & -0.258*** & -0.260*** \\
 & (0.006) & (0.006) & (0.006) & (0.006) \\
\quad Log of fixed assets & 0.012*** & 0.012*** & 0.012*** & 0.012*** \\
 & (0.001) & (0.001) & (0.001) & (0.001) \\
\quad Profitability & -0.166*** & -0.166*** & -0.165*** & -0.164*** \\
 & (0.005) & (0.005) & (0.005) & (0.006) \\
\quad Opportunity & 0.042*** & 0.042*** & 0.036*** & 0.037*** \\
 & (0.004) & (0.004) & (0.005) & (0.006) \\
     Macroeconomic controls &  &  &  \\
\quad Inflation rate & -0.363*** & -0.365*** &  &  \\
 & (0.061) & (0.061) &  &  \\
\quad GDP growth rate & -0.236*** & -0.236*** &  &  \\
 & (0.027) & (0.027) &  &  \\
\quad Private credit to GDP & 0.010*** & 0.010*** &  &  \\
 & (0.004) & (0.004) &  &  \\
\quad Political Risk & 0.003*** & 0.003*** &  &  \\
 & (0.000) & (0.000) &  &  \\
\quad Exchange rate risk & -0.004*** & -0.004*** &  &  \\
 & (0.001) & (0.001) &  &  \\
\quad Law and order & -0.033*** & -0.033*** &  &  \\
 & (0.003) & (0.003) &  &  \\
 &  &  &  &  \\
Observations & 392,693 & 392,693 & 392,716 & 392,716 \\
Number of multinationals & 33,415 & 33,415 & 33,416 & 33,416 \\
Year fixed effects & Yes & Yes & No & No \\
Multinational fixed effects & Yes & Yes & Yes & No \\
Country*year fixed effects & No & No & Yes & Yes \\
 Multinational*year fixed effects & No & No & No & Yes \\ \hline
\multicolumn{5}{c}{ Robust standard errors in parentheses} \\
\multicolumn{5}{c}{ *** p$<$0.01, ** p$<$0.05, * p$<$0.1} \\
\end{longtable}

	 	}
	 \end{small}
 
	
	
	\singlespacing
	\bibliography{C:/Users/u1273941/Research/Projects/macroprudential_capital_structure/analysis/code/text/bibliography/references}
	\bibliographystyle{C:/Users/u1273941/Research/Projects/macroprudential_capital_structure/analysis/code/text/bibliography/te}
	
	
	
%	\clearpage
	
%	\onehalfspacing
	

	
	

\end{document} 