\documentclass[12pt]{article}
\usepackage[utf8]{inputenc}
\usepackage{amssymb,amsmath,amsfonts,eurosym,geometry,ulem,caption,color,xcolor,multicol,setspace,sectsty,comment,footmisc,caption,natbib,pdflscape,subfigure,array,hyperref,verbatim,mathpazo,longtable,ntheorem,siunitx,booktabs,threeparttable}
%\usepackage[normalem]{ulem}
\usepackage[pdftex]{graphicx}
\graphicspath{{Imagens/}}
\usepackage{fullpage}
\usepackage{indentfirst}
\usepackage{changepage}

%\setlength{\pdfpagewidth}{8.5in} \setlength{\pdfpageheight}{11in}
%\setlength{\textheight}{8.5in} \setlength{\topmargin}{0.0in}
%\setlength{\headheight}{0.0in} \setlength{\headsep}{0.0in}
%\setlength{\leftmargin}{0.5in}
%\setlength{\oddsidemargin}{0.0in}
%\setlength{\parindent}{2em}
%\setlength{\parskip}{\baselineskip}%
%\setlength{\textwidth}{6.5in}
%\linespread{1.6}
\newcommand{\sym}[1]{\rlap{#1}}% Thanks to David Carlisle

\newcommand*{\captionsource}[2]{%
  \caption[{#1}]{%
    #1%
    \\\hspace{\linewidth}%
    \textbf{Source:} #2%
  }%
}
\newcommand{\horrule}[1]{\rule{\linewidth}{#1}} % Create horizontal rule command with 1 argument of height

%***************************************************
% Fix input with \midrule problem
%***************************************************

\makeatletter
\newcommand\primitiveinput[1]
{\@@input #1 }
\makeatother

%***************************************************
% Custom subcaptions
%***************************************************
% Note/Source/Text after Tables
\newcommand{\Figtext}[1]{%
	\begin{tablenotes}[para,flushleft]
		\hspace{6pt}
		\hangindent=1.75em
		#1
	\end{tablenotes}
}

\newcommand{\Fignote}[1]{\Figtext{\emph{Note:~}~#1}}
\newcommand{\Figsource}[1]{\Figtext{\emph{Source:~}~#1}}
\newcommand{\Starnote}{\Figtext{* p < 0.1, ** p < 0.05, *** p < 0.01. Standard errors in parentheses.}}% Add significance note with \starnote

%***************************************************
% DO NOT KNOW
%***************************************************


\onehalfspacing
\newtheorem{theorem}{Theorem}
\newtheorem{corollary}[theorem]{Corollary}
\newtheorem{proposition}{Proposition}
\newenvironment{proof}[1][Proof]{\noindent\textbf{#1.} }{\ \rule{0.5em}{0.5em}}

\newtheorem{hyp}{Hypothesis}
\newtheorem{subhyp}{Hypothesis}[hyp]
\renewcommand{\thesubhyp}{\thehyp\alph{subhyp}}

\newcommand{\red}[1]{{\color{red} #1}}
\newcommand{\blue}[1]{{\color{blue} #1}}

\newcolumntype{L}[1]{>{\raggedright\let\newline\\arraybackslash\hspace{0pt}}m{#1}}
\newcolumntype{C}[1]{>{\centering\let\newline\\arraybackslash\hspace{0pt}}m{#1}}
\newcolumntype{R}[1]{>{\raggedleft\let\newline\\arraybackslash\hspace{0pt}}m{#1}}

\geometry{left=1.2in,right=1.2in,top=1.2in,bottom=1.2in}
\begin{document}
	
%	\begin{titlepage}
		\title{\underline{Bank Regulation} and Capital Structure: Evidence from Multinationals}
		\author{Lucas Avezum}
		%\author{Lucas Avezum\thanks{abc} \and \and Harry Huizinga\thanks{abc} \and Louis Raes\thanks{abc}}
		\date{\today}
		\maketitle
		%\begin{abstract}
		%	\noindent This paper studies how macroprudential policies, specifically reserve and capital requirements, relate to non-financial firms' capital structure. We first present a model where macroprudential instruments affect banks' costs which in turn are transmitted to firms' leverage through their cost of debt. Using a firm-level dataset of multinationals hosted in 37 countries we are able to identify domestic and international effects. The empirical results show that tighter capital requirements reduce firms' leverage while multinationals have an extra incentive to lower debt levels in subsidiaries hosted in countries with relative higher reserve requirements. Moreover, the domestic effect is found to be stronger for riskier firms.  \\
		%	\vspace{0in}\\
			%\noindent\textbf{Keywords:} key1, key2, key3\\
		%	\vspace{0in}\\
			%\noindent\textbf{JEL Codes:} key1, key2, key3\\
			
		%	\bigskip
	%	\end{abstract}
	%	\setcounter{page}{0}
	%	\thispagestyle{empty}
%	\end{titlepage}
%	\pagebreak \newpage
	
	
	
	
	\doublespacing
	
\normalem

	 
	\section{Introduction} \label{sec:intro}
	 Table \ref{reg:base_year} shows the results of the analysis with a few of the measures from \cite*{huizinga2008capital}. Those are: 
	 \begin{itemize}
	 	\item \underline{Restriction on banking activities}: the extent to which banks may engage in securities, insurance and real estate activities (higher values indicate more restrictive)
	 	\item \underline{Financial conglomerates restrictiveness}: restrictions on banks' ownership of nonfinancial firms and on non-bank firms owning banks (higher values indicate more restrictive),
	 	\item  \underline{Initial capital strigency}: whether certain funds may be used to initially capitalize a bank and whether they are official (higher values indicate greater stringency),
	 	\item \underline{Official supervisory power}: whether the supervisory authorities have the authority to take specific actions to prevent and correct problems (higher values indicate greater power),
	 	\item \underline{Private monitoring}: whether there incentives/ability for the private monitoring of firms (higher values indicating more private monitoring),
	 	\item \underline{Moral hazard}: degree to which actions taken to mitigate moral hazard (higher values indicate greater mitigation of moral hazard).
	 \end{itemize}
	 
	 	 	\begin{small}
	 	{\setstretch{1.0}
	 		\begin{longtable}{lrrrrr}\\
	 			\label{tab:summary}\\
	 			\multicolumn{6}{c}{Table \ref{tab:summary} - Summary statistics, bank regulation, leverage and controls}\\
	 		 \hline \hline \addlinespace  & Mean & SD & Min & Med & Max  \\
	 		   \endfirsthead
	 			\multicolumn{6}{c}{Table \ref{tab:summary} - Summary statistics, bank regulation, leverage and controls}\\
	 		 \hline \hline \addlinespace    & Mean & SD & Min & Med & Max  \\ \hline  \endhead
	 			\hline
	 			\multicolumn{6}{r}{{\textit{(Continued)}}}\\ \endfoot
	 			\multicolumn{6}{l}{{Notes: .}}\\ 	
	 			\endlastfoot
	 			\primitiveinput{C:/Users/u1273941/Research/Projects/macroprudential_capital_structure/analysis/output/tables/summary/summary.tex}
	 			\hline 			
	 		\end{longtable}	
	 	}
	 \end{small}
	 
	 Columns (1) and (2) include multinational and year fixed effects while columns (3) and (4) include multinational*year fixed effects, hence the spillover coefficient cannot be estimated. Moreover, columns (2) and (4) include the interaction of each of this variables with the tax variable.  
	  
\newgeometry{left=0.6in,right=0.6in,top=1.2in,bottom=1.2in}	    
 	\begin{landscape}	
	 	\begin{small}
	 	{\setstretch{1.0}
\begin{longtable}{lccccccc}\\
	\label{reg:base_year}\\
	\multicolumn{8}{c}{Table \ref{reg:base_year} - Regression results, bank regulation and firms' capital structure}\\
	\multicolumn{8}{c}{(\textit{Dependent variable}: financial leverage)}
	\\ \hline \hline \addlinespace
	Model & (1) & (2) & (3) & (4) & (5) & (6) & (7)  \\  \endfirsthead
	\multicolumn{8}{c}{Table \ref{reg:base_year} - Regression results, bank regulation and firms' capital structure }\\
	\multicolumn{8}{c}{(\textit{Dependent variable}: financial leverage)\textit{(Continued)}}
	\\ \hline \hline \addlinespace Model & (1) & (2) & (3) & (4) & (5) & (6) & (7) \\ \hline \\ \endhead
	\hline
	\multicolumn{8}{r}{{\textit{(Continued)}}}\\ \endfoot
	\multicolumn{8}{l}{{Notes: Robust standard errors (in parentheses) clustered at the multinational level.}}\\ 	
	\endlastfoot
			\primitiveinput{C:/Users/u1273941/Research/Projects/macroprudential_capital_structure/analysis/output/tables/regressions/benchmark_table.tex}
				\hline 			
	\end{longtable}	
 			}
 		\end{small}
	\end{landscape}

\restoregeometry
	
	\singlespacing
	\bibliography{C:/Users/u1273941/Research/Projects/macroprudential_capital_structure/analysis/code/text/bibliography/references}
	\bibliographystyle{C:/Users/u1273941/Research/Projects/macroprudential_capital_structure/analysis/code/text/bibliography/te}
	
	
	
%	\clearpage
	
%	\onehalfspacing
	

	
	

\end{document} 