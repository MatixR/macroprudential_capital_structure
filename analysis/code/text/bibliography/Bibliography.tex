\documentclass[12pt,runningheads]{article}
\usepackage[utf8]{inputenc}
\usepackage{amsmath,amssymb,hyperref,array,xcolor,multicol,verbatim,mathpazo,caption}
%\usepackage[normalem]{ulem}
\usepackage[pdftex]{graphicx}
\graphicspath{{Imagens/}}
\usepackage{fullpage}
\usepackage{natbib}
\usepackage{indentfirst}


\setlength{\pdfpagewidth}{8.5in} \setlength{\pdfpageheight}{11in}
%\setlength{\textheight}{8.5in} \setlength{\topmargin}{0.0in}
\setlength{\headheight}{0.0in} \setlength{\headsep}{0.0in}
\setlength{\leftmargin}{0.5in}
\setlength{\oddsidemargin}{0.0in}
\setlength{\parindent}{2em}
\setlength{\parskip}{\baselineskip}%
\setlength{\textwidth}{6.5in}
%\linespread{1.6}
\newcommand*{\captionsource}[2]{%
  \caption[{#1}]{%
    #1%
    \\\hspace{\linewidth}%
    \textbf{Source:} #2%
  }%
}
\newcommand{\horrule}[1]{\rule{\linewidth}{#1}} % Create horizontal rule command with 1 argument of height

\begin{document}


\title{Bibliography \ for:\\ \textbf{
   Macroprudential Policy Spill-over: Evidence from Multinational Firms' Capital Structure \large}} 


%\vspace*{20pt}
\author{Lucas Avezum \\
%EndAName
Tilburg University \vspace*{20pt}}
\date{\today }

\maketitle
\section{Topic}
\cite*{ayyagari2017credit} provided the first attempt to relate macroprudential policies and firm's credit growth.They found that the effectiveness of macroprudential policies at smoothing credit depends on firm location (emerging or developed country), size, type of debt (short or long term) and type of instruments (borrower-target or financial institution target).

\cite*{buch2017international} discuss findings on macroprudential policies spill-over to banking lending growth. On average, the economic significance of international spill-overs were found to be not large.

\cite*{jimenez2012macroprudential} finds that tighter policy measures tend to decrease credit growth.

 More regulated bank sector improve financial stability which may increase firms' optimal leverage. \cite*{beck2004bank} and \cite{demirgucc1998law} found positive relation of firms' external financing to finance and banking sector development.
 
\section{Methodology}
\cite*{huizinga2008capital} build estimation strategy to identify debt shift between firms within one multinational. Assuming an optimal capital structure at the multinational level, diferences in macroprudential policies across countries might incentive debt shifting within multinational subsidiaries through a differential in credit conditions channel.



\bibliography{references}
\bibliographystyle{te}


\end{document} 