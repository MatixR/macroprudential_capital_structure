\documentclass[12pt]{article}
\usepackage[utf8]{inputenc}
\usepackage{amssymb,amsmath,amsfonts,eurosym,geometry,ulem,caption,color,xcolor,multicol,setspace,sectsty,comment,footmisc,caption,natbib,pdflscape,subfigure,array,hyperref,verbatim,mathpazo,longtable}
%\usepackage[normalem]{ulem}
\usepackage[pdftex]{graphicx}
\graphicspath{{Imagens/}}
\usepackage{fullpage}
\usepackage{indentfirst}
\usepackage{changepage}

%\setlength{\pdfpagewidth}{8.5in} \setlength{\pdfpageheight}{11in}
%\setlength{\textheight}{8.5in} \setlength{\topmargin}{0.0in}
%\setlength{\headheight}{0.0in} \setlength{\headsep}{0.0in}
%\setlength{\leftmargin}{0.5in}
%\setlength{\oddsidemargin}{0.0in}
%\setlength{\parindent}{2em}
%\setlength{\parskip}{\baselineskip}%
%\setlength{\textwidth}{6.5in}
%\linespread{1.6}
\newcommand*{\captionsource}[2]{%
  \caption[{#1}]{%
    #1%
    \\\hspace{\linewidth}%
    \textbf{Source:} #2%
  }%
}
\newcommand{\horrule}[1]{\rule{\linewidth}{#1}} % Create horizontal rule command with 1 argument of height

\onehalfspacing
\newtheorem{theorem}{Theorem}
\newtheorem{corollary}[theorem]{Corollary}
\newtheorem{proposition}{Proposition}
\newenvironment{proof}[1][Proof]{\noindent\textbf{#1.} }{\ \rule{0.5em}{0.5em}}

\newtheorem{hyp}{Hypothesis}
\newtheorem{subhyp}{Hypothesis}[hyp]
\renewcommand{\thesubhyp}{\thehyp\alph{subhyp}}

\newcommand{\red}[1]{{\color{red} #1}}
\newcommand{\blue}[1]{{\color{blue} #1}}

\newcolumntype{L}[1]{>{\raggedright\let\newline\\arraybackslash\hspace{0pt}}m{#1}}
\newcolumntype{C}[1]{>{\centering\let\newline\\arraybackslash\hspace{0pt}}m{#1}}
\newcolumntype{R}[1]{>{\raggedleft\let\newline\\arraybackslash\hspace{0pt}}m{#1}}

\geometry{left=1.0in,right=1.0in,top=1.0in,bottom=1.0in}
\begin{document}
	
	\begin{titlepage}
		\title{Macroprudential Policies and Capital Structure: Evidence from Multinationals \thanks{Preliminary version}}
		\author{Lucas Avezum\thanks{abc}}
		%\author{Lucas Avezum\thanks{abc} \and \and Harry Huizinga\thanks{abc} \and Louis Raes\thanks{abc}}
		\date{\today}
		\maketitle
		\begin{abstract}
			\noindent Placeholder\\
			\vspace{0in}\\
			\noindent\textbf{Keywords:} key1, key2, key3\\
			\vspace{0in}\\
			\noindent\textbf{JEL Codes:} key1, key2, key3\\
			
			\bigskip
		\end{abstract}
		\setcounter{page}{0}
		\thispagestyle{empty}
	\end{titlepage}
	\pagebreak \newpage
	
	
	
	
	\doublespacing
	
	
	\section{Introduction} \label{sec:introduction}
	As \cite*{blinder2016necessity} documented, the mandate of central banks has expanded considerably. Triggered mostly by the 2008 financial crisis, financial stability is perceived by many central banks as a goal together with price stability. To achieve the former, several macro-prudential instruments were included or extended in central banker's toolbox. However, the authors also note that the extent in which macro-prudential policy will be used in the future is still unclear. Given its early stage, the effectiveness and externalities of macro-prudential instruments have still to be properly evaluated. 
	
	This paper studies how macro-prudential policies relate to firms' leverage. We also analyze potential international spillovers. More specifically, we show that changes in one country's macro-prudential tool affect firm's leverage decision hosted in another country. 
	
	In order to assess potential international spillovers from macro-prudential policy our estimation strategy is based on \cite*{huizinga2008capital}. Although their work was designed to study tax incentive to shift debt, we believe that the same strategy is also suitable to our case. Multinational groups take into account both tax and non-tax factors when deciding their optimal financial structure. For instance, looser credit market in one country may lead to relative higher leverage at subsidiaries in this country, as the multinational will take advantage of better financing conditions. Given an optimal leverage at the multinational level, subsidiaries at other countries should reduce external financing. As macro-prudential instruments aim at smoothing credit markets, to the degree of their domestic effectiveness, international spillover in the form of debt shifting are plausible.
	
	Empirical research on the effects of prudential policies has increased considerably after the 2008 financial crisis. Several authors rely on cross-country aggregate data to study the relationship between macro-prudential policies and financial indicators. \cite*{cerutti2017use} introduces a dataset containing the usage of 12 macro-prudential policies at 119 countries from 2000 to 2013. They document that prudential instruments are mostly used in developing countries where they also appear to be more effective. Among their findings, borrower-related policies, such as limits on LTVs and DTIs ratios are associated with reductions in credit growth and house prices. The author also provide an extensive review of the literature on macro-prudential policies. 
	
	A second line of studies provides micro-level evidence. With loan-level data \cite*{jimenez2012macroprudential} study the  effects of capital buffers in Spain. \cite*{aiyar2014does} find lending to leak through foreign branches following an increase in capital requirements to regulated banks. Their work shows that capital requirements in UK were effective in curbing lending for regulated banks, but the leakage is substantial. Most studies in this second group in the literature focus on few instruments, on the bank or housing sectors and on one country. An exception is \cite*{ayyagari2017credit} that uses firm-level data to relate macro-prudential policies across 119 countries to firm's credit growth. They find that the effectiveness of prudential tools at smoothing credit depend on the firm's location (emerging or developed country), size, debt maturity and the type of instrument (borrower-target or financial institution target). We extend on this line of research by also providing evidence of international spillovers.
	
	Our work is also closely related to the series of papers developed under the International Banking Research Network (IBRN) 2015 initiative. \cite*{buch2017cross} summarize the methodology and provide a meta-study. The authors also rely on multinational relationships to identify international spillovers. However, their analysis is restricted to the effects on banking loans, while we study changes in firm's capital structure. Among their findings, the effects of prudential tools are heterogeneous on type and location. On average, the economic significance of international spillover is found to be small.  
	
	Finally, this paper also builds on the optimal capital structure literature. \cite*{demirgucc1998law}, \cite*{beck2004bank}, show how institutions impact leverage decisions. Our contribution is to study if policies intended to affect credit markets are incorporated to firms beyond the financial sector.   
	
   The main contribution of the paper is to show that macro-prudential policy affect firm's leverage by changing their credit condition relative to other firms within the same multinational group. Our findings provide stronger evidence of prudential tools' spillover.
	
	The paper proceeds as follows: Section \ref{sec:strategy} explains the empirical strategy, Section \ref{sec:data} describes the data while the results are discussed in Section \ref{sec:result}. We provide extensions to our baseline model in Section \label{sec:discussion}. Section \ref{sec:conclusion} concludes. 
	
	\section{Empirical strategy} \label{sec:strategy}
	We extend \cite{huizinga2008capital} by allowing the cost of debt to vary. Several factors could drive this variation, but we are interested in the effects of different macroprudential policies. IMPROVE
	\subsection{Balance sheets}
	 \label{subsec:balancesheet}
	We consider a multinational group that is composed of $n-1$ subsidiaries and the parent firm. Each subsidiary has assets $A_i$ and is financed by external debt $L_i$ and parent firm's equity $I_i$. For each subsidiary the balance sheet is
		\begin{equation}
	\begin{aligned}
	 A_i=I_i+L_i. 
	\end{aligned}
	\label{eq:sub balance sheet}
	\end{equation}

	 For simplicity, we assume that the parent firm is the sole owner of each subsidiary. Hence, if $A_p$, $E_p$ and $L_p$ are, respectively, the assets, equity and debt of the parent firm, its balance sheet can be stated as  
		\begin{equation}
	\begin{aligned}
	A_p+\sum_{i=1}^{n-1}I_i=E_p+L_p. 
	\end{aligned}
	\label{eq:sparent balance sheet}
	\end{equation}
	Financial leverage is defined as total liabilities to total assets, that is, $\lambda_i=L_i/A_i$. At the multinational level, total leverage is
		\begin{equation}
	\begin{aligned}
	\lambda_m=\frac{\sum_{i=1}^{n}L_i}{\sum_{i=1}^{n}A_i}=\sum_{i=1}^{n}\lambda_i\rho_i, 
	\end{aligned}
	\label{eq:sparent balance sheet}
	\end{equation} 
	 where $\rho_i=A_i/\sum_{i=1}^{n}A_i$ is the asset share of firm $i$ within the multinational. The second equality is reached by replacing the definition of leverage for a subsidiary in the first equality of equation \ref{eq:sparent balance sheet}. We assume that any adjustment in leverage is done by changes in debt positions rather than assets. 
	\subsection{Costs associated with leverage}
		 \label{subsec:costs}
	We assume that the debt of any subsidiary firm is implicitly or explicitly guaranteed by the parent firm. Consequently, the expected cost of bankruptcy associated with higher leverage is contemplated at the multinational level. We assume the following quadratic expected bankruptcy cost function ($C_m$):
	 
	\begin{equation}
	\begin{aligned}
	C_m=\frac{\gamma}{2}\lambda_m^2\bigg(\sum_{i}^{n}A_i\bigg).
	\end{aligned}
	\label{eq:cost bankruptcy}
	\end{equation}
	
	(IMPROVE) We also consider costs of deviating from the optimal leverage at the subsidiary level. Those costs are related to incentives to local managers and assumed to be quadratic in the deviation of $\lambda_i$ from the optimal level $\lambda^*$ as follows:  
	
	\begin{equation}
	\begin{aligned}
	C_i=\frac{\mu}{2}(\lambda_i-\lambda_i^*)^2A_i-\frac{\mu}{2}(\lambda_i^*)^2A_i, \quad i=1,...,n.
	\end{aligned}
	\label{eq:agency cost}
	\end{equation}
	We extend \cite{huizinga2008capital} model by allowing the cost of debt to vary across subsidiaries. Macroprudential policies have the potential to alter credit market condition which in turn may affect the cost of debt at subsidiary level. We assume a competitive banking sector. The costs for a bank when lending to a firm are the opportunity cost to not lend at the discount rate in the economy ($\delta$) and costs that arise from macroprudential policies. The latter costs are assumed to be proportional to the prudential policies and also the discount rate. The cost of debt can be written as:
	  \begin{equation}
	  \begin{aligned}
	  r_i=\delta(1+\phi\Pi_i)
	  \end{aligned}
	  \label{eq:cost of debt}
	  \end{equation}
	where $\phi$ is a parameter vector specific to the set of macroprudential instruments $\Pi_i$ that are implemented in the country of firm $i$. Note that the elements of vector $\phi$ could be positive as negative parameters. Substitution effects can be present as some tools are target at non-firm related products.
		\subsection{Multinational's value}
			 \label{subsec:value}
	Let $R_i$ be the gross revenue of firm $i$. For simplicity, we assume a positive and constant cash flow for each firm. Therefore, considering the debt taken, the tax payments, the perpetuity of the cash flow, the agency costs associated with leverage, and the fact that interest payments are deductible to taxable income, firm $i$'s value is equal to
	 \begin{equation}
	\begin{aligned}
	V_i^L=L_i+\frac{(R_i-r_iL_i)}{\delta}(1-\tau_{i})-C_i,
	\end{aligned}
	\label{eq:value 1}
	\end{equation}	
	 where $\tau_{i}$ is the effective tax rate payed by firm $i$. Replacing the cost of debt equation \ref{eq:cost of debt} on the value function \label{eq:value 2} and rearranging terms:
	  \begin{equation}
	 \begin{aligned}
	 V_i^L=V_i^U+\tau_{i}L_i-\phi\Pi_iL_i+\tau_{i}\phi\Pi_iL_i-C_i,
	 \end{aligned}
	 \label{eq:value 2}
	 \end{equation}	
	 where $V_i^U=R_i(1-\tau_{i})/\delta$ is the value of the firm if completely unleveraged. The second term of the right hand side is the usual debt tax shield. The third term is the effect of macroprudential policies. For an instrument that increase the cost of debt, the value of the firm decreases when the instrument is tightened. The fourth term is an interaction term between the level of corporate tax and macroprudential policies. Therefore, according to this expression, the magnitude of the effect of both tax and prudential policy on the value of the firm depend on each other's level. 
	 \subsection{Optimal leverage}
	 			 \label{subsec:opt_leverage}
	The multinational firm's objective is to maximize its value by choosing each establishment (subsidiaries and parent firm) debt level taking into account the costs and benefits associated with leverage. The problem can be stated as follows: 
	
	 \begin{equation}
	\begin{aligned}
	\max_{L_i}V_L=V_u+\sum_{i}^{n}\tau_iL_i-C_m-\sum_{i}^{n}C_i,
	\end{aligned}
	\label{eq:4}
	\end{equation}
	where $V_L$, and $V_u$ are the multinational firm's values when leveraged and completely unleveraged, respectively. The second term of the right hand side are the gains from interest rate deduction on taxable income while the remaining terms are the costs associated to higher leverage. The first order conditions of the problem \ref{eq:4} are:
	\begin{equation}
	\begin{aligned}
	\tau_i-\gamma\lambda_m-\mu(\lambda_i-\lambda^*-\phi\Pi_i)=0, \quad i=1,...,n.
	\end{aligned}
		\label{eq:5}
	\end{equation}
	Subtracting the first order condition for subsidiary $j$ from the first order condition for subsidiary $i$ we find
	\begin{equation}
	\begin{aligned}
	\lambda_i-\lambda_j=\frac{1}{\mu}(\tau_i-\tau_j)+\phi(\Pi_i-\Pi_j).
	\end{aligned}
	\label{eq:6}
	\end{equation}
	Using equations \ref{eq:5}, \ref{eq:6} and the definition of $\lambda_m$, the optimal leverage level for subsidiary $i$ can be written as  
	\begin{equation}
	\begin{aligned}
	\lambda_i=&\frac{\mu\lambda^*}{(\mu+\gamma)}+\frac{\tau_i}{(\mu+\gamma)}+\frac{\phi\Pi_i}{(\mu+\gamma)}\\
	&+\frac{\gamma}{\mu(\mu+\gamma)}\sum_{j=1}^{n}(\tau_i-\tau_j)\rho_j+\frac{\phi\gamma}{(\mu+\gamma)}\sum_{j=1}^{n}(\Pi_i-\Pi_j)\rho_j
	\end{aligned}
	\label{eq:7}
	\end{equation}
	Equation \ref{eq:7} provides the theoretical foundation for our benchmark regression: 
	\begin{equation}
	\begin{aligned}
	\lambda_{i,t}=\alpha_0\tau_{i,t}+\beta_0\Pi_i+\alpha_1\sum_{j=1}^{n}(\tau_{i,t}-\tau_{j,t})\rho_{j,t}+\beta_1\sum_{j=1}^{n}(\Pi_{i,t}-\Pi_{j,t})\rho_j
	+\Gamma X_{t}+f_{i}+f_{t}+\varepsilon_{i,t},
	\label{eq:8}
	\end{aligned}
	\end{equation}
	
	where $i$ denotes firm (either subsidiary or parent) and $t$ year. The dependent variable is a financial leverage measure of the firm. $\Pi_{i,t}$ is the index for a macro-prudential instrument used in host country. $\rho_{i,t}$ is the asset share of firm $i$ within the multinational group. $f_{i}$, $f_{t}$ and $f_{s}$ are respectively the firm, time and sector fixed effects. $X_t$ is a set of firm- and country-level control variables. We also include the corporate tax level at the host country and the tax incentive to shift debt. We are interested in both $\beta_0$ and $\beta_1$, respectively, the domestic and international effect of prudential policies on firm's capital structure choice.
	
	We also perform a second model specification to test the domestic effect. The assumption that, at least, part of the capital structure decision is taken at the multinational level allow us to control for unobservables characteristics at the multinational firm level at each year as follows   
	
	\begin{equation}
	\begin{aligned}
	\lambda_{im,t}=\alpha_0\tau_{i,t}+\beta_0\Pi_i
	+\Gamma X_{t}+f_{m,t}+f_{s}+\varepsilon_{i,t}.
	\label{eq:9}
	\end{aligned}
	\end{equation}
	
	The model includes multinational*year ($f_{m,t}$) and industry ($f_{s}$) fixed effect, besides the corporate tax, the macro-prudential indexes and the firm- and country-level controls. The drawbacks of this specification are that we have to drop the international effect variables and the sample is restricted to firms that belong to a multinational group. 
	
	Lastly, we allow for the possibility that the capital structure decision is not static and adjustments towards the optimal level might take time. Hence,  we perform two partial adjustment models based on \cite*{flannery2006partial}. The first model includes the domestic and international effects of tax and macro-prudential policies, firm- and country-level controls, firm and time fixed effects and one year lag of the dependent variable
	
	\begin{equation}
	\begin{aligned}
	\lambda_{i,t}=&(\Lambda\alpha_0)\tau_{i,t}+(\Lambda\beta_0)\Pi_i+(\Lambda\alpha_1)\sum_{j=1}^{n}(\tau_{i,t}-\tau_{j,t})\rho_{j,t}+(\Lambda\beta_1)\sum_{j=1}^{n}(\Pi_{i,t}-\Pi_{j,t})\rho_j\\
	&+(\Lambda\Gamma)X_{t}+(1-\Lambda)\lambda_{i,t-1}+f_i+f_t+\varepsilon_{i,t},
	\label{eq:10}
	\end{aligned}
	\end{equation}
	Where $\Lambda$ is the speed of adjustment towards the optimal leverage. In order to account for the endogeneity that arises when we control for firm fixed effects, equation \ref{eq:10} is estimated following \cite*{arellano1991some}. We also perform an partial adjustment model based on equation \ref{eq:9} as follows
	\begin{equation}
	\begin{aligned}
	\lambda_{im,t}=(\Lambda\alpha_0)\tau_{i,t}+(\Lambda\beta_0)\Pi_i
	+(\Lambda\Gamma) X_{t}+(1-\Lambda)\lambda_{i,t-1}+f_{m,t}+f_{s}+\varepsilon_{i,t}.
	\label{eq:11}
	\end{aligned}
	\end{equation}
	 Model \ref{eq:11} has the advantage that it can be estimated by OLS since endogeneity is not a issue when controlling for the multinational*year fixed effects. 
	\section{Data} \label{sec:data}
	\subsection{Macroprudential policy} \label{subsec:MPI}
	
	We rely on the efforts of \cite*{cerutti2017changes} for our measure of macroprudential policy \footnote{In turn, their work builds on \cite*{cerutti2015use}}, which was also developed within the IBRN macroprudential study initiative. The measures consist of indexes for nine instruments: reserve requirements on local currency, reserve requeriments on foreign currency, loan-to-value ratio limits, concentration limits, interbank exposure, general capital requirements, real estate loans capital buffer, consumer loans capital buffer and other loans capital buffers. The time unit is a quarter and the database spans from the first quarter of 2001 to the last quarter of 2014. Sixty-four countries are included. Each quarter data point is a positive, negative or zero entry meaning tightening, loosening or no change in the instrument, respectively. These indexes not only track the use but also the intensity of each change. A cumulative measure of each instrument is also available. Since we are interest at the effect of macro-prudential policy on the leverage level, the cumulative indexes are the appropriate choice to our analysis. 
	
	Due to firm level data being at a yearly basis, we create 4-quarter moving averages of each macro-prudential index and match these data points according to the balance sheet closing date for each firm and year. As an example, take a firm that has reported its balance sheet on May 2012. The corresponding macro-prudential index for this firm is an average between the third and fourth quarters of 2011 and the first and second quarters of 2012. 
	  
	\subsection{Firm level} \label{subsec:firm}
	Firm level data and ownership relationships are taken from the Orbis database compiled by Bureau Van Dijk.	The dataset consist of worldwide accounting and ownership information on both private and public owned companies. Ownership is considered if one firm owns at least 50\% of another in a given year. We call the second a subsidiary firm. If a company owns one or more firms but is itself owned by another, this firm is called an intermediate and for all purposes is also considered as a subsidiary. A parent firm is the ultimate owner of a group, that is, a firm that owns one or more companies but none of its shareholders have more than 50\% of its shares.
	
	Given our interest in cross-border effects, we keep only multinationals groups, that is, ownership networks that have firms in at least two countries. Banks, insurance and financial related companies\footnote{Check IBRN studies on the banking sector} and firms in the utility sector\footnote{NACE code} are also excluded since their capital structure decision is constrained by regulation.
	
	The benchmark sample consists of around 2.4 million firms from 2007 to 2015 resulting in approximately 8.8 million firm-year observations. Table \ref{tab:number} provides information on the amount of parent and subsidiary firms per host and home country in our sample. The total number of parent and subsidiary firms are 64,810 and 305,367, respectively. There are 51,255 subsidiaries for which ownership has changed during the period analyzed at least once, as shown in table \ref{tab:movers}.
	
	Statistics on firm level data are summarized in table 4. Financial leverage is defined as the ratio of total non-equity liabilities to total assets. Adjusted financial leverage is a similar measure but subtracting cash and equivalent from both the numerator and the denominator. Our variable of interest is the macro-prudential incentive to shift debt described in section \ref{sec:strategy}. Among the control variables tangibility is construct as the ratio of fixed assets to total assets. While tangible assets can be used as collateral, implying a positive relationship between tangibility and leverage, the depreciation of fixed assets reduces taxable income. Hence, tangible assets can also be substitute to debt as tax shield. We use the logarithm of fixed assets to proxy for firm size, which is expected to be positively associated to leverage. Profitability is the ratio of earning before interest, tax, depreciation and amortization (EBITDA) to total assets. Higher profits may facilitate access to credit but firms with larger cash flow may also opt to finance themselves with retained earnings. Thus, the relationship between profitability and leverage is ambiguous. 
	%As an alternative measure, adjusted tangibility is the ratio of tangible fixed assets to total assets. 
	%%Growth opportunities is BLA BLA
	\subsection{Country level} \label{subsec:country}
	Table 5 provides statistics at the country level. We control for political risk using the index from the \textit{International Country Risk Guide}. Higher scores mean lower political risk. \cite*{kesternich2010afraid} find that political risk can both increase or decrease firm leverage. A more unstable political environment may discourage banks to provide loans but also, parent firms might want to reduce their value at risk by leveraging their subsidiaries operations. We also include two macro controls from the \textit{World Development Indicators} database from the World bank. Private credit to GDP is the share of credit to the private sector to GDP and is a proxy to financial development. We expect financial development to affect leverage positively. Inflation is the annual log change in the consumer price index.     	
	     
	\section{Results} \label{sec:result}
	 Tables \ref{tab:reg1} and \ref{tab:reg2} present the model \ref{eq:8} estimated for financial leverage. All the regressions in the tables include firm and year fixed effects. The results suggest that the domestic and international components of macro-prudential policies affect financial leverage on opposite directions. the reserve requirements on local and foreign currency, the interbank exposure limit and all the capital buffers have a positive domestic effect while the incentive to shift debt is negative. To illustrate the impact of both effects, we consider a multinational firm that consists of the parent firm and one subsidiary with assets of equal size. A tightening on the reserve requirements on foreign currency in the country of the subsidiary, on the one hand, has a positive domestic effect on financial leverage of 0.5\%. On the other hand, there is an international spillover to the parent firm that reduces financial leverage by 0.02\% (the estimated coefficient from column (2), -0.004, times the  asset share of the parent firm, 0.5). The net effect of the tightening policy is 0.48\% for the subsidiary firm and 0.02\% for the parent firm. A remark on the magnitude of the coefficients estimated for the macro-prudential instruments is necessary. Although we know the direction in which a prudential policy has been changed, the degree of this change varies across countries and time. Therefore, the example shown above is intended to be an explanation of the effects and we focus on statistical significant and coefficient sign for the remaining results. 
	 
\begin{table}	
	\caption{}       
	\scalebox{0.8}{\begin{tabular}{lccccc}
\multicolumn{6}{c}{Effect of macroprudential policies on firm's financial leverage} \\ \hline
 & (1) & (2) & (3) & (4) & (5) \\
VARIABLES &  &  &  &  &  \\ \hline
 &  &  &  &  &  \\
Reserve req. on local currency & 0.000 & 0.104*** & 0.097*** & -0.026 &  \\
 & (0.006) & (0.020) & (0.023) & (0.122) &  \\
Reserve req. on foreign currency & -0.017*** & 0.406*** & 0.330*** & -0.427 &  \\
 & (0.003) & (0.059) & (0.067) & (0.575) &  \\
LTV ratio limits & -0.066*** & 0.056 & 0.050 & -0.455*** &  \\
 & (0.004) & (0.041) & (0.052) & (0.163) &  \\
Reserve req. on local currency spillover &  & -0.056*** & -0.050*** &  & -0.063*** \\
 &  & (0.014) & (0.018) &  & (0.018) \\
Reserve req. on foreign currency spillover &  & -0.018 & -0.029 &  & -0.039** \\
 &  & (0.017) & (0.018) &  & (0.019) \\
LTV ratio limits spillover &  & 0.002 & 0.015 &  & 0.010 \\
 &  & (0.010) & (0.014) &  & (0.015) \\
Reserve req. on local currency*tax &  & -0.099*** & -0.090*** & -0.027 &  \\
 &  & (0.021) & (0.024) & (0.125) &  \\
Reserve req. on foreign currency*tax &  & -0.442*** & -0.363*** & 0.438 &  \\
 &  & (0.060) & (0.068) & (0.599) &  \\
LTV ratio limits*tax &  & -0.117*** & -0.112** & 0.489*** &  \\
 &  & (0.043) & (0.054) & (0.173) &  \\
Tax rate & 0.055*** & 0.006 & 0.009 & 0.200*** &  \\
 & (0.006) & (0.012) & (0.015) & (0.052) &  \\
Tax rate spillover & 0.001 & -0.008 & -0.017 &  & -0.003 \\
 & (0.018) & (0.018) & (0.032) &  & (0.032) \\
Tangibility & -0.004*** & -0.004*** & -0.005*** & -0.012*** & -0.006*** \\
 & (0.001) & (0.001) & (0.001) & (0.002) & (0.001) \\
Log of fixed assets & 0.519*** & 0.516*** & 0.672*** & 0.957*** & 0.679*** \\
 & (0.014) & (0.014) & (0.021) & (0.047) & (0.021) \\
Profitability & -0.004*** & -0.004*** & -0.004*** & -0.004*** & -0.004*** \\
 & (0.000) & (0.000) & (0.000) & (0.001) & (0.000) \\
Inflation & -0.077* & -0.161*** & -0.114** & 0.656*** &  \\
 & (0.041) & (0.042) & (0.048) & (0.225) &  \\
Political Risk & 0.091*** & 0.073*** & 0.076*** & 0.033 &  \\
 & (0.015) & (0.014) & (0.017) & (0.079) &  \\
Private credit to GDP & 0.044*** & 0.032*** & 0.028*** & 0.078*** &  \\
 & (0.008) & (0.008) & (0.009) & (0.028) &  \\
 &  &  &  &  &  \\
Observations & 2,964,227 & 2,964,227 & 2,914,664 & 1,003,592 & 2,914,752 \\
R-squared & 0.75 & 0.75 & 0.55 & 0.31 & 0.55 \\
Number of firms & 559,560 & 559,560 & 564,068 & 213,623 & 564,085 \\
Firm fixed effects & Yes & Yes & No & No & No \\
Year fixed effects & Yes & Yes & Yes & No & No \\
Multinational fixed effects & No & No & Yes & No & Yes \\
Multinational*year fixed effects & No & No & No & Yes & No \\
Country fixed effects & No & No & Yes & No & No \\
Country*year fixed effects & No & No & No & No & Yes \\
 Industry fixed effects & No & No & Yes & Yes & Yes \\ \hline
\multicolumn{6}{c}{ Robust standard errors in parentheses} \\
\multicolumn{6}{c}{ *** p$<$0.01, ** p$<$0.05, * p$<$0.1} \\
\end{tabular}
}
	\label{tab:reg1}
\end{table}
The estimated coefficients for the domestic and international effect of concentration limits and capital requirements are negative and positive, respectively. The domestic effect of LTV ratio limits is associated negatively to financial leverage while the incentive to shift debt is not statistically significant. Among the control variables, tangibility and profitability have consistently significant negative coefficients at 1\%. The substitution effect seems to dominate on average in our sample for both variables. Next, as expected, log of fixed assets and private credit to GDP have positive and statistically significant at 1\% estimated coefficients. Inflation enters the regression with positive sign. Finally, although tax rate obtain a positive and statistically significant at 1\% estimated parameters, the international spillover due to tax rate appears with the opposite sign from what we would expect. However, this result should be taken with a grain of salt as in all but two (models (4) and (6) of table \ref{tab:reg1} ) specifications the coefficients are statistically insignificant at 10\%.
\begin{table}	
	\caption{}
	\scalebox{0.8}{\begin{tabular}{lcccccccccc}
\multicolumn{11}{c}{Macroprudential policy effect on firm's financial leverage (capital related indexes)} \\ \hline
 & (1) & (2) & (3) & (4) & (5) & (6) & (7) & (8) & (9) & (10) \\
VARIABLES & Financial leverage & Financial leverage & Financial leverage & Financial leverage & Financial leverage & Financial leverage & Financial leverage & Financial leverage & Financial leverage & Financial leverage \\ \hline
 &  &  &  &  &  &  &  &  &  &  \\
Tangibility & -0.0395*** & -0.0402*** & -0.0401*** & -0.0401*** & -0.0401*** & -0.0602*** & -0.0605*** & -0.0605*** & -0.0605*** & -0.0604*** \\
 & (0.0110) & (0.0110) & (0.0111) & (0.0110) & (0.0111) & (0.00111) & (0.00111) & (0.00111) & (0.00111) & (0.00111) \\
Log of fixed assets & 0.0231*** & 0.0231*** & 0.0231*** & 0.0231*** & 0.0231*** & 0.0232*** & 0.0232*** & 0.0232*** & 0.0232*** & 0.0232*** \\
 & (0.00162) & (0.00162) & (0.00161) & (0.00161) & (0.00162) & (0.000162) & (0.000162) & (0.000162) & (0.000162) & (0.000162) \\
Profitability & -0.153*** & -0.153*** & -0.153*** & -0.153*** & -0.153*** & -0.142*** & -0.142*** & -0.142*** & -0.142*** & -0.142*** \\
 & (0.00730) & (0.00731) & (0.00731) & (0.00731) & (0.00731) & (0.000623) & (0.000624) & (0.000624) & (0.000624) & (0.000624) \\
Inflation & 0.355*** & 0.281*** & 0.257*** & 0.327*** & 0.297*** & 0.270*** & 0.225*** & 0.205*** & 0.246*** & 0.236*** \\
 & (0.0656) & (0.0651) & (0.0652) & (0.0657) & (0.0654) & (0.00608) & (0.00594) & (0.00588) & (0.00607) & (0.00596) \\
Political Risk & -0.000295 & -0.000275 & -0.000350 & -0.000391 & -0.000271 & -0.000179*** & -0.000177*** & -0.000218*** & -0.000239*** & -0.000187*** \\
 & (0.000328) & (0.000328) & (0.000327) & (0.000327) & (0.000329) & (3.53e-05) & (3.54e-05) & (3.52e-05) & (3.52e-05) & (3.56e-05) \\
Private credit to GDP & 0.0109 & 0.0144 & 0.0133 & 0.0121 & 0.0150 & 0.0254*** & 0.0276*** & 0.0267*** & 0.0266*** & 0.0279*** \\
 & (0.00948) & (0.00946) & (0.00947) & (0.00949) & (0.00947) & (0.00111) & (0.00111) & (0.00111) & (0.00111) & (0.00111) \\
Tax rate & 0.0254 & 0.0285 & 0.0165 & 0.0215 & 0.0288 & 0.0399*** & 0.0394*** & 0.0331*** & 0.0360*** & 0.0384*** \\
 & (0.0176) & (0.0176) & (0.0175) & (0.0176) & (0.0177) & (0.00163) & (0.00161) & (0.00161) & (0.00161) & (0.00160) \\
Tax rate spillover & 0.104*** & 0.0998*** & 0.110*** & 0.0989*** & 0.102*** & -0.00438 & -0.00620 & -0.00171 & -0.00557 & -0.00490 \\
 & (0.0351) & (0.0354) & (0.0352) & (0.0356) & (0.0351) & (0.00384) & (0.00384) & (0.00383) & (0.00383) & (0.00384) \\
Capital requirement & -0.0130*** &  &  &  &  & -0.0101*** &  &  &  &  \\
 & (0.00333) &  &  &  &  & (0.000291) &  &  &  &  \\
Capital requirement spillover & 0.00349 &  &  &  &  & 0.00614*** &  &  &  &  \\
 & (0.0112) &  &  &  &  & (0.00105) &  &  &  &  \\
Capital buffer - overall &  & 0.00928*** &  &  &  &  & 0.00478*** &  &  &  \\
 &  & (0.00273) &  &  &  &  & (0.000285) &  &  &  \\
Capital buffer - overall spillover &  & -0.00909 &  &  &  &  & -0.00452*** &  &  &  \\
 &  & (0.00580) &  &  &  &  & (0.000586) &  &  &  \\
Capital buffer - consumers &  &  & 0.0380*** &  &  &  &  & 0.0266*** &  &  \\
 &  &  & (0.00910) &  &  &  &  & (0.000983) &  &  \\
Capital buffer - consumers spillover &  &  & -0.0512** &  &  &  &  & -0.0291*** &  &  \\
 &  &  & (0.0244) &  &  &  &  & (0.00234) &  &  \\
Capital buffer - others &  &  &  & 0.0294*** &  &  &  &  & 0.0133*** &  \\
 &  &  &  & (0.0103) &  &  &  &  & (0.00105) &  \\
Capital buffer - others spillover &  &  &  & -0.0313 &  &  &  &  & -0.0149*** &  \\
 &  &  &  & (0.0191) &  &  &  &  & (0.00199) &  \\
Capital buffer - real estate &  &  &  &  & 0.00825** &  &  &  &  & 0.00295*** \\
 &  &  &  &  & (0.00399) &  &  &  &  & (0.000398) \\
Capital buffer - real estate spillover &  &  &  &  & -0.00540 &  &  &  &  & -0.00203** \\
 &  &  &  &  & (0.00819) &  &  &  &  & (0.000847) \\
 &  &  &  &  &  &  &  &  &  &  \\
Observations & 440,184 & 440,184 & 440,184 & 440,184 & 440,184 & 8,817,102 & 8,817,102 & 8,817,102 & 8,817,102 & 8,817,102 \\
R-squared & 0.074 & 0.074 & 0.075 & 0.074 & 0.074 & 0.069 & 0.068 & 0.069 & 0.068 & 0.068 \\
Number of firm\_id & 397,849 & 397,849 & 397,849 & 397,849 & 397,849 & 2,406,876 & 2,406,876 & 2,406,876 & 2,406,876 & 2,406,876 \\
 Subsidiary and year fixed effects & YES & YES & YES & YES & YES & YES & YES & YES & YES & YES \\ \hline
\multicolumn{11}{c}{ Robust standard errors in parentheses} \\
\multicolumn{11}{c}{ *** p$<$0.01, ** p$<$0.05, * p$<$0.1} \\
\end{tabular}
}
	\label{tab:reg2}
\end{table}
	\subsection{Spillover from parent and other subsidiaries} \label{subsec:split_effect}
	
\begin{table}	
	\caption{}
	\scalebox{0.7}{\begin{tabular}{lcccccc}
\multicolumn{7}{c}{Macroprudential policy effect on firm's financial leverage: spillover from parent and others subsidiaries} \\ \hline
 & (1) & (2) & (3) & (4) & (5) & (6) \\
VARIABLES &  &  &  &  &  &  \\ \hline
 &  &  &  &  &  &  \\
Tangibility & -0.059*** & -0.060*** & -0.057*** & -0.058*** & -0.057*** & -0.057*** \\
 & (0.001) & (0.001) & (0.001) & (0.001) & (0.001) & (0.001) \\
Log of fixed assets & 0.023*** & 0.023*** & 0.023*** & 0.024*** & 0.023*** & 0.023*** \\
 & (0.000) & (0.000) & (0.000) & (0.000) & (0.000) & (0.000) \\
Profitability & -0.143*** & -0.143*** & -0.142*** & -0.142*** & -0.142*** & -0.143*** \\
 & (0.001) & (0.001) & (0.001) & (0.001) & (0.001) & (0.001) \\
Inflation & 0.218*** & 0.208*** & 0.213*** & 0.240*** & 0.240*** & 0.202*** \\
 & (0.006) & (0.006) & (0.006) & (0.006) & (0.006) & (0.006) \\
Political Risk & -0.000*** & 0.000* & 0.001*** & -0.000*** & -0.000*** & 0.000*** \\
 & (0.000) & (0.000) & (0.000) & (0.000) & (0.000) & (0.000) \\
Private credit to GDP & 0.026*** & 0.031*** & 0.049*** & 0.028*** & 0.029*** & 0.046*** \\
 & (0.001) & (0.001) & (0.001) & (0.001) & (0.001) & (0.001) \\
Tax rate & 0.042*** & 0.035*** & 0.050*** & 0.038*** & 0.037*** & 0.052*** \\
 & (0.002) & (0.002) & (0.002) & (0.002) & (0.002) & (0.002) \\
Tax rate spillover to parent & -0.015 & -0.017* & -0.029 & -0.014 & -0.035* & 0.337 \\
 & (0.009) & (0.009) & (0.027) & (0.015) & (0.021) & (1.096) \\
Tax rate spillover to other subsidiaries & -0.010 & -0.009 & -0.038 & -0.047*** & 0.010 & -0.283* \\
 & (0.009) & (0.009) & (0.024) & (0.014) & (0.017) & (0.151) \\
Reserve req. on local currency & 0.004*** &  &  &  &  & 0.000 \\
 & (0.000) &  &  &  &  & (0.000) \\
Reserve req. on local currency spillover to parent & 0.002** &  &  &  &  & -0.025 \\
 & (0.001) &  &  &  &  & (0.033) \\
Reserve req. on local currency spillover to other subsidiaries & -0.006*** &  &  &  &  & -0.022* \\
 & (0.001) &  &  &  &  & (0.012) \\
Reserve req. on foreign currency &  & 0.005*** &  &  &  & 0.005*** \\
 &  & (0.000) &  &  &  & (0.000) \\
Reserve req. on foreign currency spillover to parent &  & -0.003** &  &  &  & 0.025 \\
 &  & (0.001) &  &  &  & (0.025) \\
Reserve req. on foreign currency spillover to other subsidiaries &  & -0.004** &  &  &  & -0.020 \\
 &  & (0.002) &  &  &  & (0.019) \\
LTV ratio limits &  &  & -0.010*** &  &  & -0.006*** \\
 &  &  & (0.000) &  &  & (0.000) \\
LTV ratio limits spillover to parent &  &  & 0.002 &  &  & 0.031 \\
 &  &  & (0.003) &  &  & (0.065) \\
LTV ratio limits spillover to other subsidiaries &  &  & 0.002 &  &  & -0.048** \\
 &  &  & (0.003) &  &  & (0.024) \\
Concentration limit &  &  &  & -0.000 &  & -0.003*** \\
 &  &  &  & (0.000) &  & (0.000) \\
Concentration limit spillover to parent &  &  &  & 0.001 &  & -0.271 \\
 &  &  &  & (0.001) &  & (0.179) \\
Concentration limit spillover to other subsidiaries &  &  &  & 0.004*** &  & 0.047* \\
 &  &  &  & (0.001) &  & (0.026) \\
Interbank exposure limit &  &  &  &  & 0.001** & 0.004*** \\
 &  &  &  &  & (0.000) & (0.000) \\
Interbank exposure limit spillover to parent &  &  &  &  & -0.001 & -0.234* \\
 &  &  &  &  & (0.002) & (0.124) \\
Interbank exposure limit spillover to other subsidiaries &  &  &  &  & -0.003 & -0.028 \\
 &  &  &  &  & (0.002) & (0.050) \\
 &  &  &  &  &  &  \\
Observations & 8,275,892 & 8,275,892 & 7,925,218 & 8,092,393 & 8,028,026 & 7,826,737 \\
R-squared & 0.071 & 0.071 & 0.072 & 0.071 & 0.071 & 0.072 \\
Number of firm\_id & 2,327,059 & 2,327,059 & 2,254,908 & 2,287,160 & 2,274,033 & 2,232,009 \\
 Subsidiary and year fixed effects & YES & YES & YES & YES & YES & YES \\ \hline
\multicolumn{7}{c}{ Robust standard errors in parentheses} \\
\multicolumn{7}{c}{ *** p$<$0.01, ** p$<$0.05, * p$<$0.1} \\
\end{tabular}
}
	\label{tab:reg3}
\end{table}
\begin{table}	
	\caption{}
	\scalebox{0.7}{\begin{tabular}{lcccccc}
\multicolumn{7}{c}{Macroprudential policy effect on firm's financial leverage (capital related indexes): spillover from parent and others subsidiaries} \\ \hline
 & (1) & (2) & (3) & (4) & (5) & (6) \\
VARIABLES &  &  &  &  &  &  \\ \hline
 &  &  &  &  &  &  \\
Tangibility & -0.058*** & -0.058*** & -0.058*** & -0.058*** & -0.058*** & -0.058*** \\
 & (0.001) & (0.001) & (0.001) & (0.001) & (0.001) & (0.001) \\
Log of fixed assets & 0.024*** & 0.023*** & 0.023*** & 0.024*** & 0.024*** & 0.023*** \\
 & (0.000) & (0.000) & (0.000) & (0.000) & (0.000) & (0.000) \\
Profitability & -0.142*** & -0.142*** & -0.142*** & -0.142*** & -0.142*** & -0.142*** \\
 & (0.001) & (0.001) & (0.001) & (0.001) & (0.001) & (0.001) \\
Inflation & 0.269*** & 0.220*** & 0.198*** & 0.243*** & 0.232*** & 0.226*** \\
 & (0.006) & (0.006) & (0.006) & (0.006) & (0.006) & (0.006) \\
Political Risk & -0.000*** & -0.000*** & -0.000*** & -0.000*** & -0.000*** & -0.000*** \\
 & (0.000) & (0.000) & (0.000) & (0.000) & (0.000) & (0.000) \\
Private credit to GDP & 0.026*** & 0.029*** & 0.028*** & 0.027*** & 0.029*** & 0.025*** \\
 & (0.001) & (0.001) & (0.001) & (0.001) & (0.001) & (0.001) \\
Tax rate & 0.040*** & 0.040*** & 0.033*** & 0.036*** & 0.038*** & 0.033*** \\
 & (0.002) & (0.002) & (0.002) & (0.002) & (0.002) & (0.002) \\
Tax rate spillover to parent & -0.015 & -0.016* & -0.012 & -0.015 & -0.015* & -0.014 \\
 & (0.009) & (0.009) & (0.009) & (0.009) & (0.009) & (0.009) \\
Tax rate spillover to other subsidiaries & -0.010 & -0.011 & -0.004 & -0.011 & -0.009 & -0.002 \\
 & (0.009) & (0.009) & (0.009) & (0.009) & (0.009) & (0.009) \\
Capital requirement & -0.010*** &  &  &  &  & -0.009*** \\
 & (0.000) &  &  &  &  & (0.000) \\
Capital requirement spillover to parent & -0.000 &  &  &  &  & -0.001 \\
 & (0.002) &  &  &  &  & (0.002) \\
Capital requirement spillover to other subsidiaries & 0.013*** &  &  &  &  & 0.011*** \\
 & (0.003) &  &  &  &  & (0.003) \\
Capital buffer - overall &  & 0.005*** &  &  &  &  \\
 &  & (0.000) &  &  &  &  \\
Capital buffer - overall spillover to parent &  & -0.003** &  &  &  &  \\
 &  & (0.001) &  &  &  &  \\
Capital buffer - overall spillover to other subsidiaries &  & -0.005*** &  &  &  &  \\
 &  & (0.001) &  &  &  &  \\
Capital buffer - consumers &  &  & 0.028*** &  &  & 0.031*** \\
 &  &  & (0.001) &  &  & (0.001) \\
Capital buffer - consumers spillover to parent &  &  & -0.016*** &  &  & -0.008 \\
 &  &  & (0.006) &  &  & (0.008) \\
Capital buffer - consumers spillover to other subsidiaries &  &  & -0.039*** &  &  & -0.041*** \\
 &  &  & (0.006) &  &  & (0.008) \\
Capital buffer - others &  &  &  & 0.015*** &  & -0.004*** \\
 &  &  &  & (0.001) &  & (0.001) \\
Capital buffer - others spillover to parent &  &  &  & -0.013*** &  & -0.010 \\
 &  &  &  & (0.004) &  & (0.006) \\
Capital buffer - others spillover to other subsidiaries &  &  &  & -0.017*** &  & 0.004 \\
 &  &  &  & (0.004) &  & (0.006) \\
Capital buffer - real estate &  &  &  &  & 0.003*** & -0.004*** \\
 &  &  &  &  & (0.000) & (0.000) \\
Capital buffer - real estate spillover to parent &  &  &  &  & -0.002 & 0.001 \\
 &  &  &  &  & (0.002) & (0.002) \\
Capital buffer - real estate spillover to other subsidiaries &  &  &  &  & -0.002 & 0.004* \\
 &  &  &  &  & (0.002) & (0.002) \\
 &  &  &  &  &  &  \\
Observations & 8,275,857 & 8,275,892 & 8,275,892 & 8,275,892 & 8,275,892 & 8,275,857 \\
R-squared & 0.071 & 0.071 & 0.071 & 0.070 & 0.070 & 0.071 \\
Number of firm\_id & 2,327,038 & 2,327,059 & 2,327,059 & 2,327,059 & 2,327,059 & 2,327,038 \\
 Subsidiary and year fixed effects & YES & YES & YES & YES & YES & YES \\ \hline
\multicolumn{7}{c}{ Robust standard errors in parentheses} \\
\multicolumn{7}{c}{ *** p$<$0.01, ** p$<$0.05, * p$<$0.1} \\
\end{tabular}
}
	\label{tab:reg4}
\end{table}
	\subsection{Effects on long- and short-term debt} \label{subsec:long_short}
	When considering long-term debt to total asset as dependent variable, the biggest changes come from the control variables, as shown in tables \ref{tab:reg5} and \ref{tab:reg6}. Tangibility appears with a positive coefficient, suggesting that the effect of fixed assets as collateral dominates the substitution effect due to depreciation. Inflation is negatively associated to long-term debt. As \cite{huizinga2008capital} points out, in an inflationary environments the uncertainty about ex-post real interest rate may inhibit leverage on long-term liabilities. In all specifications, tax rate coefficients are significantly at 1\% and has a negative sign. Surprisingly, the tax incentive to shift debt is statistically significantly at 5\% and positive for nine out of 12 specifications.
	
	 
\begin{table}	
	\caption{}
	\scalebox{0.8}{\begin{tabular}{lcccccc}
\multicolumn{7}{c}{Macroprudential policy effect on firm's long-term debt} \\ \hline
 & (1) & (2) & (3) & (4) & (5) & (6) \\
VARIABLES &  &  &  &  &  &  \\ \hline
 &  &  &  &  &  &  \\
Tangibility & 0.117*** & 0.117*** & 0.117*** & 0.117*** & 0.117*** & 0.116*** \\
 & (0.001) & (0.001) & (0.001) & (0.001) & (0.001) & (0.001) \\
Log of fixed assets & 0.006*** & 0.006*** & 0.006*** & 0.006*** & 0.006*** & 0.006*** \\
 & (0.000) & (0.000) & (0.000) & (0.000) & (0.000) & (0.000) \\
Profitability & -0.026*** & -0.026*** & -0.026*** & -0.026*** & -0.026*** & -0.026*** \\
 & (0.000) & (0.000) & (0.000) & (0.000) & (0.000) & (0.000) \\
Inflation & -0.118*** & -0.105*** & -0.108*** & -0.114*** & -0.110*** & -0.116*** \\
 & (0.005) & (0.005) & (0.005) & (0.005) & (0.005) & (0.005) \\
Political Risk & -0.001*** & -0.001*** & 0.000*** & -0.000*** & -0.000*** & 0.001*** \\
 & (0.000) & (0.000) & (0.000) & (0.000) & (0.000) & (0.000) \\
Private credit to GDP & 0.031*** & 0.031*** & 0.049*** & 0.027*** & 0.032*** & 0.048*** \\
 & (0.001) & (0.001) & (0.001) & (0.001) & (0.001) & (0.001) \\
Tax rate & -0.025*** & -0.026*** & -0.011*** & -0.034*** & -0.026*** & -0.018*** \\
 & (0.002) & (0.002) & (0.002) & (0.002) & (0.002) & (0.002) \\
Tax rate spillover & 0.008*** & 0.009*** & 0.008*** & 0.003 & 0.005 & 0.001 \\
 & (0.003) & (0.003) & (0.003) & (0.003) & (0.003) & (0.003) \\
Reserve req. on local currency & 0.001*** &  &  &  &  & 0.002*** \\
 & (0.000) &  &  &  &  & (0.000) \\
Reserve req. on local currency spillover & -0.001*** &  &  &  &  & -0.001*** \\
 & (0.000) &  &  &  &  & (0.000) \\
Reserve req. on foreign currency &  & 0.004*** &  &  &  & 0.004*** \\
 &  & (0.000) &  &  &  & (0.000) \\
Reserve req. on foreign currency spillover &  & 0.000 &  &  &  & 0.000 \\
 &  & (0.001) &  &  &  & (0.001) \\
LTV ratio limits &  &  & -0.009*** &  &  & -0.011*** \\
 &  &  & (0.000) &  &  & (0.000) \\
LTV ratio limits spillover &  &  & -0.001 &  &  & -0.001 \\
 &  &  & (0.001) &  &  & (0.001) \\
Concentration limit &  &  &  & 0.003*** &  & 0.004*** \\
 &  &  &  & (0.000) &  & (0.000) \\
Concentration limit spillover &  &  &  & 0.003*** &  & 0.002*** \\
 &  &  &  & (0.000) &  & (0.000) \\
Interbank exposure limit &  &  &  &  & -0.002*** & -0.004*** \\
 &  &  &  &  & (0.000) & (0.000) \\
Interbank exposure limit spillover &  &  &  &  & -0.005*** & -0.003*** \\
 &  &  &  &  & (0.001) & (0.001) \\
 &  &  &  &  &  &  \\
Observations & 7,735,591 & 7,735,591 & 7,735,591 & 7,735,591 & 7,735,591 & 7,735,591 \\
R-squared & 0.029 & 0.029 & 0.030 & 0.029 & 0.029 & 0.030 \\
Number of firm\_id & 2,185,019 & 2,185,019 & 2,185,019 & 2,185,019 & 2,185,019 & 2,185,019 \\
 Subsidiary and year fixed effects & YES & YES & YES & YES & YES & YES \\ \hline
\multicolumn{7}{c}{ Robust standard errors in parentheses} \\
\multicolumn{7}{c}{ *** p$<$0.01, ** p$<$0.05, * p$<$0.1} \\
\end{tabular}
}
	\label{tab:reg5}
\end{table}
The results for the macro-prudential variables are almost unchanged. The exceptions are interbank exposure and concentration limits that have reversed their coefficients signs. 
\begin{table}	
	\caption{}
	\scalebox{0.8}{\begin{tabular}{lcccccc}
\multicolumn{7}{c}{Macroprudential policy effect on firm's long-term debt (capital related indexes)} \\ \hline
 & (1) & (2) & (3) & (4) & (5) & (6) \\
VARIABLES &  &  &  &  &  &  \\ \hline
 &  &  &  &  &  &  \\
Tangibility & 0.117*** & 0.117*** & 0.117*** & 0.117*** & 0.117*** & 0.117*** \\
 & (0.001) & (0.001) & (0.001) & (0.001) & (0.001) & (0.001) \\
Log of fixed assets & 0.006*** & 0.006*** & 0.006*** & 0.006*** & 0.006*** & 0.006*** \\
 & (0.000) & (0.000) & (0.000) & (0.000) & (0.000) & (0.000) \\
Profitability & -0.026*** & -0.026*** & -0.026*** & -0.026*** & -0.026*** & -0.026*** \\
 & (0.000) & (0.000) & (0.000) & (0.000) & (0.000) & (0.000) \\
Inflation & -0.109*** & -0.119*** & -0.126*** & -0.110*** & -0.114*** & -0.125*** \\
 & (0.005) & (0.005) & (0.005) & (0.005) & (0.005) & (0.005) \\
Political Risk & -0.001*** & -0.001*** & -0.001*** & -0.001*** & -0.001*** & -0.001*** \\
 & (0.000) & (0.000) & (0.000) & (0.000) & (0.000) & (0.000) \\
Private credit to GDP & 0.031*** & 0.031*** & 0.031*** & 0.031*** & 0.031*** & 0.031*** \\
 & (0.001) & (0.001) & (0.001) & (0.001) & (0.001) & (0.001) \\
Tax rate & -0.026*** & -0.025*** & -0.027*** & -0.026*** & -0.025*** & -0.029*** \\
 & (0.002) & (0.002) & (0.002) & (0.002) & (0.002) & (0.002) \\
Tax rate spillover & 0.009*** & 0.008** & 0.009*** & 0.009*** & 0.007** & 0.009*** \\
 & (0.003) & (0.003) & (0.003) & (0.003) & (0.003) & (0.003) \\
Capital requirement & -0.001** &  &  &  &  & 0.000 \\
 & (0.000) &  &  &  &  & (0.000) \\
Capital requirement spillover & -0.000 &  &  &  &  & -0.000 \\
 & (0.001) &  &  &  &  & (0.001) \\
Capital buffer - overall &  & 0.002*** &  &  &  &  \\
 &  & (0.000) &  &  &  &  \\
Capital buffer - overall spillover &  & -0.002*** &  &  &  &  \\
 &  & (0.000) &  &  &  &  \\
Capital buffer - consumers &  &  & 0.008*** &  &  & 0.009*** \\
 &  &  & (0.001) &  &  & (0.001) \\
Capital buffer - consumers spillover &  &  & -0.007*** &  &  & -0.007*** \\
 &  &  & (0.001) &  &  & (0.002) \\
Capital buffer - others &  &  &  & 0.004*** &  & 0.000 \\
 &  &  &  & (0.001) &  & (0.001) \\
Capital buffer - others spillover &  &  &  & -0.001 &  & 0.004** \\
 &  &  &  & (0.001) &  & (0.002) \\
Capital buffer - real estate &  &  &  &  & 0.001*** & -0.002*** \\
 &  &  &  &  & (0.000) & (0.000) \\
Capital buffer - real estate spillover &  &  &  &  & -0.003*** & -0.003*** \\
 &  &  &  &  & (0.001) & (0.001) \\
 &  &  &  &  &  &  \\
Observations & 7,735,591 & 7,735,591 & 7,735,591 & 7,735,591 & 7,735,591 & 7,735,591 \\
R-squared & 0.029 & 0.029 & 0.029 & 0.029 & 0.029 & 0.029 \\
Number of firm\_id & 2,185,019 & 2,185,019 & 2,185,019 & 2,185,019 & 2,185,019 & 2,185,019 \\
 Subsidiary and year fixed effects & YES & YES & YES & YES & YES & YES \\ \hline
\multicolumn{7}{c}{ Robust standard errors in parentheses} \\
\multicolumn{7}{c}{ *** p$<$0.01, ** p$<$0.05, * p$<$0.1} \\
\end{tabular}
}
	\label{tab:reg6}
\end{table}
Tables \ref{tab:reg7} and \ref{tab:reg8} reports the results for our benchmark model considering short-term debt to total assets as regressand. Tangibility regained the negative sign, suggesting that the substitution effect coming from the depreciation tax shield matters more when deciding to finance through short-term loans. Variables related to tax rate changed considerably. The international effect appears significant at 1\% at all regressions, while the evidence for the presence of a domestic effect is less robust.
\begin{table}
\caption{}
\scalebox{0.8}{\begin{tabular}{lcccccc}
\multicolumn{7}{c}{Macroprudential policy effect on firm's short-term debt} \\ \hline
 & (1) & (2) & (3) & (4) & (5) & (6) \\
VARIABLES &  &  &  &  &  &  \\ \hline
 &  &  &  &  &  &  \\
Tangibility & -0.005*** & -0.005*** & -0.005*** & -0.005*** & -0.004*** & -0.005*** \\
 & (0.001) & (0.001) & (0.001) & (0.001) & (0.001) & (0.001) \\
Log of fixed assets & 0.004*** & 0.004*** & 0.004*** & 0.004*** & 0.004*** & 0.004*** \\
 & (0.000) & (0.000) & (0.000) & (0.000) & (0.000) & (0.000) \\
Profitability & -0.032*** & -0.032*** & -0.032*** & -0.032*** & -0.032*** & -0.032*** \\
 & (0.000) & (0.000) & (0.000) & (0.000) & (0.000) & (0.000) \\
Inflation & -0.056*** & -0.057*** & -0.050*** & -0.051*** & -0.054*** & -0.052*** \\
 & (0.004) & (0.004) & (0.004) & (0.004) & (0.004) & (0.004) \\
Political Risk & 0.000*** & 0.000*** & 0.001*** & 0.000*** & 0.000*** & 0.000*** \\
 & (0.000) & (0.000) & (0.000) & (0.000) & (0.000) & (0.000) \\
Private credit to GDP & 0.015*** & 0.016*** & 0.028*** & 0.020*** & 0.013*** & 0.028*** \\
 & (0.001) & (0.001) & (0.001) & (0.001) & (0.001) & (0.001) \\
Tax rate & 0.001 & 0.001 & 0.010*** & 0.010*** & 0.001 & 0.017*** \\
 & (0.001) & (0.001) & (0.001) & (0.001) & (0.001) & (0.001) \\
Tax rate spillover & -0.007*** & -0.008*** & -0.008*** & -0.010*** & -0.008*** & -0.010*** \\
 & (0.002) & (0.002) & (0.002) & (0.003) & (0.002) & (0.003) \\
Reserve req. on local currency & 0.000* &  &  &  &  & 0.001*** \\
 & (0.000) &  &  &  &  & (0.000) \\
Reserve req. on local currency spillover & 0.001** &  &  &  &  & 0.001*** \\
 & (0.000) &  &  &  &  & (0.000) \\
Reserve req. on foreign currency &  & -0.001*** &  &  &  & -0.001*** \\
 &  & (0.000) &  &  &  & (0.000) \\
Reserve req. on foreign currency spillover &  & -0.002*** &  &  &  & -0.002*** \\
 &  & (0.001) &  &  &  & (0.001) \\
LTV ratio limits &  &  & -0.006*** &  &  & -0.005*** \\
 &  &  & (0.000) &  &  & (0.000) \\
LTV ratio limits spillover &  &  & 0.001** &  &  & 0.001* \\
 &  &  & (0.000) &  &  & (0.000) \\
Concentration limit &  &  &  & -0.003*** &  & -0.003*** \\
 &  &  &  & (0.000) &  & (0.000) \\
Concentration limit spillover &  &  &  & 0.001*** &  & 0.001*** \\
 &  &  &  & (0.000) &  & (0.000) \\
Interbank exposure limit &  &  &  &  & 0.004*** & 0.003*** \\
 &  &  &  &  & (0.000) & (0.000) \\
Interbank exposure limit spillover &  &  &  &  & -0.001 & -0.001 \\
 &  &  &  &  & (0.001) & (0.001) \\
 &  &  &  &  &  &  \\
Observations & 8,576,627 & 8,576,627 & 8,576,627 & 8,576,627 & 8,576,627 & 8,576,627 \\
R-squared & 0.007 & 0.007 & 0.008 & 0.007 & 0.007 & 0.008 \\
Number of firm\_id & 2,286,000 & 2,286,000 & 2,286,000 & 2,286,000 & 2,286,000 & 2,286,000 \\
 Subsidiary and year fixed effects & YES & YES & YES & YES & YES & YES \\ \hline
\multicolumn{7}{c}{ Robust standard errors in parentheses} \\
\multicolumn{7}{c}{ *** p$<$0.01, ** p$<$0.05, * p$<$0.1} \\
\end{tabular}
}
\label{tab:reg7}
\end{table}
\begin{table}	
\caption{}
\scalebox{0.8}{\begin{tabular}{lcccccccccc}
\multicolumn{11}{c}{Macroprudential policy effect on firm's short-term debt (capital related indexes)} \\ \hline
 & (1) & (2) & (3) & (4) & (5) & (6) & (7) & (8) & (9) & (10) \\
VARIABLES & Short term debt & Short term debt & Short term debt & Short term debt & Short term debt & Short term debt & Short term debt & Short term debt & Short term debt & Short term debt \\ \hline
 &  &  &  &  &  &  &  &  &  &  \\
Tangibility & 0.00158 & 0.00119 & 0.00141 & 0.00153 & 0.00112 & -0.00451*** & -0.00475*** & -0.00460*** & -0.00454*** & -0.00483*** \\
 & (0.00610) & (0.00610) & (0.00610) & (0.00610) & (0.00610) & (0.000614) & (0.000614) & (0.000613) & (0.000613) & (0.000614) \\
Log of fixed assets & 0.00352*** & 0.00349*** & 0.00351*** & 0.00352*** & 0.00347*** & 0.00403*** & 0.00401*** & 0.00403*** & 0.00403*** & 0.00400*** \\
 & (0.000934) & (0.000934) & (0.000934) & (0.000934) & (0.000934) & (9.05e-05) & (9.05e-05) & (9.05e-05) & (9.05e-05) & (9.05e-05) \\
Profitability & -0.0334*** & -0.0334*** & -0.0334*** & -0.0334*** & -0.0334*** & -0.0316*** & -0.0316*** & -0.0316*** & -0.0316*** & -0.0316*** \\
 & (0.00346) & (0.00346) & (0.00346) & (0.00346) & (0.00346) & (0.000293) & (0.000293) & (0.000293) & (0.000293) & (0.000293) \\
Inflation & 0.0302 & -0.00833 & 0.00491 & 0.0244 & -0.0174 & -0.0467*** & -0.0754*** & -0.0632*** & -0.0573*** & -0.0854*** \\
 & (0.0438) & (0.0425) & (0.0431) & (0.0428) & (0.0425) & (0.00390) & (0.00376) & (0.00378) & (0.00379) & (0.00375) \\
Political Risk & 0.000626*** & 0.000683*** & 0.000620*** & 0.000622*** & 0.000769*** & 0.000253*** & 0.000287*** & 0.000242*** & 0.000259*** & 0.000363*** \\
 & (0.000223) & (0.000221) & (0.000221) & (0.000221) & (0.000220) & (2.38e-05) & (2.36e-05) & (2.35e-05) & (2.35e-05) & (2.35e-05) \\
Private credit to GDP & 0.0101* & 0.0109* & 0.0103* & 0.0106* & 0.0126** & 0.0151*** & 0.0159*** & 0.0155*** & 0.0164*** & 0.0173*** \\
 & (0.00609) & (0.00601) & (0.00603) & (0.00605) & (0.00598) & (0.000712) & (0.000703) & (0.000703) & (0.000704) & (0.000702) \\
Tax rate & -0.000817 & 0.00495 & -0.00241 & -0.00118 & 0.0120 & 0.00168* & 0.00459*** & 0.000353 & 0.000861 & 0.0100*** \\
 & (0.0100) & (0.0100) & (0.00996) & (0.00997) & (0.0101) & (0.000945) & (0.000934) & (0.000932) & (0.000933) & (0.000931) \\
Tax rate spillover & 0.0441** & 0.0432** & 0.0447** & 0.0431** & 0.0441** & -0.00785*** & -0.00915*** & -0.00728*** & -0.00828*** & -0.00940*** \\
 & (0.0218) & (0.0218) & (0.0219) & (0.0214) & (0.0221) & (0.00242) & (0.00243) & (0.00241) & (0.00242) & (0.00244) \\
Capital requirement & -0.000899 &  &  &  &  & -0.00198*** &  &  &  &  \\
 & (0.00200) &  &  &  &  & (0.000183) &  &  &  &  \\
Capital requirement spillover & -0.00647 &  &  &  &  & 0.00162*** &  &  &  &  \\
 & (0.00619) &  &  &  &  & (0.000608) &  &  &  &  \\
Capital buffer - overall &  & 0.00724*** &  &  &  &  & 0.00484*** &  &  &  \\
 &  & (0.00150) &  &  &  &  & (0.000157) &  &  &  \\
Capital buffer - overall spillover &  & -0.00302 &  &  &  &  & -0.00338*** &  &  &  \\
 &  & (0.00285) &  &  &  &  & (0.000361) &  &  &  \\
Capital buffer - consumers &  &  & 0.0106** &  &  &  &  & 0.00511*** &  &  \\
 &  &  & (0.00503) &  &  &  &  & (0.000509) &  &  \\
Capital buffer - consumers spillover &  &  & -0.00903 &  &  &  &  & -0.00488*** &  &  \\
 &  &  & (0.00952) &  &  &  &  & (0.00140) &  &  \\
Capital buffer - others &  &  &  & -0.00113 &  &  &  &  & -0.00708*** &  \\
 &  &  &  & (0.00699) &  &  &  &  & (0.000736) &  \\
Capital buffer - others spillover &  &  &  & -0.00188 &  &  &  &  & -0.00198 &  \\
 &  &  &  & (0.0119) &  &  &  &  & (0.00128) &  \\
Capital buffer - real estate &  &  &  &  & 0.0138*** &  &  &  &  & 0.0105*** \\
 &  &  &  &  & (0.00217) &  &  &  &  & (0.000222) \\
Capital buffer - real estate spillover &  &  &  &  & -0.00349 &  &  &  &  & -0.00471*** \\
 &  &  &  &  & (0.00416) &  &  &  &  & (0.000510) \\
 &  &  &  &  &  &  &  &  &  &  \\
Observations & 428,236 & 428,236 & 428,236 & 428,236 & 428,236 & 8,576,627 & 8,576,627 & 8,576,627 & 8,576,627 & 8,576,627 \\
R-squared & 0.009 & 0.010 & 0.009 & 0.009 & 0.010 & 0.007 & 0.007 & 0.007 & 0.007 & 0.008 \\
Number of firm\_id & 386,453 & 386,453 & 386,453 & 386,453 & 386,453 & 2,286,000 & 2,286,000 & 2,286,000 & 2,286,000 & 2,286,000 \\
 Subsidiary and year fixed effects & YES & YES & YES & YES & YES & YES & YES & YES & YES & YES \\ \hline
\multicolumn{11}{c}{ Robust standard errors in parentheses} \\
\multicolumn{11}{c}{ *** p$<$0.01, ** p$<$0.05, * p$<$0.1} \\
\end{tabular}
}
\label{tab:reg8}
\end{table}

\subsection{Domestic effect on multinationals} \label{subsec:multyearFE}
\begin{table}	
	\caption{}
	\scalebox{0.8}{\begin{tabular}{lcccccccccc}
\multicolumn{11}{c}{Macroprudential policy effect on firm's leverage: multinational*time fixed effects} \\ \hline
 & (1) & (2) & (3) & (4) & (5) & (6) & (7) & (8) & (9) & (10) \\
VARIABLES & Financial leverage & Financial leverage & Financial leverage & Financial leverage & Financial leverage & Financial leverage & Financial leverage & Financial leverage & Financial leverage & Financial leverage \\ \hline
 &  &  &  &  &  &  &  &  &  &  \\
Tangibility & -0.193*** & -0.193*** & -0.192*** & -0.192*** & -0.192*** & -0.182*** & -0.181*** & -0.181*** & -0.181*** & -0.180*** \\
 & (0.0118) & (0.0118) & (0.0118) & (0.0118) & (0.0118) & (0.00215) & (0.00215) & (0.00215) & (0.00215) & (0.00215) \\
Log of fixed assets & 0.0133*** & 0.0133*** & 0.0133*** & 0.0133*** & 0.0131*** & 0.0116*** & 0.0115*** & 0.0116*** & 0.0115*** & 0.0113*** \\
 & (0.00118) & (0.00118) & (0.00118) & (0.00118) & (0.00118) & (0.000199) & (0.000200) & (0.000199) & (0.000199) & (0.000199) \\
Profitability & -0.142*** & -0.141*** & -0.142*** & -0.141*** & -0.141*** & -0.149*** & -0.149*** & -0.150*** & -0.149*** & -0.149*** \\
 & (0.0124) & (0.0124) & (0.0124) & (0.0124) & (0.0123) & (0.00208) & (0.00208) & (0.00208) & (0.00208) & (0.00208) \\
Inflation & -0.413* & -0.340 & -0.403* & -0.391* & -0.432* & -0.260*** & -0.198*** & -0.242*** & -0.220*** & -0.236*** \\
 & (0.225) & (0.221) & (0.222) & (0.224) & (0.221) & (0.0427) & (0.0424) & (0.0427) & (0.0431) & (0.0425) \\
Political Risk & 0.00273*** & 0.00307*** & 0.00224*** & 0.00249*** & 0.00110** & 0.00226*** & 0.00227*** & 0.00150*** & 0.00183*** & 0.000782*** \\
 & (0.000517) & (0.000522) & (0.000536) & (0.000520) & (0.000537) & (8.28e-05) & (8.39e-05) & (8.81e-05) & (8.40e-05) & (8.56e-05) \\
Private credit to GDP & 0.00620 & 0.00916 & 0.00695 & 0.00315 & 0.00984 & 0.00441*** & 0.00158 & 0.00202* & -0.00243** & 0.00350*** \\
 & (0.00757) & (0.00729) & (0.00747) & (0.00719) & (0.00720) & (0.00125) & (0.00123) & (0.00120) & (0.00120) & (0.00122) \\
Tax rate & 0.174*** & 0.204*** & 0.177*** & 0.181*** & 0.176*** & 0.178*** & 0.196*** & 0.181*** & 0.184*** & 0.175*** \\
 & (0.0239) & (0.0254) & (0.0240) & (0.0275) & (0.0240) & (0.00432) & (0.00458) & (0.00429) & (0.00462) & (0.00435) \\
Reserve req. on local currency & -0.00291 &  &  &  &  & -0.00604*** &  &  &  &  \\
 & (0.00230) &  &  &  &  & (0.000332) &  &  &  &  \\
Reserve req. on foreign currency &  & 0.0130*** &  &  &  &  & 0.00904*** &  &  &  \\
 &  & (0.00350) &  &  &  &  & (0.000514) &  &  &  \\
LTV ratio limits &  &  & 0.00746* &  &  &  &  & 0.00958*** &  &  \\
 &  &  & (0.00415) &  &  &  &  & (0.000633) &  &  \\
Concentration limit &  &  &  & -0.00147 &  &  &  &  & -0.00193*** &  \\
 &  &  &  & (0.00256) &  &  &  &  & (0.000428) &  \\
Interbank exposure limit &  &  &  &  & 0.0230*** &  &  &  &  & 0.0188*** \\
 &  &  &  &  & (0.00285) &  &  &  &  & (0.000470) \\
 &  &  &  &  &  &  &  &  &  &  \\
Observations & 49,341 & 49,341 & 49,341 & 49,341 & 49,341 & 983,212 & 983,212 & 983,212 & 983,212 & 983,212 \\
R-squared & 0.090 & 0.091 & 0.090 & 0.090 & 0.094 & 0.078 & 0.078 & 0.078 & 0.078 & 0.080 \\
Number of debt\_shifting\_group & 34,136 & 34,136 & 34,136 & 34,136 & 34,136 & 217,783 & 217,783 & 217,783 & 217,783 & 217,783 \\
 Multinational-year and industry fixed effects & YES & YES & YES & YES & YES & YES & YES & YES & YES & YES \\ \hline
\multicolumn{11}{c}{ Robust standard errors in parentheses} \\
\multicolumn{11}{c}{ *** p$<$0.01, ** p$<$0.05, * p$<$0.1} \\
\end{tabular}
}
	\label{tab:9}
\end{table}
\begin{table}	
	\caption{}
	\scalebox{0.8}{\begin{tabular}{lcccccc}
\multicolumn{7}{c}{Macroprudential policy effect on firm's leverage (capital related indexes): multinational*time fixed effects} \\ \hline
 & (1) & (2) & (3) & (4) & (5) & (6) \\
VARIABLES &  &  &  &  &  &  \\ \hline
 &  &  &  &  &  &  \\
Tangibility & -0.181*** & -0.181*** & -0.181*** & -0.181*** & -0.180*** & -0.180*** \\
 & (0.002) & (0.002) & (0.002) & (0.002) & (0.002) & (0.002) \\
Log of fixed assets & 0.012*** & 0.012*** & 0.012*** & 0.012*** & 0.012*** & 0.012*** \\
 & (0.000) & (0.000) & (0.000) & (0.000) & (0.000) & (0.000) \\
Profitability & -0.150*** & -0.150*** & -0.149*** & -0.149*** & -0.149*** & -0.149*** \\
 & (0.002) & (0.002) & (0.002) & (0.002) & (0.002) & (0.002) \\
Inflation & -0.186*** & -0.257*** & -0.184*** & -0.195*** & -0.303*** & -0.292*** \\
 & (0.043) & (0.043) & (0.043) & (0.042) & (0.043) & (0.043) \\
Political Risk & 0.002*** & 0.001*** & 0.002*** & 0.002*** & 0.001*** & 0.002*** \\
 & (0.000) & (0.000) & (0.000) & (0.000) & (0.000) & (0.000) \\
Private credit to GDP & -0.002 & -0.009*** & -0.005*** & -0.001 & -0.008*** & -0.004*** \\
 & (0.001) & (0.001) & (0.001) & (0.001) & (0.001) & (0.001) \\
Tax rate & 0.177*** & 0.145*** & 0.173*** & 0.183*** & 0.146*** & 0.163*** \\
 & (0.004) & (0.004) & (0.004) & (0.005) & (0.004) & (0.005) \\
Capital requirement & -0.010*** &  &  &  &  & -0.009*** \\
 & (0.002) &  &  &  &  & (0.002) \\
Capital buffer - overall &  & -0.011*** &  &  &  &  \\
 &  & (0.001) &  &  &  &  \\
Capital buffer - consumers &  &  & -0.029*** &  &  & -0.009*** \\
 &  &  & (0.003) &  &  & (0.003) \\
Capital buffer - others &  &  &  & 0.008*** &  & 0.016*** \\
 &  &  &  & (0.002) &  & (0.002) \\
Capital buffer - real estate &  &  &  &  & -0.018*** & -0.018*** \\
 &  &  &  &  & (0.001) & (0.001) \\
 &  &  &  &  &  &  \\
Observations & 983,212 & 983,212 & 983,212 & 983,212 & 983,212 & 983,212 \\
R-squared & 0.078 & 0.078 & 0.078 & 0.078 & 0.079 & 0.079 \\
Number of debt\_shifting\_group & 217,783 & 217,783 & 217,783 & 217,783 & 217,783 & 217,783 \\
 Multinational-year and industry fixed effects & YES & YES & YES & YES & YES & YES \\ \hline
\multicolumn{7}{c}{ Robust standard errors in parentheses} \\
\multicolumn{7}{c}{ *** p$<$0.01, ** p$<$0.05, * p$<$0.1} \\
\end{tabular}
}
	\label{tab:10}
\end{table}
	\subsection{Partial adjustment}
	\label{subsec:partial}
	\begin{table}	
		\caption{}
		\scalebox{0.8}{\begin{tabular}{lc}
\multicolumn{2}{c}{Macroprudential policy and optimal capital adjustment} \\ \hline
 & (1) \\
VARIABLES & Financial leverage \\ \hline
 &  \\
Tangibility & -0.176*** \\
 & (0.00523) \\
Log of fixed assets & 0.0265*** \\
 & (0.000726) \\
Profitability & -0.249*** \\
 & (0.00372) \\
Inflation & -0.0116 \\
 & (0.0246) \\
Political Risk & -0.000262** \\
 & (0.000107) \\
Private credit to GDP & 0.00226 \\
 & (0.00353) \\
Tax rate & 0.0368*** \\
 & (0.00632) \\
Tax rate spillover & -0.0210*** \\
 & (0.00562) \\
Reserve req. on local currency & 8.19e-05 \\
 & (0.000770) \\
Reserve req. on local currency spillover & 0.000285 \\
 & (0.000488) \\
 &  \\
Observations & 493,687 \\
Number of firm\_id & 146,580 \\
 Subsidiary and year fixed effects & YES \\ \hline
\multicolumn{2}{c}{ Robust standard errors in parentheses} \\
\multicolumn{2}{c}{ *** p$<$0.01, ** p$<$0.05, * p$<$0.1} \\
\end{tabular}
}
		\label{tab:11}
	\end{table}

\begin{table}	
	\caption{}
	\scalebox{0.8}{\begin{tabular}{lcccccc}
\multicolumn{7}{c}{Macroprudential policy and optimal capital adjustment (capital related indexes)} \\ \hline
 & (1) & (2) & (3) & (4) & (5) & (6) \\
VARIABLES &  &  &  &  &  &  \\ \hline
 &  &  &  &  &  &  \\
Tangibility & -0.105 & -0.103 & -0.099 & -0.103 & -0.101 & -0.112 \\
 & (0.113) & (0.112) & (0.112) & (0.112) & (0.112) & (0.112) \\
Log of fixed assets & -0.009 & -0.011 & -0.011 & -0.010 & -0.011 & -0.008 \\
 & (0.017) & (0.017) & (0.017) & (0.017) & (0.017) & (0.017) \\
Profitability & -0.231*** & -0.230*** & -0.229*** & -0.230*** & -0.231*** & -0.231*** \\
 & (0.083) & (0.084) & (0.083) & (0.084) & (0.084) & (0.084) \\
Inflation & 0.624 & 0.587 & 0.620 & 0.638 & 0.553 & 0.653 \\
 & (0.589) & (0.532) & (0.535) & (0.530) & (0.542) & (0.574) \\
Political Risk & 0.006* & 0.007** & 0.007** & 0.007** & 0.007** & 0.006** \\
 & (0.003) & (0.003) & (0.003) & (0.003) & (0.003) & (0.003) \\
Private credit to GDP & 0.048 & 0.061 & 0.061 & 0.049 & 0.071 & 0.029 \\
 & (0.121) & (0.117) & (0.118) & (0.116) & (0.117) & (0.120) \\
Tax rate & 0.088 & 0.148 & 0.151 & 0.137 & 0.156 & 0.084 \\
 & (0.194) & (0.185) & (0.183) & (0.183) & (0.186) & (0.196) \\
Tax rate spillover & -0.074 & -0.016 & -0.045 & -0.020 & -0.026 & -0.028 \\
 & (0.168) & (0.158) & (0.164) & (0.159) & (0.161) & (0.164) \\
Capital requirement & 0.030 &  &  &  &  & 0.027 \\
 & (0.030) &  &  &  &  & (0.030) \\
Capital requirement spillover & 0.018 &  &  &  &  & 0.018 \\
 & (0.034) &  &  &  &  & (0.033) \\
Capital buffer - overall &  & 0.008 &  &  &  &  \\
 &  & (0.028) &  &  &  &  \\
Capital buffer - overall spillover &  & 0.033 &  &  &  &  \\
 &  & (0.029) &  &  &  &  \\
Capital buffer - consumers &  &  & 0.036 &  &  & 0.000 \\
 &  &  & (0.096) &  &  & (0.000) \\
Capital buffer - consumers spillover &  &  & 0.075 &  &  & -0.039 \\
 &  &  & (0.084) &  &  & (0.122) \\
Capital buffer - others &  &  &  & -0.008 &  & -0.011 \\
 &  &  &  & (0.104) &  & (0.113) \\
Capital buffer - others spillover &  &  &  & 0.143 &  & 0.154 \\
 &  &  &  & (0.091) &  & (0.125) \\
Capital buffer - real estate &  &  &  &  & 0.025 & 0.009 \\
 &  &  &  &  & (0.041) & (0.045) \\
Capital buffer - real estate spillover &  &  &  &  & 0.030 & 0.018 \\
 &  &  &  &  & (0.040) & (0.036) \\
 &  &  &  &  &  &  \\
Observations & 507 & 507 & 507 & 507 & 507 & 507 \\
Number of firm\_id & 468 & 468 & 468 & 468 & 468 & 468 \\
 Subsidiary and year fixed effects & YES & YES & YES & YES & YES & YES \\ \hline
\multicolumn{7}{c}{ Robust standard errors in parentheses} \\
\multicolumn{7}{c}{ *** p$<$0.01, ** p$<$0.05, * p$<$0.1} \\
\end{tabular}
}
	\label{tab:12}
\end{table}

\begin{table}	
	\caption{}
	\scalebox{0.8}{\begin{tabular}{lcccccc}
\multicolumn{7}{c}{Macroprudential policy and Capital structure partial adjustment: multinational*time fixed effects} \\ \hline
 & (1) & (2) & (3) & (4) & (5) & (6) \\
VARIABLES &  &  &  &  &  &  \\ \hline
 &  &  &  &  &  &  \\
Tangibility & -0.042*** & -0.042*** & -0.042*** & -0.042*** & -0.042*** & -0.042*** \\
 & (0.001) & (0.001) & (0.001) & (0.001) & (0.001) & (0.001) \\
Log of fixed assets & 0.004*** & 0.004*** & 0.004*** & 0.004*** & 0.004*** & 0.004*** \\
 & (0.000) & (0.000) & (0.000) & (0.000) & (0.000) & (0.000) \\
Profitability & -0.138*** & -0.138*** & -0.138*** & -0.138*** & -0.138*** & -0.138*** \\
 & (0.002) & (0.002) & (0.002) & (0.002) & (0.002) & (0.002) \\
Inflation & 0.004 & 0.011 & 0.002 & 0.027 & 0.012 & 0.017 \\
 & (0.021) & (0.021) & (0.021) & (0.021) & (0.021) & (0.022) \\
Political Risk & 0.001*** & 0.000*** & 0.000*** & 0.001*** & 0.000*** & 0.000*** \\
 & (0.000) & (0.000) & (0.000) & (0.000) & (0.000) & (0.000) \\
Private credit to GDP & -0.003*** & -0.004*** & -0.003*** & -0.005*** & -0.003*** & -0.002*** \\
 & (0.001) & (0.001) & (0.001) & (0.001) & (0.001) & (0.001) \\
Tax rate & 0.021*** & 0.023*** & 0.022*** & 0.015*** & 0.021*** & 0.016*** \\
 & (0.002) & (0.002) & (0.002) & (0.002) & (0.002) & (0.003) \\
Reserve req. on local currency & -0.001*** &  &  &  &  & -0.000** \\
 & (0.000) &  &  &  &  & (0.000) \\
Reserve req. on foreign currency &  & 0.001*** &  &  &  & 0.001* \\
 &  & (0.000) &  &  &  & (0.000) \\
LTV ratio limits &  &  & 0.002*** &  &  & 0.001*** \\
 &  &  & (0.000) &  &  & (0.000) \\
Concentration limit &  &  &  & 0.001*** &  & 0.001*** \\
 &  &  &  & (0.000) &  & (0.000) \\
Interbank exposure limit &  &  &  &  & 0.003*** & 0.003*** \\
 &  &  &  &  & (0.000) & (0.000) \\
 &  &  &  &  &  &  \\
Observations & 730,993 & 730,993 & 730,993 & 730,993 & 730,993 & 730,993 \\
R-squared & 0.766 & 0.766 & 0.766 & 0.766 & 0.766 & 0.766 \\
Number of debt\_shifting\_group & 162,110 & 162,110 & 162,110 & 162,110 & 162,110 & 162,110 \\
 Multinational-year and country fixed effects & YES & YES & YES & YES & YES & YES \\ \hline
\multicolumn{7}{c}{ Robust standard errors in parentheses} \\
\multicolumn{7}{c}{ *** p$<$0.01, ** p$<$0.05, * p$<$0.1} \\
\end{tabular}
}
	\label{tab:13}
\end{table}

\begin{table}	
	\caption{}
	\scalebox{0.8}{\begin{tabular}{lcccccc}
\multicolumn{7}{c}{Macroprudential policy and Capital structure partial adjustment (capital related variables): multinational*time fixed effects} \\ \hline
 & (1) & (2) & (3) & (4) & (5) & (6) \\
VARIABLES &  &  &  &  &  &  \\ \hline
 &  &  &  &  &  &  \\
Tangibility & -0.042*** & -0.042*** & -0.042*** & -0.042*** & -0.042*** & -0.042*** \\
 & (0.001) & (0.001) & (0.001) & (0.001) & (0.001) & (0.001) \\
Log of fixed assets & 0.004*** & 0.004*** & 0.004*** & 0.004*** & 0.004*** & 0.004*** \\
 & (0.000) & (0.000) & (0.000) & (0.000) & (0.000) & (0.000) \\
Profitability & -0.138*** & -0.138*** & -0.138*** & -0.138*** & -0.138*** & -0.138*** \\
 & (0.002) & (0.002) & (0.002) & (0.002) & (0.002) & (0.002) \\
Inflation & 0.015 & -0.005 & 0.015 & 0.013 & -0.012 & -0.007 \\
 & (0.021) & (0.021) & (0.021) & (0.021) & (0.021) & (0.021) \\
Political Risk & 0.000*** & 0.000*** & 0.000*** & 0.000*** & 0.000*** & 0.000*** \\
 & (0.000) & (0.000) & (0.000) & (0.000) & (0.000) & (0.000) \\
Private credit to GDP & -0.004*** & -0.006*** & -0.005*** & -0.005*** & -0.005*** & -0.005*** \\
 & (0.001) & (0.001) & (0.001) & (0.001) & (0.001) & (0.001) \\
Tax rate & 0.022*** & 0.014*** & 0.020*** & 0.020*** & 0.015*** & 0.017*** \\
 & (0.002) & (0.002) & (0.002) & (0.002) & (0.002) & (0.002) \\
Capital requirement & -0.002** &  &  &  &  & -0.003** \\
 & (0.001) &  &  &  &  & (0.001) \\
Capital buffer - overall &  & -0.002*** &  &  &  &  \\
 &  & (0.000) &  &  &  &  \\
Capital buffer - consumers &  &  & -0.007*** &  &  & -0.003* \\
 &  &  & (0.001) &  &  & (0.002) \\
Capital buffer - others &  &  &  & -0.001 &  & 0.001 \\
 &  &  &  & (0.001) &  & (0.001) \\
Capital buffer - real estate &  &  &  &  & -0.003*** & -0.003*** \\
 &  &  &  &  & (0.000) & (0.000) \\
 &  &  &  &  &  &  \\
Observations & 730,993 & 730,993 & 730,993 & 730,993 & 730,993 & 730,993 \\
R-squared & 0.766 & 0.766 & 0.766 & 0.766 & 0.766 & 0.766 \\
Number of debt\_shifting\_group & 162,110 & 162,110 & 162,110 & 162,110 & 162,110 & 162,110 \\
 Multinational-year and country fixed effects & YES & YES & YES & YES & YES & YES \\ \hline
\multicolumn{7}{c}{ Robust standard errors in parentheses} \\
\multicolumn{7}{c}{ *** p$<$0.01, ** p$<$0.05, * p$<$0.1} \\
\end{tabular}
}
	\label{tab:14}
\end{table}
	\section{Conclusion} \label{sec:conclusion}
	
	
	
	\singlespacing
	\bibliography{C:/Users/User/work/master_thesis/analysis/code/text/bibliography/references}
	\bibliographystyle{C:/Users/User/work/master_thesis/analysis/code/text/bibliography/te}
	
	
	
%	\clearpage
	
%	\onehalfspacing
	
%	\section*{Tables} \label{sec:tab}
%	\addcontentsline{toc}{section}{Tables}
	
		%\input{C:/Users/User/work/master_thesis/analysis/temp/summary_MPI}
	
		%\input{C:/Users/User/work/master_thesis/analysis/temp/summary_MPI_year}
		
		%\input{C:/Users/User/work/master_thesis/analysis/temp/Tex/number_firms_orbis}
		
		%\input{C:/Users/User/work/master_thesis/analysis/temp/Tex/summary_firm_orbis_clean}
		
		%\input{C:/Users/User/work/master_thesis/analysis/temp/Tex/summary_country_orbis}
	
	%	\begin{table}
	%		\centering
	%		\caption{Number of firms that changed ownership}
	%		\label{tab:movers}
	%	%            &\multicolumn{1}{c}{(1)}\\
            &\multicolumn{1}{c}{Movers}\\
            &           b\\
\hline
2           &       27990\\
3           &        7544\\
4           &        1783\\
5           &         314\\
6           &          50\\
7           &           1\\
Total       &       37682\\

	%\end{table}

		
	%\section*{Figures} \label{sec:fig}
	%\addcontentsline{toc}{section}{Figures}
	
	%\begin{figure}[hp]
	%  \centering
	%  \includegraphics[width=.6\textwidth]{../fig/placeholder.pdf}
	%  \caption{Placeholder}
	%  \label{fig:placeholder}
	%\end{figure}
	
	
%	\clearpage
	
%	\section*{Appendix A.} 
%	\label{sec:appendixa}
%	\addcontentsline{toc}{section}{Appendix A}
	%% Please add the following required packages to your document preamble:
% \usepackage{graphicx}
%\begin{longtable}[]
	\centering
	%\resizebox{\textwidth}{!}{%
		\begin{longtable}{p{1.7in}p{2.6in}p{1.7in}}
				\label{tab:definition}\\
			\multicolumn{3}{c}{Table \ref{tab:definition} - Variable definitions and data sources}\\
			\hline 
			Variable      & Definition & Source \\
			\hline \endfirsthead
			
				\multicolumn{3}{c}{Table \ref{tab:definition} - Variable definitions and data sources \textit{(Continued)}}\\
			\hline 
			Variable      & Definition & Source \\
			\hline \endhead
			
			\hline
			\multicolumn{3}{r}{{\textit{Continued}}}\\ 
			\endfoot
			\hline
			\endlastfoot
			Financial leverage      & Ratio of non-equity liabilities to total assets & Orbis \\
		
			Reserve req. on local currency & Average of quarterly index of changes in reserve requirements on local currency & \cite{cerutti2017changes}\\
			Capital requirement & Average of quarterly index of changes in capital requirements & \cite{cerutti2017changes}\\
			Reserve req. on local currency spillover & Sum of international reserve requirement on local currency differences weighted by local asset shares& \cite{cerutti2017changes}\\
			Capital requirement spillover & Average of quarterly index of changes in capital requirements  & \cite{cerutti2017changes}\\
			Tax rate & Taxes and other mandatory contributions after accounting  for deductions and exemptions to total commercial profit & World Bank Doing Business indicators\\
			Tax rate spillover & Sum of international corporate tax rate  differences weighted by local asset shares& World Bank Doing Business indicators\\
			Tangibility& Ratio of fixed assets to total assets & Orbis\\
			Log of fixed assets& logarithm of fixed assets & Orbis\\
			Profitability& Ratio of EBITDA to total assets & Orbis\\
			Risk & Standard deviation of the firm's ratio of EBITDA to total assets over the period 2008-2014& Orbis\\
			Opportunity & Median of the annual growth rate of sales per country and industry& Orbis\\
			Private credit to GDP & Ratio of credit to the private sector to GDP & World Bank indicators\\	
			Inflation & Annual log change in the CPI & World Bank indicators\\
			GDP growth rate &Annual percentage change in the GDP & World Bank indicators\\
			 Policy rate &Central Bank Policy Rate, Percent per annum & IMF International Financial Statistics \\
			Exchange rate risk &Annual (December) index of exchange rate risk &International Country Risk Guide\\
			Law and order &Annual (December) index of law and order &International Country Risk Guide\\
			Political risk &Annual (December) index of political risk &International Country Risk Guide\\
		\end{longtable}%
	%}
%\end{longtable}

\end{document} 