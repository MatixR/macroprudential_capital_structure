\documentclass[12pt]{article}
\usepackage[utf8]{inputenc}
\usepackage{amssymb,amsmath,amsfonts,eurosym,geometry,ulem,caption,color,xcolor,multicol,setspace,sectsty,comment,footmisc,caption,natbib,pdflscape,subfigure,array,hyperref,verbatim,mathpazo,longtable}
%\usepackage[normalem]{ulem}
\usepackage[pdftex]{graphicx}
\graphicspath{{Imagens/}}
\usepackage{fullpage}
\usepackage{indentfirst}
\usepackage{changepage}

%\setlength{\pdfpagewidth}{8.5in} \setlength{\pdfpageheight}{11in}
%\setlength{\textheight}{8.5in} \setlength{\topmargin}{0.0in}
%\setlength{\headheight}{0.0in} \setlength{\headsep}{0.0in}
%\setlength{\leftmargin}{0.5in}
%\setlength{\oddsidemargin}{0.0in}
%\setlength{\parindent}{2em}
%\setlength{\parskip}{\baselineskip}%
%\setlength{\textwidth}{6.5in}
%\linespread{1.6}
\newcommand*{\captionsource}[2]{%
  \caption[{#1}]{%
    #1%
    \\\hspace{\linewidth}%
    \textbf{Source:} #2%
  }%
}
\newcommand{\horrule}[1]{\rule{\linewidth}{#1}} % Create horizontal rule command with 1 argument of height

\onehalfspacing
\newtheorem{theorem}{Theorem}
\newtheorem{corollary}[theorem]{Corollary}
\newtheorem{proposition}{Proposition}
\newenvironment{proof}[1][Proof]{\noindent\textbf{#1.} }{\ \rule{0.5em}{0.5em}}

\newtheorem{hyp}{Hypothesis}
\newtheorem{subhyp}{Hypothesis}[hyp]
\renewcommand{\thesubhyp}{\thehyp\alph{subhyp}}

\newcommand{\red}[1]{{\color{red} #1}}
\newcommand{\blue}[1]{{\color{blue} #1}}

\newcolumntype{L}[1]{>{\raggedright\let\newline\\arraybackslash\hspace{0pt}}m{#1}}
\newcolumntype{C}[1]{>{\centering\let\newline\\arraybackslash\hspace{0pt}}m{#1}}
\newcolumntype{R}[1]{>{\raggedleft\let\newline\\arraybackslash\hspace{0pt}}m{#1}}

\geometry{left=1.0in,right=1.0in,top=1.0in,bottom=1.0in}
\begin{document}
	
	\begin{titlepage}
		\title{Master Thesis: \\ Macroprudential Policy International Spill-over: Evidence from Multinational's Capital Structure\thanks{Preliminary version}}
		\author{Lucas Avezum\thanks{abc}}
		%\author{Lucas Avezum\thanks{abc} \and \and Harry Huizinga\thanks{abc} \and Louis Raes\thanks{abc}}
		\date{\today}
		\maketitle
		\begin{abstract}
			\noindent Placeholder\\
			\vspace{0in}\\
			\noindent\textbf{Keywords:} key1, key2, key3\\
			\vspace{0in}\\
			\noindent\textbf{JEL Codes:} key1, key2, key3\\
			
			\bigskip
		\end{abstract}
		\setcounter{page}{0}
		\thispagestyle{empty}
	\end{titlepage}
	\pagebreak \newpage
	
	
	
	
	\doublespacing
	
	
	\section{Introduction} \label{sec:introduction}
	As \cite*{blinder2016necessity} have documented, the mandate of central banks have increased to include several macro-prudential tools. Triggered mostly by the 2008 financial crisis, financial stability is perceived by many central banks as a goal together with price stability. However, as \cite{blinder2016necessity} note, to what extend should macro-prudential policy be used in the future is still unclear. Given its early stage, the effectiveness and externalities of macro-prudential instruments have still to be properly evaluated. In special, it is reasonable to expect cross-border spillover of prudential policies choices. 
	
	This paper provides one of the first answers to the international effect of macro-prudential policy question. More specifically, we show that changes in a one country's macro-prudential tool affect firm's leverage decision in another country. 
	
	% \section{Literature review} \label{sec:literature}
	Empirical research on the effects of prudential policies has increase considerably after the 2008 financial crisis. With loan-level data \cite*{jimenez2012macroprudential} study effects of capital buffers in Spain. Using firm-level data, \cite*{ayyagari2017credit} relate macroprudential policies and firm's credit growth. They found that the effectiveness of prudential tools at smoothing credit depend on firm's location (emerging or developed country), size, debt maturity and type of instrument (borrower-target or financial institution target).
	
	Our work is closely related to the series of papers developed under the International Banking Research Network (IBRN) 2015 initiative, summarized by \cite{buch2017international}. They also rely on multinational relationships to identify international spillovers. However, their analysis is restricted to the effects on banking loans, while we study changes in firm's capital structure. Their findings corroborate to the evidence that the effects of prudential tools are heterogeneous on type and location. On average, the economic significance of international spillover were found to be small. Although at first we might expect our results to follow theirs, our estimation strategy allows to identify changes in firm's debt that goes beyond banking credit supply. Hence, the main contribution of the paper is to show that macroprudential policy affect firm's leverage by changing their credit condition relative to other firm's within the same multinational group. Our finding provide stronger evidence of prudential tools spillover, that should be considered given the original objetive of providing financial stability.
	
	The paper proceeds as follows: Section 2 explains the empirical strategy, Section 3 describes the data and the results are discussed in Section 4. We provide extensions to our baseline model in Section 5. Section 7 concludes. 
	
	\section{Empirical strategy} \label{sec:strategy}
	
	
	In order to assess potential international spillovers from macroprudential policy our estimation strategy is based on \cite*{huizinga2008capital}. Although their work was designed to study tax incentive to debt shifting, we believe that the same strategy is also suitable to our case. As \cite{huizinga2008capital} note, multinational groups take into account both tax and non-tax factor when deciding their optimal financial structure. For instance, looser credit market in one country may lead to relative higher leverage at subsidiaries in this country, as the multinational will take advantage of better financing conditions. Given an optimal leverage at the multinational level, subsidiaries at other countries should reduce external financing. As macroprudential instruments aim at smoothing credit markets, to the degree of their domestic effectiveness, internatinal spillover in the form of debt shifting seems plausible.
	
	Macroprudential cross-border effect is identified by averaging the differentials of indexes by the asset share. Our baseline regression is:
	
	\begin{equation*}
	\begin{aligned}
	y_{imc,t}=&\beta_0MPI_{c,t}+\beta_1\sum_j\bigg[MPI_{ic,t} - MPI_{jc',t}\bigg]\rho_{j,t}\\
	&+\alpha_0Tax_{ic,t}+\alpha_1\sum_j\bigg[Tax_{ic,t} - Tax_{jc',t}\bigg]\rho_{j,t}\\
	&+\Gamma X_{t}+f_{m}+f_{t}+f_{c}+\varepsilon_{i,t}
	\end{aligned}
	\end{equation*}
	
	Where $i$ denotes firm, $m$ multinational group (indexed by parent), $c$ country and $t$ year. The dependent variable is the log change of a capital structure measure such as leverage, adjusted leverage, long term or short term debt. $MPI$ is a macroprudential measure such as reserve requirement, LTV-ratio. $\rho_{i,t}$ is the asset share within multinational of firm $i$ at time $t$. $f_{m}$, $f_{t}$ and $f_{c}$ are respectively multinational, time and country fixed effects. $X$ is a set of controls variables.
	
	We are interest both in $\beta_0$ and $\beta_1$, respectively, the domestic and international effect of prudential policies on firm's capital structure choice. By our hypothesis and the way we construct the debt shifting variable we expect both $\beta_0$ and $\beta_1$ to be negative. Given prior evidence, we control by both corporate tax level at the host country as by tax incentive to shift debt \footnote{our measure of corporate tax is only a proxy of what \cite{huizinga2008capital} have construct, but the results remain in line}. We also control for several other firm and country level characteristics as described in Section \ref{sec:data}.
	
	\section{Data} \label{sec:data}
	\subsection{Macroprudential policy} \label{subsec:MPI}
	
	We rely on the efforts of \cite*{cerutti2016changes} on our measure of macroprudential policy \footnote{ In turn, their work build on \cite*{cerutti2015use}}, which was also developed within the IBRN macroprudential study initiative. The measures constructed consists of indexes of 9 instruments: general capital requirements (GCR), real estate loans capital buffer (RECB), consumer loans capital buffer (CCB), concentration limits (CL), interbank exposure (IE), loan-to-value ratio limits (LTV), reserve requeriments on foreign currency (RRFC) and reserve requirements on local currency (RRLC). The time unit is quarterly and the database span from the first quarter of 2001 to the last quarter of 2014. Sixty four countries are included. Each quarter data point is an positive, negative or zero entry meaning respectively, tightening, loosening or no change in the instrument. The indexes not only track the use but also the intensity of each change. A cumulative measure of each instrument is also available.    
	
	Due to firm level data availability we only use XX countries from 2007 to 2014 and average the indexes per year. Our main focus in this paper is to study the effects of reserve requirements and loan-to-value ratios as those measures have been shown in the literature to be used in counter-cyclically. The other prudential tools are little or less correlated to business cycles measures suggesting a structural use, as noted by \cite{cerutti2016changes}. 
	
	Table \ref{tab:1} shows the mean and standard deviation of prudential indexes by country. On one hand, loan-to-value ratios are used only by 36 countries out of a total of 64 in our sample. On the other hand, LTV ratios are more intensively used, as seen by a higher standard deviation relative to the reserve requirement indexes. During the sample period, LTV ratios have on average been tightened, exceptions being, Chile, Iceland, Luxembourg and Spain. Israel appears the country that most changed LTV ratios during the sample period with 0.58 standard deviation while Colombia and Portugal had no change in their policy. Reserve requirements are used in all countries as this instrument is also part of the monetary policy tool kit. Reserve requirements on foreign currency on average were slightly tightened but most countries did not changed their policy stance with this tool between 2001 and 2014. Changes in local currency reserve requirements were more used both extensive and intensively. Also, this was the only instrument to have been loosened in the sample period on average. Importantly, the indexes for both reserve requirement measures are the same for all members of the euro area, as those tools are set by the European Central Bank (ECB). Moreover, neither the US, the UK or Japan have policies measures for LTV ratio limits or changed their stance with respect to reserve requirements.   
	
	Summary statistics by year are shown in table \ref{tab:2}
	    
	
    
    
	\subsection{Firm level} \label{subsec:firm}
	Firm level data and ownership relationships are taken from the Orbis database compiled by Bureau Van Dijk.	The dataset consist of worldwide accounting and ownership information on both private and public owned companies. Ownership is considered if one firm owns at least 50\% of another at a given year. We call the second a subsidiary firm. If a company owns one or more firms but itself is owned by another, this firm is then called an intermediate and for all purposes is also considerer as a subsidiary. A parent firm is the ultimate owner of a group, that is, a firm that owns one or more companies but none of its shareholders have more than 50\% of its shares.
	
	Given our interest in cross-border effects, we keep only multinationals groups, that is, ownership networks that have firms in at least two countries. Banks, insurance and financial related companies are also excluded since their capital structure decision differs from other industries \footnote{Check IBRN studies on the banking sector}. Only multinatinals that we have information of the parent firm are kept. Table \ref{tab:number} provides information on the amount of parent and subsidiary firms per host and home country in our sample. The total number of parent and subsidiary firms are 64,810 and 305,367 respectively. There are 51,255 subsidiaries which ownership has change during the period analyzed at least once, as shown in table \ref{tab:movers}. 
	%Our sample goes from 2007 to 2015, yielding XXX parent-year and XXX subsidiary-year observations. DESCRIBE COUNTRY SPECIFICS.  
	
	Statistics on firm level data is summarized in table 4. First, leverage is defined as the ratio of total non-equity liabilities to total assets. Adjusted financial leverage is a similar measure but subtracting cash and equivalent from both numerator and denominator. Our variable of interest is the macroprudential incentive to shift debt described in section \ref{sec:strategy}. Among the control variables tangibility is construct as the ratio of fixed assets to total assets. As an alternative measure, adjusted tangibility is the ratio of tangible fixed assets to total assets. We use the logarithm of sales to proxy for firm size. Profitability is the ratio of earning before interest, tax, depreciation and amortization (EBITDA) to total assets. 
	%%Growth opportunities is BLA BLA
	
	Table 5 provides statistics at the country level. We control for several institutional measures from the \textit{International Country Risk Guide}, such as political, economic, financial exchange rate risk and law and order. 
	%Importantly, given that our multinationals can operate at countries with different currencies, we also control for exchange risks to account for possible under allocation of group's resource due to fears of host country currency volatility. As an alternative measure we control for the standard deviation of host country currency to the north American dollar during the period as a proxy for currency risk. 
	We also include several macro controls from the World Development Indicators database of the World bank. To account for the effects of business cycle on leverage we add GDP growth as the annual change of GDP. 
	%GDP per capita is included in some specifications to control for BLA BLA. 
	Inflation is the annual log change in the consumer price index.     
	
	
	
	     
	\section{Results} \label{sec:result}

	\section{Discussions} \label{sec:discussion}
	
	\section{Conclusion} \label{sec:conclusion}
	
	
	
	\singlespacing
	\bibliography{C:/Users/User/work/master_thesis/analysis/code/text/bibliography/references}
	\bibliographystyle{C:/Users/User/work/master_thesis/analysis/code/text/bibliography/te}
	
	
	
	\clearpage
	
	\onehalfspacing
	
	\section*{Tables} \label{sec:tab}
	\addcontentsline{toc}{section}{Tables}
	
		\input{C:/Users/User/work/master_thesis/analysis/temp/summary_MPI}
	
		\input{C:/Users/User/work/master_thesis/analysis/temp/summary_MPI_year}
		
		\input{C:/Users/User/work/master_thesis/analysis/temp/Tex/number_firms_orbis}
		
		\input{C:/Users/User/work/master_thesis/analysis/temp/Tex/summary_firm_orbis_clean}
		
		\input{C:/Users/User/work/master_thesis/analysis/temp/Tex/summary_country_orbis}
	
		\begin{table}
			\centering
			\caption{Number of firms that changed ownership}
			\label{tab:movers}
		            &\multicolumn{1}{c}{(1)}\\
            &\multicolumn{1}{c}{Movers}\\
            &           b\\
\hline
2           &       27990\\
3           &        7544\\
4           &        1783\\
5           &         314\\
6           &          50\\
7           &           1\\
Total       &       37682\\

	\end{table}
	\begin{table}	
	\caption{Macroprudential policy effect on firm's financial leverage}
	\label{tab:regression}
	\scalebox{0.8}{\input{C:/Users/User/work/master_thesis/analysis/temp/Tex/regression_orbis_leverage}}
\end{table}
		
	%\section*{Figures} \label{sec:fig}
	%\addcontentsline{toc}{section}{Figures}
	
	%\begin{figure}[hp]
	%  \centering
	%  \includegraphics[width=.6\textwidth]{../fig/placeholder.pdf}
	%  \caption{Placeholder}
	%  \label{fig:placeholder}
	%\end{figure}
	
	
	\clearpage
	
	\section*{Appendix A.} 
	\label{sec:appendixa}
	\addcontentsline{toc}{section}{Appendix A}
	% Please add the following required packages to your document preamble:
% \usepackage{graphicx}
%\begin{longtable}[]
	\centering
	%\resizebox{\textwidth}{!}{%
		\begin{longtable}{p{1.7in}p{2.6in}p{1.7in}}
				\label{tab:definition}\\
			\multicolumn{3}{c}{Table \ref{tab:definition} - Variable definitions and data sources}\\
			\hline 
			Variable      & Definition & Source \\
			\hline \endfirsthead
			
				\multicolumn{3}{c}{Table \ref{tab:definition} - Variable definitions and data sources \textit{(Continued)}}\\
			\hline 
			Variable      & Definition & Source \\
			\hline \endhead
			
			\hline
			\multicolumn{3}{r}{{\textit{Continued}}}\\ 
			\endfoot
			\hline
			\endlastfoot
			Financial leverage      & Ratio of non-equity liabilities to total assets & Orbis \\
		
			Reserve req. on local currency & Average of quarterly index of changes in reserve requirements on local currency & \cite{cerutti2017changes}\\
			Capital requirement & Average of quarterly index of changes in capital requirements & \cite{cerutti2017changes}\\
			Reserve req. on local currency spillover & Sum of international reserve requirement on local currency differences weighted by local asset shares& \cite{cerutti2017changes}\\
			Capital requirement spillover & Average of quarterly index of changes in capital requirements  & \cite{cerutti2017changes}\\
			Tax rate & Taxes and other mandatory contributions after accounting  for deductions and exemptions to total commercial profit & World Bank Doing Business indicators\\
			Tax rate spillover & Sum of international corporate tax rate  differences weighted by local asset shares& World Bank Doing Business indicators\\
			Tangibility& Ratio of fixed assets to total assets & Orbis\\
			Log of fixed assets& logarithm of fixed assets & Orbis\\
			Profitability& Ratio of EBITDA to total assets & Orbis\\
			Risk & Standard deviation of the firm's ratio of EBITDA to total assets over the period 2008-2014& Orbis\\
			Opportunity & Median of the annual growth rate of sales per country and industry& Orbis\\
			Private credit to GDP & Ratio of credit to the private sector to GDP & World Bank indicators\\	
			Inflation & Annual log change in the CPI & World Bank indicators\\
			GDP growth rate &Annual percentage change in the GDP & World Bank indicators\\
			 Policy rate &Central Bank Policy Rate, Percent per annum & IMF International Financial Statistics \\
			Exchange rate risk &Annual (December) index of exchange rate risk &International Country Risk Guide\\
			Law and order &Annual (December) index of law and order &International Country Risk Guide\\
			Political risk &Annual (December) index of political risk &International Country Risk Guide\\
		\end{longtable}%
	%}
%\end{longtable}

\end{document} 