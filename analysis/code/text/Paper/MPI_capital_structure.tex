\documentclass[12pt]{article}
\usepackage[utf8]{inputenc}
\usepackage{amssymb,amsmath,amsfonts,eurosym,geometry,ulem,caption,color,xcolor,multicol,setspace,sectsty,comment,footmisc,caption,natbib,pdflscape,subfigure,array,hyperref,verbatim,mathpazo,longtable,ntheorem,siunitx}
%\usepackage[normalem]{ulem}
\usepackage[pdftex]{graphicx}
\graphicspath{{Imagens/}}
\usepackage{fullpage}
\usepackage{indentfirst}
\usepackage{changepage}

%\setlength{\pdfpagewidth}{8.5in} \setlength{\pdfpageheight}{11in}
%\setlength{\textheight}{8.5in} \setlength{\topmargin}{0.0in}
%\setlength{\headheight}{0.0in} \setlength{\headsep}{0.0in}
%\setlength{\leftmargin}{0.5in}
%\setlength{\oddsidemargin}{0.0in}
%\setlength{\parindent}{2em}
%\setlength{\parskip}{\baselineskip}%
%\setlength{\textwidth}{6.5in}
%\linespread{1.6}
\newcommand*{\captionsource}[2]{%
  \caption[{#1}]{%
    #1%
    \\\hspace{\linewidth}%
    \textbf{Source:} #2%
  }%
}
\newcommand{\horrule}[1]{\rule{\linewidth}{#1}} % Create horizontal rule command with 1 argument of height

\onehalfspacing
\newtheorem{theorem}{Theorem}
\newtheorem{corollary}[theorem]{Corollary}
\newtheorem{proposition}{Proposition}
\newenvironment{proof}[1][Proof]{\noindent\textbf{#1.} }{\ \rule{0.5em}{0.5em}}

\newtheorem{hyp}{Hypothesis}
\newtheorem{subhyp}{Hypothesis}[hyp]
\renewcommand{\thesubhyp}{\thehyp\alph{subhyp}}

\newcommand{\red}[1]{{\color{red} #1}}
\newcommand{\blue}[1]{{\color{blue} #1}}

\newcolumntype{L}[1]{>{\raggedright\let\newline\\arraybackslash\hspace{0pt}}m{#1}}
\newcolumntype{C}[1]{>{\centering\let\newline\\arraybackslash\hspace{0pt}}m{#1}}
\newcolumntype{R}[1]{>{\raggedleft\let\newline\\arraybackslash\hspace{0pt}}m{#1}}

\geometry{left=1.0in,right=1.0in,top=1.0in,bottom=1.0in}
\begin{document}
	
	\begin{titlepage}
		\title{Macroprudential Policies and Capital Structure: Evidence from Multinationals \thanks{Preliminary version}}
		\author{Lucas Avezum\thanks{abc}}
		%\author{Lucas Avezum\thanks{abc} \and \and Harry Huizinga\thanks{abc} \and Louis Raes\thanks{abc}}
		\date{\today}
		\maketitle
		\begin{abstract}
			\noindent Placeholder\\
			\vspace{0in}\\
			\noindent\textbf{Keywords:} key1, key2, key3\\
			\vspace{0in}\\
			\noindent\textbf{JEL Codes:} key1, key2, key3\\
			
			\bigskip
		\end{abstract}
		\setcounter{page}{0}
		\thispagestyle{empty}
	\end{titlepage}
	\pagebreak \newpage
	
	
	
	
	\doublespacing
	
	
	\section{Introduction} \label{sec:introduction}
	As \cite*{blinder2016necessity} documented, the mandate of central banks has expanded considerably. Triggered mostly by the 2008 financial crisis, financial stability is perceived by many central banks as a goal together with price stability. To achieve the former, several macro-prudential instruments were included or extended in central banker's toolbox. However, the authors also note that the extent in which macro-prudential policy will be used in the future is still unclear. Given its early stage, the effectiveness and externalities of macro-prudential instruments have still to be properly evaluated. 
	
	This paper studies how macro-prudential policies relate to firms' leverage. We also analyze potential international spillovers. More specifically, we show that changes in one country's macro-prudential tool affect firm's leverage decision hosted in another country. Our identification strategy relies on two aspects of our data: first, firm-level data allow us to study changes in macroprudential policies that are exogenous to each individual firm; and second, by studying multinationals we use variations in prudential tools that are not common to all firms within a multinational group.
	
	In order to assess potential international spillovers from macro-prudential policy our estimation strategy is based on \cite*{huizinga2008capital}. Although their work was designed to study tax incentive to shift debt, we believe that the same strategy is also suitable to our case. Multinational groups take into account both tax and non-tax factors when deciding their optimal financial structure. For instance, looser credit market in one country may lead to relative higher leverage at subsidiaries in this country, as the multinational will take advantage of better financing conditions. Given an optimal leverage at the multinational level, subsidiaries at other countries should reduce external financing. As macro-prudential instruments aim at smoothing credit markets, to the degree of their domestic effectiveness, international spillover in the form of debt shifting are plausible.
	
	Empirical research on the effects of prudential policies has increased considerably after the 2008 financial crisis. Several authors rely on cross-country aggregate data to study the relationship between macro-prudential policies and financial indicators. \cite*{cerutti2017use} introduces a dataset containing the usage of 12 macro-prudential policies at 119 countries from 2000 to 2013. They document that prudential instruments are mostly used in developing countries where they also appear to be more effective. Among their findings, borrower-related policies, such as limits on LTVs and DTIs ratios are associated with reductions in credit growth and house prices. The author also provide an extensive review of the literature on macro-prudential policies. 
	
	A second line of studies provides micro-level evidence. With loan-level data \cite*{jimenez2012macroprudential} study the  effects of capital buffers in Spain. \cite*{aiyar2014does} find lending to leak through foreign branches following an increase in capital requirements to regulated banks. Their work shows that capital requirements in UK were effective in curbing lending for regulated banks, but the leakage is substantial. Most studies in this second group in the literature focus on few instruments, on the bank or housing sectors and on one country. An exception is \cite*{ayyagari2017credit} that uses firm-level data to relate macro-prudential policies across 119 countries to firm's credit growth. They find that the effectiveness of prudential tools at smoothing credit depend on the firm's location (emerging or developed country), size, debt maturity and the type of instrument (borrower-target or financial institution target). We extend on this line of research by also providing evidence of international spillovers.
	
	Our work is also closely related to the series of papers developed under the International Banking Research Network (IBRN) 2015 initiative. \cite*{buch2017cross} summarize the methodology and provide a meta-study. The authors also rely on multinational relationships to identify international spillovers. However, their analysis is restricted to the effects on banking loans, while we study changes in firm's capital structure. Among their findings, the effects of prudential tools are heterogeneous on type and location. On average, the economic significance of international spillover is found to be small.  
	
	Finally, this paper also builds on the optimal capital structure literature. \cite*{demirgucc1998law}, \cite*{beck2004bank}, show how institutions impact leverage decisions. Our contribution is to study if policies intended to affect credit markets are incorporated to firms beyond the financial sector.   
	
   The main contribution of the paper is to show that macro-prudential policy affect firm's leverage by changing their credit condition relative to other firms within the same multinational group. Our findings provide stronger evidence of prudential tools' spillover.
	
	The paper proceeds as follows: Section \ref{sec:strategy} explains the empirical strategy, Section \ref{sec:data} describes the data while the results are discussed in Section \ref{sec:result}. We provide extensions to our baseline model in Section \label{sec:discussion}. Section \ref{sec:conclusion} concludes. 
	
 
	\section{Data} \label{sec:data}	
	\subsection{Macroprudential policy} \label{subsec:MPI}
	
	We rely on the efforts of \cite*{cerutti2017changes} for our measure of macroprudential policy \footnote{In turn, their work builds on \cite*{cerutti2015use}, LIM, AKINCI, KUTNER SHIN REInhart and sowerbutts}, which was also developed within the IBRN macroprudential study initiative. The measures consist of indexes for nine instruments: reserve requirements on local currency, reserve requeriments on foreign currency, loan-to-value ratio limits, concentration limits, interbank exposure, general capital requirements, capital buffers divided in three subgroups: related to real estate loans, consumer loans and other loans. The time unit is a quarter and the database spans from the first quarter of 2001 to the last quarter of 2014. Sixty-four countries are included. Each quarter data point is a positive, negative or zero entry meaning tightening, loosening or no change in the instrument, respectively. Besides for reserve requirements, intensity of each change is not accounted for. Comparability across each move is difficult given the complexity in which measures can have. For instance, from the EXAMPLES OF MEASURES

Lower LTV ratio limits increase the quantity and quality of the collateral pledged against loans, thus, decreasing the cost of debt associated with it ($\phi_{ltv}<0$). Lastly, a tightening in sector specific capital buffers can turn funds once destined to this sector available to other sectors. Hence, substitution effect from capital buffers for real estate loans to affect firms' leverage ($\phi_{recb}\leq 0$). Still, banks may have other options to shift their funds rather than to firms, so this effect may not be significant.	 
	
    Cumulative measures of each instrument are also available. Importantly, those cumulative indexes allow us to track how tight or loose macroprudential policies were in time for each country but not across them. In section \ref{subsec:firm} we address this comparability issue. To match the IBRN dataset to our firm-level data, we construct cumulative indexes for each prudential tool starting at zero in the first quarter of 2007 adding the tightening and loosening data points until the last quarter of 2014. Moreover, due to firm-level data being available only at a yearly basis, we create 4-quarter moving averages of each macro-prudential index and match these data points according to the balance sheet closing date for each firm and year. As an example, take a firm that has reported its balance sheet on May 2012. The corresponding macro-prudential index for this firm is an average between the third and fourth quarters of 2011 and the first and second quarters of 2012. 
	  
 	\subsection{Firm level} \label{subsec:firm}
	Firm level data and ownership relationships are taken from the Orbis database compiled by Bureau Van Dijk.	The dataset consist of worldwide accounting and ownership information on both private and public owned companies. Ownership is considered if one firm owns at least 50\% of another in a given year. We call the second a subsidiary firm. If a company owns one or more firms but is itself owned by another, this firm is called an intermediate and for all purposes is also considered as a subsidiary. A parent firm is the ultimate owner of a group, that is, a firm that owns one or more companies but none of its shareholders have more than 50\% of its shares.
	
	Given our interest in cross-border effects, we keep only multinationals groups, that is, ownership networks that have firms in at least two countries. Banks, insurance and financial related companies\footnote{Check IBRN studies on the banking sector} and firms in the utility sector\footnote{NACE code} are also excluded since their capital structure decision is constrained by regulation.
	
	The benchmark sample consists of around 2.4 million firms from 2007 to 2015 resulting in approximately 8.8 million firm-year observations. Table \ref{tab:number} provides information on the amount of parent and subsidiary firms per host and home country in our sample. The total number of parent and subsidiary firms are 64,810 and 305,367, respectively. There are 51,255 subsidiaries for which ownership has changed during the period analyzed at least once, as shown in table \ref{tab:movers}.
	
	Statistics on firm level data are summarized in table 4. Financial leverage is defined as the ratio of total non-equity liabilities to total assets. Adjusted financial leverage is a similar measure but subtracting cash and equivalent from both the numerator and the denominator. Our variable of interest is the macro-prudential incentive to shift debt described in section \ref{sec:strategy}. Among the control variables tangibility is construct as the ratio of fixed assets to total assets. While tangible assets can be used as collateral, implying a positive relationship between tangibility and leverage, the depreciation of fixed assets reduces taxable income. Hence, tangible assets can also be substitute to debt as tax shield. We use the logarithm of fixed assets to proxy for firm size, which is expected to be positively associated to leverage. Profitability is the ratio of earning before interest, tax, depreciation and amortization (EBITDA) to total assets. Higher profits may facilitate access to credit but firms with larger cash flow may also opt to finance themselves with retained earnings. Thus, the relationship between profitability and leverage is ambiguous. 
	%As an alternative measure, adjusted tangibility is the ratio of tangible fixed assets to total assets. 
	%%Growth opportunities is BLA BLA
	\subsection{Country level} \label{subsec:country}
	Table 5 provides statistics at the country level. We control for political risk using the index from the \textit{International Country Risk Guide}. Higher scores mean lower political risk. \cite*{kesternich2010afraid} find that political risk can both increase or decrease firm leverage. A more unstable political environment may discourage banks to provide loans but also, parent firms might want to reduce their value at risk by leveraging their subsidiaries operations. We also include two macro controls from the \textit{World Development Indicators} database from the World bank. Private credit to GDP is the share of credit to the private sector to GDP and is a proxy to financial development. We expect financial development to affect leverage positively. Inflation is the annual log change in the consumer price index.   
	
		\section{Model, empirical strategy and hypothesis} \label{sec:strategy}
	In this section we present a model of optimal capital structure for multinationals based on \cite{huizinga2008capital} who consider tax and non-tax factors in the firms' decision problem. We extend their model by allowing the macroprudential tools to affect the cost of debt to firms and consequently the optimal leverage attained.  
	\subsection{Balance sheets and financial leverage}
	\label{subsec:balancesheet}
	We consider a multinational group that is composed of $n-1$ subsidiaries and the parent firm. Each subsidiary has assets $A_i$ and is financed by external debt $L_i$ and the parent firm's equity $I_i$. For each subsidiary the balance sheet is
	\begin{equation}
	\begin{aligned}
	A_i=I_i+L_i. 
	\end{aligned}
	\label{eq:sub balance sheet}
	\end{equation}
	
	For simplicity, we assume that the parent firm is the sole owner of each subsidiary. Hence, if $A_p$, $E_p$ and $L_p$ are, respectively, the assets, equity and debt of the parent firm, its balance sheet can be stated as  
	\begin{equation}
	\begin{aligned}
	A_p+\sum_{i=1}^{n-1}A_i=E_p+L_p. 
	\end{aligned}
	\label{eq:parent balance sheet}
	\end{equation}
	Financial leverage ($\lambda_i$) is defined as total liabilities to total assets, that is, $\lambda_i=L_i/A_i$. At the multinational level, total leverage is
	\begin{equation}
	\begin{aligned}
	\lambda_m=\frac{\sum_{i=1}^{n}L_i}{\sum_{i=1}^{n}A_i}=\sum_{i=1}^{n}\lambda_i\rho_i, 
	\end{aligned}
	\label{eq:total leverage}
	\end{equation} 
	where $\rho_i=A_i/\sum_{i=1}^{n}A_i$ is the asset share of firm $i$ within the multinational. The second equality in equation \ref{eq:parent balance sheet} is reached by replacing the definition of leverage for a subsidiary in the first equality of the same equation. Equation \ref{eq:parent balance sheet} shows that leverage at the multinational level can be stated as the wighted average of the subsidiaries and parent firms' leverage by their respective asset share. Adjustments in the capital structure is assumed to be done by changes in debt positions rather than assets. 
	\subsection{Costs associated with leverage}
	\label{subsec:costs}
	We assume that the debt of any subsidiary firm is implicitly or explicitly guaranteed by the parent firm. Consequently, the expected cost of bankruptcy associated with higher leverage is contemplated at the multinational level. We assume the following quadratic expected bankruptcy cost function ($C_m$):
	
	\begin{equation}
	\begin{aligned}
	C_m=\frac{\gamma}{2}\lambda_m^2\bigg(\sum_{i}^{n}A_i\bigg).
	\end{aligned}
	\label{eq:cost bankruptcy}
	\end{equation}
	
	We also consider costs at the subsidiary level which are related to incentives that leverage bring to local managers. As an example, high leverage, on the one hand, may inhibit overspending while on the other hand it can lead to unnecessarily risk-averse managers. Let $\lambda^*$ be the optimal leverage considering only the incentives to local managers. The cost of deviating $\lambda^*$ is assumed to be quadratic as follows:  
	
	\begin{equation}
	\begin{aligned}
	C_i=\frac{\mu}{2}(\lambda_i-\lambda^*)^2A_i-\frac{\mu}{2}(\lambda^*)^2A_i, \quad i=1,...,n.
	\end{aligned}
	\label{eq:agency cost}
	\end{equation}
	Lastly, we consider how macroprudential instruments affect the cost of debt to firms. Macroprudential policies are implemented to alter credit market condition which in turn may change the cost of debt at subsidiary level. We assume a competitive banking sector. The costs for a bank when lending to a firm are the opportunity cost to not lend at the discount rate in the economy ($\delta_i$) and the costs that arise from macroprudential policies, such as, requirements on reserves and capital. Those latter costs are assumed to be proportional to the prudential policies and also the discount rate. The cost of debt can be written as:
	\begin{equation}
	\begin{aligned}
	r_i=\delta_i(1+\phi'\Pi_i)
	\end{aligned}
	\label{eq:cost of debt}
	\end{equation}
	where $\phi$ is a parameter vector specific to the set of macroprudential instruments $\Pi_i$ that are implemented in the country of firm $i$. 
	\begin{equation}
	\phi'\Pi=\begin{bmatrix}
	\phi_{rr} &  \phi_{cb}
	\end{bmatrix}
	\begin{bmatrix}
	\Pi_{rr} \\   
	\Pi_{cr} 
	\end{bmatrix}
	\label{eq:phi vector}
	\end{equation}
	where $\Pi_{rr}$ and $\Pi_{cr}$ are reserve on local currency and capital requirements respectively.
	\subsection{Multinational's value}
	\label{subsec:value}
	Let $R_i$ be the gross revenue of firm $i$, which is assumed to be positive and constant. The value of firm $i$, if fully financed by equity, is the discounted cash flow considering perpetuity minus the corporate tax payments 
	\begin{equation}
	\begin{aligned}
	V_i^U=\frac{R_i}{\delta_i}(1-\tau_{i}),
	\end{aligned}
	\label{eq:v_u}
	\end{equation}	
	where $\tau_{i}$ is the effective corporate tax rate payed by firm $i$. The discount and tax rates are assumed to be invariant in time. Next, let consider firm $i$'s value with external financing $L_i$ and the costs associated with leverage (equation \ref{eq:agency cost})
	\begin{equation}
	\begin{aligned}
	V_i^L=L_i+\frac{(R_i-r_iL_i)}{\delta_i}(1-\tau_{i})-C_i,
	\end{aligned}
	\label{eq:v_l_1}
	\end{equation}	
	where the fact that interest rate payments are deductible from taxable income is accounted for. Replacing equations \ref{eq:cost of debt} and \ref{eq:v_u} on the value function \ref{eq:v_l_1} and rearranging terms:
	\begin{equation}
	\begin{aligned}
	V_i^L=V_i^U+\tau_{i}L_i-\phi'\Pi_iL_i+\tau_{i}\phi'\Pi_iL_i-C_i.
	\end{aligned}
	\label{eq:v_l_2}
	\end{equation}	
	The second term of the right hand side is the debt tax shield. The third term is the effect of macroprudential policies through the cost of debt. Abstracting from taxes, higher interest payments decrease cash flow and consequently the value of the firm. The fourth term reflects the interaction between cost of debt and tax payments. Higher interest rates decrease taxable income and, considering this effect alone, increase after tax cash flow.
	
	The value of the multinational firm $m$ is equal to the sum of equation \ref{eq:v_l_2} for each subsidiary and parent firms minus the expected cost of bankruptcy (equation \ref{eq:cost bankruptcy}) as shown in the following expression
	\begin{equation}
	\begin{aligned}
	V_m^L=V_m^U+\sum_{i=1}^{n}\tau_iL_i-\phi'\sum_{i=1}^{n}\Pi_iL_i+\phi'\sum_{i=1}^{n}\tau_i\Pi_i L_i-C_m-\sum_{i=1}^{n}C_i
	\end{aligned}
	\label{eq:v_l}
	\end{equation}
	where $V_m^L$, and $V_m^U$ are the multinational firm's values when leveraged and completely unleveraged, respectively.
	\subsection{Optimal leverage}
	\label{subsec:opt_leverage}
	The multinational firm's objective is to maximize its value by choosing each establishment's (subsidiaries and parent firm) debt level taking into account the costs and benefits associated with leverage. The problem can be stated as follows: 
	\begin{equation}
	\begin{aligned}
	\max_{L_i}V_m^U+\sum_{i=1}^{n}\tau_iL_i-\phi'\sum_{i=1}^{n}\Pi_iL_i+\phi'\sum_{i=1}^{n}\tau_i\Pi_i L_i-C_m-\sum_{i=1}^{n}C_i, \quad i=1,...,n,
	\end{aligned}
	\label{eq:problem}
	\end{equation}
	
	The first order conditions of the problem \ref{eq:problem} are:
	\begin{equation}
	\begin{aligned}
	\tau_i-\phi'\Pi_i+\phi'\Pi_i\tau_{i}-\gamma\lambda_m-\mu(\lambda_i-\lambda^*)=0, \quad i=1,...,n\\
	\end{aligned}
	\label{eq:FOC}
	\end{equation}
	which, together with equation \ref{eq:total leverage} can be written as  
	\begin{equation}
	\mu\lambda_i=\mu\lambda^*+\tau_{i}-\phi'\Pi_i+\phi'\Pi_i\tau_{i}-\gamma \sum_{j=1}^{n}\lambda_j\rho_j, \quad i=1,...,n.
	\label{eq:FOC2}
	\end{equation}
	Subtracting the first order condition for a subsidiary $j$ from the first order condition for a subsidiary $i$ we find
	\begin{equation}
	\begin{aligned}
	\lambda_j=\lambda_i-\frac{1}{\mu}(\tau_i-\tau_j)+\frac{\phi'}{\mu}(\Pi_i-\Pi_j)-\frac{\phi'}{\mu}(\Pi_i\tau_i-\Pi_j\tau_j).
	\end{aligned}
	\label{eq:joint FOC}
	\end{equation}
	Replacing equation \ref{eq:joint FOC} in \ref{eq:FOC2}, the optimal leverage level for subsidiary $i$ can be written as  
	\begin{equation}
	\begin{aligned}
	\lambda_i=&\theta_0\lambda^*+\theta_1\tau_i-\theta_2\Pi_i+\theta_2\Pi_i\tau_{i}+\theta_3\sum_{j=1}^{n}(\tau_i-\tau_j)\rho_j\\
	&-\theta_4\sum_{j=1}^{n}(\Pi_i-\Pi_j)\rho_j+\theta_4'\sum_{j=1}^{n}(\Pi_i\tau_i-\Pi_j\tau_j)\rho_j, \quad i=1,...,n
	\end{aligned}
	\label{eq:optimal leverage in theory}
	\end{equation}
	\begin{equation*}
	\begin{aligned}
	\text{where} &\\ &\theta_0=\frac{\mu}{(\mu+\gamma)}, \ \theta_1=\frac{1}{(\mu+\gamma)}, \
	\theta_2=\frac{1}{(\mu+\gamma)}\phi', \
	\theta_3=\frac{\gamma}{\mu(\mu+\gamma)}, \
	\theta_4=\frac{\gamma}{\mu(\mu+\gamma)}\phi'.
	\end{aligned}
	\end{equation*}
	\subsection{Empirical strategy}
	\label{subsec:empirics}
	Equation \ref{eq:optimal leverage in theory} provides the theoretical foundation for our empirical models. We start by the benchmark model specification in the capital structure literature\footnote{ADD the paper that mention the importance of firm FE} 
	\begin{equation}
	\begin{aligned}
	\lambda_{imsc(i),t}=&\alpha_0\tau_{c(i),t}+\beta_0\Pi_{c(i),t}+\beta_1\tau_{c(i),t}\Pi_{c(i),t}\\
	&+\alpha_1\sum_{j=1}^{n}(\tau_{c(i),t}-\tau_{c(j),t})\rho_{j,t}+\beta_2\sum_{j=1}^{n}(\Pi_{c(i),t}-\Pi_{c(j),t})\rho_{j,t}\\
	&+\Gamma_1 X_{i,t}+\Gamma_2 X_{c(i),t}+f_{i}+f_{t}+\varepsilon_{i,t},
	\label{eq:optimal leverage empirically 1}
	\end{aligned}
	\end{equation}
	
	where $i$ denotes firm (either subsidiary or parent), $m$ the multinational (indexed by the parent firm), $s$ industry and $t$ year. $c(i)$ stands for the country where firm $i$ is hosted. For example, $\tau_{c(i),t}$ represents the effective tax rate in the country where firm $i$ is hosted. The dependent variable is a financial leverage measure of the firm. $\Pi_{i,t}$ is the index for a macro-prudential instrument used in the host country. $\rho_{j,t}$ is the asset share of firm $j$ within the multinational group. $f_{i}$ and $f_{t}$ are respectively the firm and time fixed effects. $X_{i,t}$ and $X_{c(i),t}$ are the set of firm- and country-level control variables, respectively. We also include the corporate tax level at the host country $\tau_{c(i),t}$, the tax incentive to shift debt, and the interaction between tax and macroprudential policies. Note, that we abstract from the last term of equation \ref{eq:optimal leverage in theory}. 
	
	However, if our intention is to find causal relationship between macroprudential policies and capital structure, model \ref{eq:optimal leverage empirically 1} might not be robust enough.
	 Macroprudential instruments can be correlated to other factors at the country level that could impact the supply of credit to firms. The inclusion of country-level variables helps to mitigate those identification concerns but it might not be enough. In order to address this, we rely once more on the variation within multinational groups and perform the following model
	\begin{equation}
	\begin{aligned}
	\lambda_{imsc(i),t}=&\alpha_1\sum_{j=1}^{n}(\tau_{c(i),t}-\tau_{c(j),t})\rho_{j,t}+\beta_2\sum_{j=1}^{n}(\Pi_{c(i),t}-\Pi_{c(j),t})\rho_{j,t}\\
	&+\beta_4Risk_{i}\Pi_{c(i),t}+\Gamma_1 X_{i,t}+f_{c(i),t}+f_{m}+\varepsilon_{i,t}.
	\label{eq:optimal leverage empirically 2}
	\end{aligned}
	\end{equation}
	In this model we include country*year ($f_{c(i),t}$) to account for the other unobservable characteristics that could be driving supply rather than macroprudential policy. All country-time specific variables cannot be included in this specification. In order to study the domestic effect we interact both macroprudential indexes with our proxy of firms' riskiness, their volatility of EBITDA ($Risk_{i}$) in the sample period. We keep the multinatinal $f_{m}$ fixed effects and firm-level control variables.  
	
	Still, a complete causal relationship cannot be inferred by performing regression \ref{eq:optimal leverage empirically 2}. On the basis of demand factors alone, the optimal leverage level does not have to be constant in time. Consequently, firm fixed effects might not account for all the unobservable characteristics that could be driving demand. The inclusion of year fixed effects is not sufficient since it does not capture cross-firm unobservable variability.  
	
	According to the model developed in this section, the capital structure decision is taken at the multinational level. Hence, we can push further our identification strategy towards separating demand from supply effects on leverage. By tracking multinationals composed of firms hosted at different countries exposed to cross-country and time varying macroprudential policies we can control for unobservables characteristics of every multinational group at each year as follows   
	
	\begin{equation}
	\begin{aligned}
	\lambda_{imsc(i),t}=&+\beta_4Risk_{i}\Pi_{c(i),t}+\Gamma_1 X_{i,t}+f_{m,t}+f_{c(i),t}+\varepsilon_{i,t}.
	\label{eq:optimal leverage empirically 3}
	\end{aligned}
	\end{equation}
	
	The model includes multinational*year ($f_{m,t}$) and country*year ($f_{c(i),t}$) fixed effect, besides the interaction of macroprudential indexes with firms' risk and firm-level controls. The drawbacks of this specification are that we have to drop the international effect variables and the sample is restricted to firms that belong to a multinational group. 		
	
		\subsection{Hypothesis}
	\label{subsec:hypothesis}
	We restrict our analysis to the following four macroprudential tools: reserve requirement on local currency, LTV ratio limits, capital requirements and capital buffers related to real estate loans. Among those instruments, reserve and capital requirements are expected to be positively correlated to banks' cost of providing loans ($\phi_{rr}>0$ and $\phi_{cr}>0$) as they need to hold more capital and reserves for the same amount of loans. 
	
	Given expression \ref{eq:phi vector} and equation 	\ref{eq:optimal leverage in theory} our hypothesis on the parameters of models \ref{eq:optimal leverage empirically 1}, \ref{eq:optimal leverage empirically 2} and \ref{eq:optimal leverage empirically 3} can be stated as follows

	\begin{hyp}[H\ref{hyp:H1}] \label{hyp:H1}
		holding everything else constant and considering only the domestic effect through the cost of debt ($\beta_{0}$), a tightening/loosening in reserve and capital requirements reduces/increases firms' leverage ($\beta_{0,rr}<0$, $\beta_{0,cr}<0$).
	\end{hyp}
	\begin{hyp}[H\ref{hyp:H2}] \label{hyp:H2}
		holding everything else constant the riskiness of the firm and the incentive to shift debt within a multinational reinforce the effects described in hypothesis \ref{hyp:H1}. 
	\end{hyp}
Note that the equation \ref{eq:optimal leverage in theory} implies $\beta_{1}=-\beta_{0}$, that is, the effect of macroprudential policies through higher tax shields will have the same magnitude but opposite sign to the direct effect via higher interest payments. From hypothesis \ref{hyp:H1} we expect $\beta_{1}>0$.
	 
	\section{Results} \label{sec:result}
	 Table \ref{tab:reg1} present the models developed in section \ref{subsec:empirics}. The dependent variable is financial leverage. The first column reports the results when estimating equation \ref{eq:optimal leverage empirically 1}. Besides the set of variables being tested by our hypothesis this model includes firm-, country-level controls, (except for the volatility of profits which is time invariant), firm and year fixed effects. The first four lines are related to hypothesis \ref{hyp:H1}. The signs are as expected but only the effect of capital requirement are statistically significant at 1\%. Given the way we construct our variables, a tightening in capital requirements is associated with leverage being 11.2\% lower on average. Next, the interaction terms between corporate tax and macroprudential policies have also the expected sign, and when significant, the intensity. Again, only the interaction with capital requirements is statistically significant, but now at 5\% level. 
	 
	 	\begin{small}
	 	{\setstretch{1.0}
	 		\begin{longtable}{lcccc}\\
	\label{reg:benchmark}\\
	\multicolumn{5}{c}{Table \ref{reg:benchmark} - Regression results, macroprudential policies and capital structure }\\
	\multicolumn{5}{c}{(\textit{Dependent variable}: financial leverage)(\textit{Base year}: 2007=1)}
	\\ \hline \hline
	Model & (1) & (2) & (3) & (4)  \\ \hline
	&  &  &  \\ \endfirsthead
	\multicolumn{5}{c}{Table \ref{reg:benchmark} - Regression results, macroprudential policies and capital structure }\\
	\multicolumn{5}{c}{(\textit{Dependent variable}: financial leverage)(\textit{Base year}: 2007=1)\textit{(Continued)}}
	\\ \hline \hline
	Model & (1) & (2) & (3) & (4)  \\ \hline 
	&  &  &  & \\ \endhead
	\hline
	\multicolumn{5}{r}{{\textit{(Continued)}}}\\ \endfoot	
	\endlastfoot
	Effects of macroprudential policies  &  &  &  \\
\quad Capital requirement & -0.165*** & -0.149*** &  &  \\
 & (0.051) & (0.051) &  &  \\
\quad Reserve requirement & -0.069** & -0.053 &  &  \\
 & (0.035) & (0.035) &  &  \\
\quad Capital requirement*tax & 0.196*** & 0.197*** &  &  \\
 & (0.050) & (0.050) &  &  \\
\quad Reserve requirement*tax & 0.080** & 0.081** &  &  \\
 & (0.033) & (0.033) &  &  \\
\quad Capital requirement spillover & -0.033** & -0.032** & -0.018 &  \\
 & (0.015) & (0.015) & (0.015) &  \\
\quad Reserve requirement spillover & -0.022** & -0.021** & -0.017* &  \\
 & (0.009) & (0.009) & (0.009) &  \\
\quad Capital requirement*risk &  & -0.190*** & -0.177** & -0.264*** \\
 &  & (0.071) & (0.069) & (0.077) \\
\quad Reserve requirement*risk &  & -0.165* & -0.152* & -0.176* \\
 &  & (0.092) & (0.090) & (0.099) \\
    Effects of corporate tax &  &  &  \\
\quad Tax rate & 0.046 & 0.043 &  &  \\
 & (0.051) & (0.051) &  &  \\
\quad Tax rate spillover & 0.225*** & 0.224*** & 0.070 &  \\
 & (0.067) & (0.067) & (0.064) &  \\
     Firm characteristics &  &  &  \\
\quad Risk & 0.073* & 0.084** & 0.090** & 0.118*** \\
 & (0.041) & (0.040) & (0.040) & (0.040) \\
\quad Tangibility & -0.005 & -0.005 & -0.005 & -0.005 \\
 & (0.004) & (0.004) & (0.004) & (0.004) \\
\quad Log of fixed assets & 0.973*** & 0.971*** & 0.969*** & 1.024*** \\
 & (0.101) & (0.101) & (0.101) & (0.110) \\
\quad Profitability & -0.006*** & -0.006*** & -0.006*** & -0.007*** \\
 & (0.002) & (0.002) & (0.002) & (0.002) \\
\quad Opportunity & 0.097*** & 0.098*** & 0.102*** & 0.135*** \\
 & (0.021) & (0.021) & (0.022) & (0.030) \\
      Macroeconomic controls &  &  &  \\
\quad Inflation rate & 0.319 & 0.293 &  &  \\
 & (0.248) & (0.245) &  &  \\
\quad GDP growth rate & 0.224* & 0.230** &  &  \\
 & (0.117) & (0.117) &  &  \\
\quad Private credit to GDP & 0.049 & 0.047 &  &  \\
 & (0.033) & (0.033) &  &  \\
\quad Political Risk & -0.167* & -0.161* &  &  \\
 & (0.092) & (0.092) &  &  \\
\quad Exchange rate risk & -0.085*** & -0.087*** &  &  \\
 & (0.020) & (0.020) &  &  \\
\quad Law and order & 0.183 & 0.166 &  &  \\
 & (0.232) & (0.231) &  &  \\
 &  &  &  &  \\
Observations & 367,003 & 367,003 & 367,014 & 367,014 \\
Number of multinationals & 23,194 & 23,194 & 23,194 & 23,194 \\
Year fixed effects & Yes & Yes & No & No \\
Multinational fixed effects & Yes & Yes & Yes & No \\
Country*year fixed effects & No & No & Yes & Yes \\
 Multinational*year fixed effects & No & No & No & Yes \\ \hline
\multicolumn{5}{c}{ Robust standard errors in parentheses} \\
\multicolumn{5}{c}{ *** p$<$0.01, ** p$<$0.05, * p$<$0.1} \\
\end{longtable}
	 	}
	 \end{small}
 
	 Considering the average tax level relative to 2007 in our sample (XX) the domestic effect of capital requirements is estimated to be -11.2\%+XX*10.9\%=YY\% on average. Evidence supporting hypothesis \ref{hyp:H3} is found only for reserve requirement on local currency at 1\% significance level. To illustrate the effect, we consider a multinational firm that consists of the parent firm and one subsidiary with assets of equal size. A tightening on the reserve requirements on local currency in the country of the subsidiary is associated with a spillover to the parent firm that reduces financial leverage by 3.05\% (the estimated coefficient, -0.061, times the  asset share of the parent firm, 0.5). Overall, the results suggest that firms respond locally to changes in capital requirements while multinationals have an extra incentive to shift debt due to moves in reserve requirements on local currency.
	 
	 Both the corporate tax rate and the tax rate spillover are statistically insignificant different from zero. But considering the average level of capital requirements relative to 2007 in our sample (XX), an increase in the effective tax rate by XX\% (one standard deviation) is associated, on average, with XX*10.9\%=YY\% higher leverage. Among the control variables, tangibility and profitability have significant negative coefficients at 1\%. The substitution effect seems to dominate on average in our sample for both variables. Next, as expected, log of fixed assets and opportunity have positive and statistically significant at 1\% estimated coefficients. Inflation, GDP growth rate and private credit to GDP enters the regression with positive sign, suggesting that leverage is pro-cyclical. These results also point to the importance of accounting for country-year specific characteristics when explaining leverage. Moreover, none of the risk measures or policy rate are statistically significant. 
	 
	 As highlighted in section \ref{subsec:empirics}, the specification in column 1 misses both cross-firm and time variation that could be associated to macroprudential policies and driving the results (for example, the bank sector size and different responses to the global crisis). To this purpose, we estimate model \ref{eq:optimal leverage empirically 2} which account for unobservable at the multinational level for every year in our sample. We also add industry fixed effect to account for sector specific differences within multinationals' subsidiaries that could affect capital structure decisions. 
	 
	 The results are shown in column 2 of table \ref{tab:reg1}. Note that with this specification, by construction, the parameters related to international spillover cannot be included in the model. Once multinational*year fixed effects are included, the spillover coefficients only capture the domestic effect. Accounting for unobservables at the multinational level for every year changes considerably the results. First, the magnitude and significant of capital requirements domestic effect are greater than the model with firm and year fixed effects. 
	 
	 Again, considering the average tax level relative to 2007 in our sample (XX) the domestic effect of capital requirements is now estimated to be -54.4\%+XX*59.6\%=YY\% on average. Second, tax rate appears positively associated with leverage, as expected. Raising the effective tax rate by XX\% (one standard deviation) is correlated to XX*19.7\%+XX*cap*59.6\%=YY\% higher leverage, on average and considering capital requirements at cap\%. Firm-level controls remain statistically significant and with the expected signs while most of country level controls become statistically insignificant with the exception of exchange rate risk which in this model is significant at 1\% level. 
	 


We still cannot infer causality from the results in column 2 of table \ref{tab:reg1} since we cannot disentangle macroprudential policies variation to other common shocks to firms hosted in the same country.  In column 3 we include country*year fixed effects to account for any variation at the country-level. Consequently, in this specification we have to drop the macroprudential indexes and their interaction with tax rates. In order to account for the domestic effect we introduce an interaction term between each of the macroprudential indexes and our proxy of firms' riskiness. So far we have assumed the cost of debt to be homogeneous across firms hosted in the same country. However, we should expect the interest rate charged on loans to be higher for riskier firms. Those terms should still    We replace the multinationals time specific to time invariant fixed effects allowing us to add the international spillover variables. In order to   
%	\section{Robustness tests and extensions} \label{sec:robustness}
%	\subsection{Alternative leverage measures} %\label{subsec:alt_leverage}
	
 Tangibility appears with a positive coefficient, suggesting that the effect of fixed assets as collateral dominates the substitution effect due to depreciation. Inflation is negatively associated to long-term debt. As \cite{huizinga2008capital} points out, in an inflationary environments the uncertainty about ex-post real interest rate may inhibit leverage on long-term liabilities. In all specifications, tax rate coefficients are significantly at 1\% and has a negative sign. Surprisingly, the tax incentive to shift debt is statistically significantly at 5\% and positive for nine out of 12 specifications.
	
Tangibility regained the negative sign, suggesting that the substitution effect coming from the depreciation tax shield matters more when deciding to finance through short-term loans. 

%\subsection{Alternative models} \label{subsec:alt_models}

	\section{Conclusion} \label{sec:conclusion}
	
	
	
	\singlespacing
	\bibliography{C:/Users/User/work/master_thesis/analysis/code/text/bibliography/references}
	\bibliographystyle{C:/Users/User/work/master_thesis/analysis/code/text/bibliography/te}
	
	
	
%	\clearpage
	
%	\onehalfspacing
	
%	\section*{Tables} \label{sec:tab}
%	\addcontentsline{toc}{section}{Tables}
	
		%\input{C:/Users/User/work/master_thesis/analysis/temp/summary_MPI}
	
		%\input{C:/Users/User/work/master_thesis/analysis/temp/summary_MPI_year}
		
		%\input{C:/Users/User/work/master_thesis/analysis/temp/Tex/number_firms_orbis}
		
		%\input{C:/Users/User/work/master_thesis/analysis/temp/Tex/summary_firm_orbis_clean}
		
		%\input{C:/Users/User/work/master_thesis/analysis/temp/Tex/summary_country_orbis}
	
	%	\begin{table}
	%		\centering
	%		\caption{Number of firms that changed ownership}
	%		\label{tab:movers}
	%	%            &\multicolumn{1}{c}{(1)}\\
            &\multicolumn{1}{c}{Movers}\\
            &           b\\
\hline
2           &       27990\\
3           &        7544\\
4           &        1783\\
5           &         314\\
6           &          50\\
7           &           1\\
Total       &       37682\\

	%\end{table}

		
	%\section*{Figures} \label{sec:fig}
	%\addcontentsline{toc}{section}{Figures}
	
	%\begin{figure}[hp]
	%  \centering
	%  \includegraphics[width=.6\textwidth]{../fig/placeholder.pdf}
	%  \caption{Placeholder}
	%  \label{fig:placeholder}
	%\end{figure}
	
	
%	\clearpage
	
%	\section*{Appendix A.} 
%	\label{sec:appendixa}
%	\addcontentsline{toc}{section}{Appendix A}
	%% Please add the following required packages to your document preamble:
% \usepackage{graphicx}
%\begin{longtable}[]
	\centering
	%\resizebox{\textwidth}{!}{%
		\begin{longtable}{p{2.5in}p{4in}p{2in}}
				\label{tab:definition}\\
			\multicolumn{3}{c}{Table \ref{tab:definition} - Variable definitions and data sources}\\
			\hline 
			Variable      & Definition & Source \\
			\hline \endfirsthead
			
				\multicolumn{3}{c}{Table \ref{tab:definition} - Variable definitions and data sources \textit{(Continued)}}\\
			\hline 
			Variable      & Definition & Source \\
			\hline \endhead
			
			\hline
			\multicolumn{3}{r}{{\textit{Continued}}}\\ 
			\endfoot
			\hline
			\endlastfoot
			Financial leverage      & Ratio of non-equity liabilities to total assets & Orbis \\
		
		Restriction on banking activities & Index for the extent to which banks may engage in securities, insurance and real estate activities(higher values indicate more restrictive) & \cite{barth2013bank}\\
			
			Financial conglomerates restrictiveness & Index for restrictions on banks' ownership of nonfinancial firms and on non-bank firms owning banks (higher values indicate more restrictive) & \cite{barth2013bank}\\	
			
			Capital regulatory stringency & Index measuring stringency in capital requirements (higher values indicate higher stringency) & \cite{barth2013bank}\\
			
			Official supervisory power & Index capturing whether the supervisory authorities have the authority to take specific actions to prevent and correct problems (higher values indicate greater power) & \cite{barth2013bank}\\	
			
				External governance & Index for the quality and presence of certain accounting practices, financial statement transparency, external rating and auditing  (Higher values indicate better corporate governance) & \cite{barth2013bank}\\
			
			Restriction on banking activities spillover & Sum of restriction on banking activities index differences weighted by local asset shares & \cite{barth2013bank}\\
			
			Financial conglomerates restrictiveness spillover & Sum of Financial conglomerates restrictiveness index  differences weighted by local asset shares & \cite{barth2013bank}\\	
			
			Capital regulatory stringency spillover & Sum of capital regulatory stringency index differences weighted by local asset shares & \cite{barth2013bank}\\
			
			Official supervisory power spillover & Sum of official supervisory power index differences weighted by local asset shares & \cite{barth2013bank}\\	
			
			External governance spillover & Sum of external governance index differences weighted by local asset shares & \cite{barth2013bank}\\
				
			Tax rate & Taxes and other mandatory contributions after accounting  for deductions and exemptions to total commercial profit & World Bank Doing Business indicators\\
			Tax rate spillover & Sum of international corporate tax rate  differences weighted by local asset shares& World Bank Doing Business indicators\\
			Tangibility& Ratio of fixed assets to total assets & Orbis\\
			Log of fixed assets& logarithm of fixed assets & Orbis\\
			Profitability& Ratio of EBITDA to total assets & Orbis\\
			Risk & Standard deviation of the firm's ratio of EBITDA to total assets over the period 2008-2014& Orbis\\
			Opportunity & Median of the annual growth rate of sales per country and industry& Orbis\\
			Private credit to GDP & Ratio of credit to the private sector to GDP & World Bank indicators\\	
			Inflation & Annual log change in the CPI & World Bank indicators\\
			GDP growth rate &Annual percentage change in the GDP & World Bank indicators\\
			 Policy rate &Central Bank Policy Rate, Percent per annum & IMF International Financial Statistics \\
			Exchange rate risk &Annual (December) index of exchange rate risk &International Country Risk Guide\\
			Law and order &Annual (December) index of law and order &International Country Risk Guide\\
			Political risk &Annual (December) index of political risk &International Country Risk Guide\\
		\end{longtable}%
	%}
%\end{longtable}

\end{document} 