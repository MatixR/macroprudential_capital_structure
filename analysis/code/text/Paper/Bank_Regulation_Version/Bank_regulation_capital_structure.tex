\documentclass[12pt]{article}
\usepackage[utf8]{inputenc}
\usepackage{amssymb,amsmath,amsfonts,eurosym,geometry,ulem,caption,color,xcolor,multicol,setspace,sectsty,comment,footmisc,caption,natbib,pdflscape,subfigure,array,hyperref,verbatim,mathpazo,longtable,ntheorem,siunitx,booktabs,threeparttable,afterpage,mathtools}
%\usepackage[normalem]{ulem}
\usepackage[pdftex]{graphicx}
\graphicspath{{Imagens/}}
\usepackage{fullpage}
\usepackage{indentfirst}
\usepackage{changepage}

%\setlength{\pdfpagewidth}{8.5in} \setlength{\pdfpageheight}{11in}
%\setlength{\textheight}{8.5in} \setlength{\topmargin}{0.0in}
%\setlength{\headheight}{0.0in} \setlength{\headsep}{0.0in}
%\setlength{\leftmargin}{0.5in}
%\setlength{\oddsidemargin}{0.0in}
%\setlength{\parindent}{2em}
%\setlength{\parskip}{\baselineskip}%
%\setlength{\textwidth}{6.5in}
%\linespread{1.6}

\newcommand{\sym}[1]{\rlap{#1}}% Thanks to David Carlisle

\newcommand*{\captionsource}[2]{%
  \caption[{#1}]{%
    #1%
    \\\hspace{\linewidth}%
    \textbf{Source:} #2%
  }%
}
\newcommand{\horrule}[1]{\rule{\linewidth}{#1}} % Create horizontal rule command with 1 argument of height

%***************************************************
% Fix input with \midrule problem
%***************************************************

\makeatletter
\newcommand\primitiveinput[1]
{\@@input #1 }
\makeatother


\onehalfspacing
\newtheorem{theorem}{Theorem}
\newtheorem{corollary}[theorem]{Corollary}
\newtheorem{proposition}{Proposition}
\newenvironment{proof}[1][Proof]{\noindent\textbf{#1.} }{\ \rule{0.5em}{0.5em}}

\newtheorem{hyp}{Hypothesis}
\newtheorem{subhyp}{Hypothesis}[hyp]
\renewcommand{\thesubhyp}{\thehyp\alph{subhyp}}

\newcommand{\red}[1]{{\color{red} #1}}
\newcommand{\blue}[1]{{\color{blue} #1}}

\newcolumntype{L}[1]{>{\raggedright\let\newline\\arraybackslash\hspace{0pt}}m{#1}}
\newcolumntype{C}[1]{>{\centering\let\newline\\arraybackslash\hspace{0pt}}m{#1}}
\newcolumntype{R}[1]{>{\raggedleft\let\newline\\arraybackslash\hspace{0pt}}m{#1}}

\geometry{left=1.2in,right=1.2in,top=1.2in,bottom=1.2in}
\begin{document}
	
	\begin{titlepage}
		\title{The Impact of Bank Regulation on  Firms' Capital Structure: Evidence from Multinationals}
		
		\author{Lucas Avezum \and Harry Huizinga \and Louis Raes\thanks{Lucas Avezum: CentER, Tilburg University PO Box 90153 5000 LE Tilburg. Email:
				l.avezum@uvt.nl (corresponding author). Harry Huizinga: CentER, Tilburg University, CEPR. Louis Raes: CentER and EBC, Tilburg University.}}
		\date{\today}
		\maketitle
		\begin{abstract}
			\noindent This paper studies how bank regulatory policies relate to non-financial firms' capital structure. We first present a model where regulation affects banks' funding costs which in turn are transmitted to firms through their cost of debt. The solution of the model is a simple equation predicting firms' capital structure to be sensitive to bank regulatory policies and their interaction with tax rates. In the absence of taxes, bank regulation affects firms' financial leverage by changing their cost of debt and ultimately their value. However, the interaction of bank regulatory policies and tax rates has the opposite effect due to gains or losses coming from the interest tax shield. We empirically identify those effects by comparing firms within multinational corporations hosted in several countries and consequently exposed to bank regulation at different levels. Our results show that tighter capital stringency and greater official supervisory power are negatively associated with firms' financial leverage while restrictions on banking activities have a positive effect.   \\
			\vspace{0in}\\
			\noindent\textbf{Keywords:} capital structure, bank regulation, multinational corporation\\
			\vspace{0in}\\
			\noindent\textbf{JEL Codes:} G28, G32, F23 \\
			
			\bigskip
		\end{abstract}
		\setcounter{page}{0}
		\thispagestyle{empty}
	\end{titlepage}
	\pagebreak \newpage
	
	
	\normalem
	
	\doublespacing
	
	
	\section{Introduction} \label{sec:introduction}
	The mandate of central banks has expanded considerably. Triggered by the 2008 financial crisis, financial stability is now perceived by many central banks as a goal on top of price stability \citep*{blinder2017necessity}. To achieve the former, several macroprudential instruments were added to the toolboxes of central bankers and other supervisory authorities, reshaping the bank regulatory framework across the globe. \cite*{barth2013bank} show that several country-level measures for bank regulation and supervision have converged between 1999 and 2011. For example, requirements to open a bank and regulation on bank capital have converged in the sense that there is less variation across countries in the last survey than there is in the first. Nevertheless, the authors point that bank regulatory and supervisory policies were still diverse across countries in 2011.
	
	The effects on the bank sector of regulatory policies have been extensively studied (\cite*{barth2013}; \cite*{anginer2014does};  \cite*{caprio2014macro};\cite*{demirguc2013bank}). However, there is little evidence of their impact on non-financial sectors of the economy. In this paper we study how bank regulation and supervision influence non-financial firms' capital structure. We focus on multinational groups for our empirical identification strategy. We first present a model of the multinational corporation's optimal capital structure. This model considers the trade-off between tax advantages of debt versus the bankruptcy and agency costs related to it. On top of this standard capital structure framework, we add the transmission of bank regulatory policies to firms' cost of debt. Regulation affects banks' funding and operational costs which may be transmitted to firms via changes in lending conditions. Our theoretical model provides a simple equation in which we can test the impact of regulatory policies on firms' financial leverage.
	
	Empirical evidence on the impact of bank regulatory policies on firms' capital structure is provided for a sample of 377,999 firms hosted in 54 countries and belonging to 56,702 multinational corporate groups. Our identification strategy relies on two aspects of the data. First, firm-level data mitigate endogeneity concerns as the leverage level of non-financial firms is not likely to be considered by supervisory authorities when deciding on bank regulation. Second, we mitigate concerns with omitted variables by exploring variation of bank regulation \textit{across establishments} of multinationals at \textit{each year}. Our dataset allow us to track multinational groups with ownership information. We include multinational-year fixed effects, effectively controlling for unobservable characteristics that drive demand for debt and could be correlated to bank regulation.
	
	The results show that one standard deviation higher restrictiveness on banking activities is associated with 0.75 percentage points higher financial leverage. Additionally, one standard deviation tighter capital stringency and greater official supervisory power raise firms' financial leverage by 0.42 and 0.55 percentage points, respectively. The results are economically meaningful. Those variables combined explain 5\% of financial leverage in our sample. We find no evidence for an effect of restrictions on financial conglomerates.
	
	Research on the effects of bank regulation, such as, prudential policies has increased considerably after the 2008 financial crisis. Several authors rely on cross-country aggregate data to study the relationship between macroprudential policies and financial indicators. \cite*{cerutti2017use} introduce a dataset containing the usage of 12 macroprudential policies in119 countries from 2000 to 2013. Among their findings, borrower-related policies, such as limits on loan-to-value and debt-to-income ratios are associated with reductions in credit growth and house prices. As mentioned, most of the literature focus on the effects of regulation and supervision on the bank and financial sectors. Two recent exceptions are \cite*{ayyagari2017credit} who use firm-level data to relate macroprudential policies across 119 countries to firms' debt growth and \cite*{epure2017household} who study the impact of macroprudential tools on household credit in Romania with loan-level data.
	
	Finally, this paper also builds on the capital structure literature. Many papers study the determinants of firms' leverage, both theoretical and empirically. \cite*{titman1988determinants} and more recently, \cite*{oztekin2015capital} provide an overview of the different proposed capital structure's determinants. Besides highlighting bank regulation as another determinant of capital structure we are the first, to our knowledge, to use variation across firms belonging to a multinational group as an identification strategy to control for credit demand factors.
	
	The paper proceeds as follows: section \ref{sec:model} presents the model and section \ref{sec:strategy} explains the empirical strategy. The data is described in section \ref{sec:data} while the results are discussed in section \ref{sec:result}. Section \ref{sec:conclusion} concludes.
	
		\section{Model} \label{sec:model}
	In this section we present a model of the optimal capital structure of multinationals based on \cite{huizinga2008capital} who consider tax and non-tax factors in the firms' decision problem. We extend their model by allowing bank regulation to affect the cost of debt to firms and consequently their optimal leverage level.
	
\subsection{Balance sheets and financial leverage}
	\label{subsec:balancesheet}
	We consider a multinational group operating in $n$ countries composed of $n-1$ subsidiaries and the parent firm \footnote{In the model we consider one establishment per country, so subscripts represent both firms and their host country. This is not the case in our dataset, since multinationals may own several companies in one country. This simplification in our model does not change the results.}. Each subsidiary has assets $A_i$ and is financed by external debt $L_i$ and the parent firm's equity $I_i$. The balance sheet of a subsidiary $i$ is $A_i=I_i+L_i$. We assume that the parent firm is the sole owner of each subsidiary. Hence, if $A_p$, $E_p$ and $L_p$ are, respectively, the assets, equity and debt of the parent firm, its balance sheet can be stated as $A_p+\sum_{i=1}^{n-1}I_i=E_p+L_p$.

	Financial leverage for each establishment of the multinational ($\lambda_i$) is defined as total non-equity liabilities to total assets, that is,  $\lambda_i=L_i/A_i$. Let $\lambda_m$ be the financial leverage for the entire multinational corporation, i.e., $\lambda_m=\sum_{i=1}^{n}L_i/\sum_{i=1}^{n}A_i$. Using $\lambda_i$ for each firm $i$ we can write $\lambda_m$ as the weighted average of establishments' financial leverage by their respective asset share, i.e., $\sum_{i=1}^{n}\lambda_i\rho_i$, where $\rho_i=A_i/\sum_{i=1}^{n}A_i$ is the asset share of firm $i$ within the multinational. Adjustments in the capital structure are assumed to be the result of changes in liabilities rather than assets, that is, assets are taken as given to firms.
	
	\subsection{Costs associated with leverage}
	\label{subsec:costs}
	We assume that the debt of any subsidiary firm is implicitly or explicitly guaranteed by the parent firm. Consequently, the expected cost of bankruptcy associated with higher leverage is  a function of the aggregate leverage of the multinational. We assume a quadratic expected bankruptcy cost function ($C_m$) as follows:
	\begin{equation}
	\begin{aligned}
	C_m=\frac{\gamma}{2}\lambda_m^2\bigg(\sum_{i=1}^{n}A_i\bigg).
	\end{aligned}
	\label{eq:cost bankruptcy}
	\end{equation}
	
	We also consider costs at the subsidiary level which are related to incentives that leverage brings to local managers. For example, high leverage may inhibit overspending but it can also lead to unnecessarily risk-averse managers. Let $\lambda^*$ be the optimal leverage ratio when considering only the incentives to local managers. The cost of deviating from $\lambda^*$ is assumed to be quadratic and proportional to the amount of assets at establishment $i$ as follows:
	\begin{equation}
	\begin{aligned}
	C_i=\frac{\mu}{2}(\lambda_i-\lambda^*)^2A_i-\frac{\mu}{2}(\lambda^*)^2A_i, \quad i=1,...,n.
	\end{aligned}
	\label{eq:agency cost}
	\end{equation}	
		
	Lastly, each establishment has a cost $r_i$ associated with the repayment of the debt. We assume a competitive bank sector in each country implying that banks charge firms their cost of funding. Banks' cost of funding is a function of the discount rate ($\delta_i$)\footnote{Think of the discount rate representing both the cost of opportunity of not lending to the government and the cost of deposits.}  and the set of regulatory instruments ($\Pi_i$) in the host country. The cost of debt for establishment $i$ is assumed to be $r_i=\delta_i(1+\phi'\Pi_i)$. The parameter vector $\phi$ contains the effects of changes in regulatory policies on the cost of funding. Note that, $\phi$ can take positive and negative values, depending on the supervisory measure.
	
	\subsection{Optimal leverage}
	\label{subsec:opt_leverage}
	Let $V_m^L$, and $V_m^U$ be the multinational's values when leveraged and completely unleveraged, respectively, and $\tau_{i}$ the effective corporate tax rate charged to firm $i$. The multinational corporation's objective is to maximize its value by choosing each establishment's debt level taking into account the costs and benefits associated with leverage according to the following equation\footnote{This expression is derived by assuming a constant and positive cash flow, perpetual debt and the fact that interest payments are tax deductible.}:
	\begin{equation}
	\begin{aligned}
	V_m^L=V_m^U+\sum_{i=1}^{n}\tau_iL_i-\phi'\sum_{i=1}^{n}\Pi_iL_i+\phi'\sum_{i=1}^{n}\tau_i\Pi_i L_i-C_m-\sum_{i=1}^{n}C_i.
	\end{aligned}
	\label{eq:v_l}
	\end{equation}
    The first order conditions with respect to debt are:
	\begin{equation}
	\begin{aligned}
	\tau_i-\phi'\Pi_i+\phi'\Pi_i\tau_{i}-\gamma\lambda_m-\mu(\lambda_i-\lambda^*)=0, \quad i=1,...,n,\\
	\end{aligned}
	\label{eq:FOC}
	\end{equation}
	which can be jointly used to derive the optimal leverage ratio for each establishment as follows:
	\begin{equation}
	\begin{aligned}
	\lambda_i=&\theta_0\lambda^*+\theta_1\tau_i-\theta_2\Pi_i+\theta_3\sum_{j=1}^{n}(\tau_i-\tau_j)\rho_j-\theta_4\sum_{j=1}^{n}(\Pi_i-\Pi_j)\rho_j\\
	&+\theta_2\Pi_i\tau_{i}+\theta_4\sum_{j=1}^{n}(\Pi_i\tau_i-\Pi_j\tau_j)\rho_j, \quad i=1,...,n
	\end{aligned}
	\label{eq:optimal leverage in theory}
	\end{equation}
	where $ \theta_0=\mu/(\mu+\gamma)$, $\theta_1=1/(\mu+\gamma)$,
	$\theta_2=\phi'/(\mu+\gamma)$,
	$\theta_3=\gamma/\mu(\mu+\gamma)$,
	$\theta_4=\gamma\phi'/\mu(\mu+\gamma)$.
	
    Equation \ref{eq:optimal leverage in theory} shows that the optimal leverage ratio at all establishments of the multinational depends on the incentives to local managers ($\lambda^*$) balanced by the cost of bankruptcy ($\theta_0$). The second term of the right hand side, $\theta_1\tau_i$, is the effect of debt tax shield. The third term, $\theta_2\Pi_i$, is the effect of bank regulation through the cost of debt. As an example, consider a regulatory measure that increases the cost of debt, that is, $\phi>0$. Abstracting from taxes, higher interest payments decrease cash flow and consequently the value of the firm. In this case, a more stringent regulation should be negatively associated to leverage. The fourth term reflects the interaction between the cost of debt and tax payments. Consider again $\phi>0$, higher interest rates decrease taxable income and, considering this effect alone, increase after-tax cash flow. Therefore, we should expect the sign of the interaction term to be opposite to the main effect of bank regulation. Lastly, the remaining terms reflect the extra incentive to shift debt that multinationals have, by taking advantage of differences in tax rate and bank regulation across host countries. In the next section we build our empirical strategy based on equation \ref{eq:optimal leverage in theory}.

    	\section{Empirical strategy}
    \label{sec:strategy}

    We are interested in evaluating the impact of bank regulation on firms' capital structure. However, a simple regression of leverage on bank regulation might suffer from omitted variable bias. Firms' unobservable characteristics could be correlated with regulatory instruments. The inclusion of firm fixed effects is not enough since credit demand factors are most likely not constant in time. We mitigate concerns with omitted variables by exploring variation of bank regulation \textit{across establishments} of multinationals at \textit{each year}. Our dataset allows us to include multinational-year fixed effects, effectively controlling for unobservable characteristics that drive the demand for debt\footnote{Our reasoning depends on the implicit assumption that decisions regarding leverage are taken at the parent level. Since we only consider parent firms that have controlling power over its subsidiaries (owns at least 50\% plus one shares of the subsidiary) this assumption seems valid.}.

    The theoretical expression \ref{eq:optimal leverage in theory} together with multinational-year fixed effects is reduced to the following regression equation:
    \begin{equation}
    \begin{aligned}
    \lambda_{imc(i)t}=&\beta_0\tau_{c(i)t}+\beta_1\Pi_{c(i)t}+\beta_2\Pi_{c(i)t}\tau_{c(i)t}+\gamma X_{it}+\alpha_{m,t}+\alpha_{c(i)}+\varepsilon_{it},
    \label{eq:regression benchmark}
    \end{aligned}
    \end{equation}

    where $i$ denotes a firm (either subsidiary or parent), $m$ the multinational (indexed by the parent firm) and $t$ the year. The subscript $c(i)$ stands for the country where firm $i$ is hosted. For example, $\tau_{c(i)t}$ represents the effective tax rate in the country where firm $i$ is hosted at year $t$. The dependent variable ($\lambda_{imc(i)t}$) is the financial leverage measure of the firm $i$, belonging to multinational $m$, hosted in country $c(i)$ at year $t$. The vector of bank regulatory instruments implemented in the host country is represented by $\Pi_{c(i)t}$. The model includes multinational-year fixed effects ($\alpha_{mt}$). We also add country fixed effects ($\alpha_c(i)$) to account for unobservable country characteristics. The set of firm-level control variables is $X_{i,t}$.

    By construction, we do not need to estimate the incentives to shift debt across establishments from equation \ref{eq:regression benchmark}. The time-varying multinational fixed effects account for their variation. To see this, consider the term $\sum_{j=1}^{n}(\Pi_i-\Pi_j)\rho_j$ which is the spillover arising from differences in bank regulation across host countries. Adding multinational-year fixed effects is equivalent to subtracting for each variable the average across all establishments in a given point in time. For the bank regulation spillover term, this can written as follows:

    \begin{equation*}
    \begin{aligned}
    &\sum_{j=1}^{n}(\Pi_i-\Pi_j)\rho_j-\frac{1}{n}\sum_{k=1}^{n}\bigg(\sum_{j=1}^{n}(\Pi_k-\Pi_j)\rho_j\bigg)\\
    &=\Pi_i\sum_{\mathclap{\substack{j=1\\j\neq i}}}^{n}\rho_j-\sum_{\mathclap{\substack{j=1\\j\neq i}}}^{n}\Pi_j\rho_j-\frac{1}{n}\sum_{j=1}^{n}\bigg(\Pi_j\sum_{\mathclap{\substack{k=1\\k\neq j}}}^{n}\rho_k-(n-1)\rho_j\bigg) \\
    &=\Pi_i\sum_{\mathclap{\substack{j=1\\j\neq i}}}^{n}\rho_j-\sum_{\mathclap{\substack{j=1\\j\neq i}}}^{n}\Pi_j\rho_j-\frac{1}{n}\bigg(\Pi_i\bigg[\sum_{\mathclap{\substack{j=1\\j\neq i}}}^{n} \rho_k-(n-1)\rho_i\bigg] + \sum_{\mathclap{\substack{j=1\\j\neq i}}}^{n} \Pi_j\bigg[ \sum_{\mathclap{\substack{k=1\\k\neq j}}}^{n} \rho_k - (n-1)\rho_j\bigg]\bigg) \\
    &=\frac{\Pi_i}{n}\bigg[n\sum_{\mathclap{\substack{j=1\\j\neq i}}}^{n}\rho_j-\sum_{\mathclap{\substack{j=1\\j\neq i}}}^{n}\rho_j-(n-1)\rho_i\bigg]-\frac{1}{n}\sum_{\mathclap{\substack{j=1\\j\neq i}}}^{n}\bigg[\Pi_j\rho_j-\Pi_j\rho_j(n-1) +\Pi_j\sum_{\mathclap{\substack{k=1\\k\neq j}}}^{n}\rho_k\bigg] \\
    &=\frac{\Pi_i}{n}\bigg[(n-1)\underbrace{\sum_{j=1}^{n}\rho_j}_{=1} \bigg]-\frac{1}{n}\sum_{\mathclap{\substack{j=1\\j\neq i}}}^{n}\bigg[\Pi_j\underbrace{\sum_{j=1}^{n}\rho_j}_{=1}\bigg] \\
    &=\Pi_i-\frac{\Pi_i}{n}-\frac{1}{n}\sum_{\mathclap{\substack{j=1\\j\neq i}}}^{n}\Pi_j \\
    &=\Pi_i-\frac{1}{n}\sum_{j=1}^{n}\Pi_j \\
    \end{aligned}
    \end{equation*}

    The last equality is equal to the level of bank regulation at the host country minus the average across all establishments of the multinational at a given year. Hence, when we include multinational-year fixed effects the remaining variation of the debt shift variable is multicollinear to its respective "domestic" variable. The result above also holds for the tax and the interaction between tax and regulation spillover variables. Consequently, equation \ref{eq:regression benchmark} is both the true empirical counterpart of our theoretical expression \ref{eq:optimal leverage in theory} and a robust regression against omitted variable bias.

	\section{Data} \label{sec:data}	
	\subsection{Bank regulation} \label{subsec:MPI}
	
	We rely on the efforts of \cite*{barth2013bank} for our measures of bank regulation. This work introduces the fourth of a series of surveys sponsored by the World Bank \footnote{Previous surveys are described and analyzed in \cite{barth2001regulation}, \cite{barth2004bank}, \cite{barth2008bank}}. Since our firm level dataset starts in 2007 we can only use the last two surveys which provide information for 2006 and 2011 across 142 and 125 countries, respectively. The measures track bank regulatory and supervisory policies with indices that allow comparison across countries.

     In the following we describe the five indices that are relevant to our analysis. We add in parentheses the indices' numbers from the original dataset for reference:
      \begin{itemize}
     	\item \textit{Restriction on banking activities} (index I.IV): this index captures the extent to which banks may engage in securities, insurance and real estate activities (higher values indicate more restrictive). Restrictions on activities imply banks are left with more resources to lending activities. Holding everything else constant, a tightening in restrictions on activities should increase credit supply and decrease interest rates. In our model this effect is represented by $\phi<0$. Hence, the expected sign of the effect on leverage for this variable is positive, that is, $\beta_1>0$.
     	\item \textit{Financial conglomerates restrictiveness} (index II.IV): this index tracks restrictions on banks' ownership of nonfinancial firms and on non-bank firms owning banks (higher values indicate more restrictive). Since equity investment in non-financial firms is restricted, banks have to rely more on debt instruments. Holding everything else constant, a tightening in restrictions on financial conglomerates should increase credit supply and decrease interest rates. In our model this effect is represented by $\phi<0$. Hence, the expected sign of the effect on leverage for this variable is positive, that is, $\beta_1>0$.
     	\item  \textit{Capital regulatory strigency} (index IV.III): this index measures regulatory requirements for the amount of capital banks must hold and the quality of the same. (higher values indicate greater stringency). Higher capital stringency implies greater cost of capital. Holding everything else constant, a tightening in capital strigency should increase interest rates, as funding become more costly. In our model this effect is represented by $\phi>0$. Hence, the expected sign of the effect on leverage for this variable is negative, that is, $\beta_1<0$.
     	\item \textit{Official supervisory power} (index V.I): this index capture whether the supervisory authorities have the authority to take specific actions to prevent and correct problems (higher values indicate greater power). A greater power to intervene creates costs associated with compliance. Holding everything else constant, a tightening in official supervisory power should increase interest rates as banking activities become more costly. In our model this effect is represented by $\phi>0$. Hence, the expected sign of the effect on leverage for this variable is negative, that is, $\beta_1<0$.
     \end{itemize}
	
	  We matched the information from the last survey to the firm level data for the year of 2011 and we use the results from the previous survey for 2007 \citep{barth2008bank}. We interpolate the indices for the years in between.  	
	  	
 	\subsection{Firm level} \label{subsec:firm}
	Firm level data and ownership information are obtained from the Orbis database compiled by the Bureau Van Dijk. The dataset consist of worldwide accounting and ownership information on both private and public owned companies.  Ownership is considered if one firm owns more than 50\% of another firm's shares in a given year. The latter firm is a subsidiary. If a company owns one or more firms but is itself owned by another, this firm is called an intermediate firm and for all purposes of this study is also considered as a subsidiary. A parent firm is the ultimate owner of a group, that is, a firm that owns one or more companies but none of its shareholders have more than 50\% of its shares. Our definition of ownership is chosen to guarantee that we are studying parent firms that have full control of their subsidiaries. As discussed in section \ref{sec:strategy}, this restriction on what constitutes ownership is important for our identification strategy.
	
	Given that our identification strategy depends on variation of exposure to bank regulation across firms of a corporate group, we restrict our analysis to multinationals. Banks, insurance and financial related companies\footnote{We keep only firms with the Orbis type "C" identifier.} and firms in the utility sector\footnote{Two digit NACE codes from 35 to 39.} are also excluded since their capital structure decisions are constrained by regulation. When available, we use unconsolidated balance sheet information. When only consolidated balance sheet is available we keep the observation from firms that are neither a parent or an intermediate company. This procedure ensures that we are not double counting and retain most of the observations. We also exclude firms with zero or negative assets and after constructing the leverage measures, we exclude observations where leverage is above one or below zero as those are likely to be erroneous. Moreover, firms with negative net worth are likely to be credit constrained and therefore not able to choose their capital structure optimally.
	
		\begin{small}
		{\setstretch{1.0}
			\begin{longtable}{lrrrrrr}\\
				\label{tab:number of firms}\\
				\multicolumn{7}{c}{Table \ref{tab:number of firms} - Subsidiary firms' country distribution}\\
				\hline \hline \addlinespace Country & 2007 & 2008 & 2009 & 2010 & 2011 & Total  \\
				\endfirsthead
				\multicolumn{7}{c}{Table \ref{tab:number of firms} - Subsidiary firms' country distribution}\\
				\hline \hline \addlinespace Country & 2007 & 2008 & 2009 & 2010 & 2011 & Total  \\
				\hline \addlinespace \endhead
				\hline
				\multicolumn{7}{r}{{\textit{(Continued)}}}\\ \endfoot
				\\ 	
				\endlastfoot
				\primitiveinput{../../../../output/tables/summary/number_firms_table.tex}
				\hline 			
			\end{longtable}	
		}
	\end{small}
	
	The final sample consists of 377,999 firms from 2007 to 2011 resulting in 1,084,023 firm-year observations. Each firm belongs to one of the 56,702 multinational groups in our sample. Table \ref{tab:number of firms} provides information on the amount of firms per host country. We track firms in 54 countries, consequently, our analysis consists in comparing differences in bank regulation across those 54 countries.
	
	Summary statistics are shown in table \ref{tab:summary}. \textit{Financial leverage} is defined as the ratio of total non-equity liabilities to total assets. Among the control variables \textit{tangibility} is the ratio of fixed assets to total assets. We use the \textit{logarithm of fixed assets} to proxy for firm size. \textit{Profitability} is the ratio of earning before interest, tax, depreciation and amortization (EBITDA) to total assets. All firm-level controls are winsorized at the 1st and 99th percentile to remove the effect of outliers. Finally, \textit{tax rate} is taxes and other mandatory contributions after accounting for deductions and exemptions to total commercial profit obtained from the World Bank Doing Business indicators.
	
		\begin{small}
		{\setstretch{1.0}
			\begin{longtable}{lrrrrrr}\\
				\label{tab:summary}\\
				\multicolumn{7}{c}{Table \ref{tab:summary} - Summary statistics of benchmark panel}\\
				\hline \hline \addlinespace  & Obs. & Mean & SD & Min & Med & Max  \\
				\endfirsthead
				\multicolumn{7}{c}{Table \ref{tab:summary} - Summary statistics of benchmark panel}\\
				\hline \hline \addlinespace  & Obs.  & Mean & SD & Min & Med & Max  \\ \hline  \endhead
				\hline
				\multicolumn{7}{r}{{\textit{(Continued)}}}\\ \endfoot
				\addlinespace
				\multicolumn{7}{p{15cm}}{{Notes: The sample period is 2007-2011. The number of observations is 1,084,023.
						Financial leverage is trimmed at a maximum value of 1 and a minimum of 0. Firm variables are winsorized at the 1\% to minimize the impact of outliers. See Table \ref{tab:definition} for variable definitions
						and sources.}}\\ 	
				\endlastfoot
				\primitiveinput{../../../../output/tables/summary/summary.tex}
				\hline 			
			\end{longtable}	
		}
	\end{small}		
	
	\section{Results} \label{sec:result}
	 Table \ref{reg:benchmark} presents our main results following regression equation \ref{eq:regression benchmark} from section \ref{sec:strategy}. All regressions are estimated with multinational-year and country fixed effects and include the \textit{tax rate} variable and the interaction term between the \textit{tax rate} and bank regulation variables. Through columns 1 to 4 we include one bank regulation index at time, in an attempt to avoid any multicollinearity among indices. Column 5 shows the result for a regression that includes all the indices to check if estimates from columns 1 to 4 are not biased due to variable omission.
	
	 \afterpage{
	 	\newgeometry{left=0.6in,right=0.6in}		
	 		\begin{small}
	 			{\setstretch{1.0}
	 				\begin{longtable}{lcccccc}\\
	 					\label{reg:benchmark}\\
	 					\multicolumn{7}{c}{Table \ref{reg:benchmark} - Regression results, bank regulation and firms' capital structure}\\
	 					\multicolumn{7}{c}{(\textit{Dependent variable}: financial leverage)}
	 					\\ \hline \hline \addlinespace
	 					Model & (1) & (2) & (3) & (4) & (5) & (6) \\  \endfirsthead
	 					\multicolumn{7}{c}{Table \ref{reg:benchmark} - Regression results, bank regulation and firms' capital structure }\\
	 					\multicolumn{7}{c}{(\textit{Dependent variable}: financial leverage)\textit{(Continued)}}
	 					\\ \hline \hline \addlinespace Model & (1) & (2) & (3) & (4) & (5) & (6) \\ \hline \\ \endhead
	 					\hline
	 					\multicolumn{7}{r}{{\textit{(Continued)}}}\\ \endfoot \addlinespace
	 					\multicolumn{7}{p{17cm}}{{Notes: The estimates in this table come from estimating equation \ref{eq:regression benchmark} over the period 2007-2011. Observations are at the firm-year level. The dependent variable is the ratio of non-equity liabilities to total assets. In all columns, Firm controls refer to firm variables (Profitability,
	 							Tangibility and Log of fixed assets). Standard errors are reported in
	 							parentheses and are clustered at the multinational level. *** indicates significance at the 1\% level, ** at the 5\% level and * at the
	 							10\% level.}}\\ 	
	 					\endlastfoot
	 					\primitiveinput{../../../../output/tables/regressions/benchmark_table.tex}
	 					\hline 			
	 				\end{longtable}	
	 			}
	 		\end{small}
	 	\restoregeometry}
	
	  In regression (1) the estimated coefficient for \textit{restriction on banking activities}  is positive and statistically significant, as expected. Restrictions on activities force banks to allocate more of their resources to lending. Enhanced competition on the loan market decreases interest rate charged to firms which take advantage of better condition to leverage. The effect is also economically relevant. Consider the average level of \textit{tax rate} in our sample and the coefficient of the interaction term between \textit{restriction on banking activities} and \textit{tax rate}, that is, 0.47 and 0.018, respectively. The effect of an increase in \textit{restriction on banking activities} by 1.85 (one standard deviation) on leverage is a 0.75 percentage point gain on average \footnote{=1.85$\times$0.021-1.85$\times$0.036$\times$0.47}. If we would move a firm from the country with the lowest level of restriction to the most restrictive country, we would expect leverage to increase by 3.3 percentage points on average.  When considering the joint estimation with other indices in column (5) the effect is estimated somewhat weaker and with lower statistical significance.
	
	  Regression (2) shows that the coefficient of \textit{financial conglomerates restrictiveness} is estimated with the expected positive sign and different from zero at the 1\% significance level. However, in regression (5) the effect looses its statistical significance. Table \ref{tab:correl} shows that the index for restrictions on financial conglomerates is significantly correlated to the other three indices. Hence, the result in the second column is most likely driven by a omitted variable bias. We hypothesize that this regulatory measure is not likely to be binding in most countries. The optimal amount of nonfinancial firms' equity that banks want to hold can be already satisfied by the regulation. If that is true, restrictions on financial conglomerates would not change banks' operations and consequently the cost of debt charged to firms would remain unchanged.
	
	  The estimated coefficient of \textit{capital regulatory stringency} in column (3) is negative, as expected, and statistically significant. In regression (5) the effect of this variable retains the magnitude. Capital requirements are expected to increase banks' costs which in our model yields lower firms' leverage. To evaluate the economical significance of this measure, consider again the average level of tax rate in our sample. The coefficient of the interaction term between \textit{capital regulatory stringency} and \textit{tax rate} is estimated to be 0.018 in column (5). The effect of an increase in the \textit{capital regulatory stringency} variable by 1.67 (one standard deviation) on leverage is a decrease of 0.42 percentage point on average. A firm moving from the least to the most stringent country with respect to capital requirements would have a leverage ratio 1.8 percentage points lower on average.
	
	  Regression (4) shows that the coefficient of \textit{official supervisory power} is not statistically different from zero. However, in column (5) the coefficient gains significance and the expected sign. This result is in line with our reasoning that higher costs of compliance associated to greater supervision power are transmitted to firms thorough higher cost of debt. The average effect of an increase in \textit{official supervisory power} by 1.44 (one standard deviation) on leverage is a 0.55 percentage point decline, considering again the average level of tax rate in our sample and an estimated coefficient of the interaction term between this variable and \textit{tax rate} at 0.018. In our hypothetical experiment of moving a firm from the lowest to the greatest supervision power, we expect leverage to be 4.0 percentage points lower on average.
	
	  Finally, in regression (6) we add the following firm-level controls: \textit{profitability}, \textit{tangibility} and the \textit{log of fixed assets}. Those firm specific characteristics are important explanatory variables of financial leverage, even within a multinational group. Moreover, those characteristics could be correlated to the bank regulation measures and consequently biasing our estimates. Nevertheless, the results are virtually unchanged from regression (5). Statistical significance increases for the \textit{restriction on banking activities} variable and decreases for the \textit{capital regulatory stringency} (although still significant at the 10\% level). The magnitude of the estimated coefficients remain equal to regression (5) for \textit{capital regulatory stringency} and \textit{official supervisory power} while it increases for \textit{restriction on banking activities}.
	   	    			   		
\subsection{Extension} \label{sec:sub_index}

Thus far, the analysis was restricted to the impact of four main indices from \cite{barth2013bank} on firms' financial leverage. Those indices were chosen as they are the best summary for bank regulation and supervision. However, the components of each summary index may have different effects on leverage. In table \ref{reg:sub_index} we break down three of the main indices in order to investigate the impact of each of their components on leverage.

\afterpage{
\begin{small}
	{\setstretch{1.0}
		\begin{longtable}{lccc}\\
			\label{reg:sub_index}\\
			\multicolumn{4}{c}{Table \ref{reg:sub_index} - Bank regulation and firms' capital structure: extension}\\
			\multicolumn{4}{c}{(\textit{Dependent variable}: financial leverage)}
			\\ \hline \hline \addlinespace
			Model & (1) & (2) & (3)     \\  \endfirsthead
			\multicolumn{4}{c}{Table \ref{reg:sub_index} - Bank regulation and firms' capital structure: sub-indexes effect}\\
			\multicolumn{4}{c}{(\textit{Dependent variable}: financial leverage)\textit{(Continued)}}
			\\ \hline \hline \addlinespace Model & (1) & (2) & (3)   \\ \hline \\ \endhead
			\hline
			\multicolumn{4}{r}{{\textit{(Continued)}}}\\ \endfoot
			\addlinespace
			\multicolumn{4}{p{12cm}}{{Notes: The estimates in this table come from estimating equation \ref{eq:regression benchmark} over the period 2007-2011. Observations are at the firm-year level. The dependent variable is the ratio of non-equity liabilities to total assets. In all columns, Firm controls refer to firm variables (Profitability,
					Tangibility and Log of fixed assets). Standard errors are reported in
					parentheses and are clustered at the multinational level. *** indicates significance at the 1\% level, ** at the 5\% level and * at the
					10\% level.}}\\  	
			\endlastfoot
			\primitiveinput{../../../../output/tables/regressions/sub_indexes_table.tex}
			\hline 			
		\end{longtable}	
	}
\end{small}}

In regression (1), \textit{restriction on bank activities} is broken in three variables: \textit{restrictions on securities activities}, \textit{restrictions on insurance activities} and \textit{restrictions on real estate activities}. Restriction on securities appears with a positive sign, implying a substitution effect to providing loans. Restriction on insurance gets a negative sign, suggesting complementarity. On the one hand, restrictions on other activities reduce banks' options to lending, implying a positive sign of the estimated coefficient on leverage. On the other hand, there might be some activities, such as insurance, that are complementaries services to lending which could reduce the cost of debt to firms. The coefficient for Restriction on real estate activities is not statistically different from zero.

Next, we consider the components that form the \textit{financial conglomerates restrictiveness}: \textit{restriction on banks owning nonfinancial firms}, \textit{restriction on nonfinancial firms owning banks} and \textit{restriction on financial firms owning banks}. Regression (2) shows that none of the indices are statistically significant different from zero.

In regression (3), we evaluate the two components of the \textit{capital regulatory stringency} index: \textit{overall capital stringency} and \textit{initial capital stringency}.  As explained in \cite*{barth2001regulation}, the former index captures if regulatory capital is considered simply as an accounting concept or if the requirements actually reflect market-value risk elements. \textit{Initial capital stringency} tracks the source of capital and if the sources are verified by authorities. Not surprisingly, both coefficients are estimated with the expected negative sign. However, only the coefficient for \textit{overall capital stringency} is statistically different from zero.
    	
	\section{Conclusion} \label{sec:conclusion}
		
	This paper shows that firms' capital structure is sensitive to bank regulation, specifically restrictions on bank activities, capital regulatory policies and the degree of official supervisory power. We propose a simple theoretical model where changes in regulatory framework impact banks' cost of providing loans that ultimately are transmitted to firms via their cost of debt. Using data from multinationals in 54 countries from 2007 to 2011 we are able to identify this transmission channel since the data allow us to control for unobservable factors on the demand side that could be correlated to leverage and regulatory policies. Therefore, we use the cross-country variation in bank regulation across subsidiaries within one multinational to identify their effects on capital structure.
	
	Overall, our results suggest that firms' respond to changes in bank regulation. Consequently, this study points to the importance of bank supervision authorities to track the real sector when defining policies. Although the effect of most policies on firms' leverage has the same direction as intended by the supervisory authorities, adverse effects are possible. For instance, our results show that higher restrictions on bank activities are associated with higher leverage in the nonfinancial sector. While those restrictions are intended to enhance financial stability, the increase of overall leverage in the real sector can create risks to stability.
	
	Finally, even though we control for country specific fixed effects in our analysis, these country characteristics might not be fixed in time. If bank regulation is correlated with time-varying variables at the country level, our results could be biased. Hence, an interesting extension to our work is to obtain a dataset that would allow to control for those unobservable characteristics at the country level for each year.

		
\singlespacing
\bibliography{../../bibliography/references}
\bibliographystyle{../../bibliography/te}
	
	
	
%	
	
%	\onehalfspacing
	
	
\section*{Appendix }
\label{sec:appendixa}

\setcounter{table}{0}
\renewcommand{\thetable}{A.\arabic{table}}

	\addcontentsline{toc}{section}{Appendix A}

	\afterpage{	
	\newgeometry{left=1in,right=1in,top=1.2in,bottom=1.2in}	
	\begin{landscape}
	\begin{small}
	{\setstretch{1.0}
		% Please add the following required packages to your document preamble:
% \usepackage{graphicx}
%\begin{longtable}[]
	\centering
	%\resizebox{\textwidth}{!}{%
		\begin{longtable}{p{1.7in}p{2.6in}p{1.7in}}
				\label{tab:definition}\\
			\multicolumn{3}{c}{Table \ref{tab:definition} - Variable definitions and data sources}\\
			\hline 
			Variable      & Definition & Source \\
			\hline \endfirsthead
			
				\multicolumn{3}{c}{Table \ref{tab:definition} - Variable definitions and data sources \textit{(Continued)}}\\
			\hline 
			Variable      & Definition & Source \\
			\hline \endhead
			
			\hline
			\multicolumn{3}{r}{{\textit{Continued}}}\\ 
			\endfoot
			\hline
			\endlastfoot
			Financial leverage      & Ratio of non-equity liabilities to total assets & Orbis \\
		
			Reserve req. on local currency & Average of quarterly index of changes in reserve requirements on local currency & \cite{cerutti2017changes}\\
			Capital requirement & Average of quarterly index of changes in capital requirements & \cite{cerutti2017changes}\\
			Reserve req. on local currency spillover & Sum of international reserve requirement on local currency differences weighted by local asset shares& \cite{cerutti2017changes}\\
			Capital requirement spillover & Average of quarterly index of changes in capital requirements  & \cite{cerutti2017changes}\\
			Tax rate & Taxes and other mandatory contributions after accounting  for deductions and exemptions to total commercial profit & World Bank Doing Business indicators\\
			Tax rate spillover & Sum of international corporate tax rate  differences weighted by local asset shares& World Bank Doing Business indicators\\
			Tangibility& Ratio of fixed assets to total assets & Orbis\\
			Log of fixed assets& logarithm of fixed assets & Orbis\\
			Profitability& Ratio of EBITDA to total assets & Orbis\\
			Risk & Standard deviation of the firm's ratio of EBITDA to total assets over the period 2008-2014& Orbis\\
			Opportunity & Median of the annual growth rate of sales per country and industry& Orbis\\
			Private credit to GDP & Ratio of credit to the private sector to GDP & World Bank indicators\\	
			Inflation & Annual log change in the CPI & World Bank indicators\\
			GDP growth rate &Annual percentage change in the GDP & World Bank indicators\\
			 Policy rate &Central Bank Policy Rate, Percent per annum & IMF International Financial Statistics \\
			Exchange rate risk &Annual (December) index of exchange rate risk &International Country Risk Guide\\
			Law and order &Annual (December) index of law and order &International Country Risk Guide\\
			Political risk &Annual (December) index of political risk &International Country Risk Guide\\
		\end{longtable}%
	%}
%\end{longtable}
	}
\end{small}
\end{landscape}
\restoregeometry}

	\afterpage{	
		\newgeometry{left=2in,right=2in,top=1.2in,bottom=1.2in}	
	\begin{landscape}
	\begin{small}
		{\setstretch{1}
			\begin{longtable}{p{4cm}p{3cm}p{3cm}p{3cm}p{3cm}p{3cm}}\\
				\label{tab:correl}\\
				\multicolumn{6}{c}{Table \ref{tab:correl} - Correlation matrix of main regressors}\\
				\hline \hline \addlinespace	
				\primitiveinput{../../../../output/tables/summary/correl.tex}
				\hline 			
			\end{longtable}	
		}
	\end{small}
\end{landscape}
\restoregeometry}


\end{document} 